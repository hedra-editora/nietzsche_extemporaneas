%\newcommand{\linhalayout}[2]{{\tiny\textbf{#1}\quad#2\par}}
\newcommand{\linha}[2]{\ifdef{#2}{\linhalayout{#1}{#2}}{}}

\begingroup\tiny
\parindent=0cm
\thispagestyle{empty}

\textbf{copyright}\quad                      	  {}\\
\textbf{edição brasileira©}\quad			      {Hedra \the\year}\\
\textbf{tradução©}\quad			 			 {André Luís Mota Itaparica}\\
\textbf{organização©}\quad			 		 {André Luís Mota Itaparica}\\
\textbf{introdução©}\quad			 		 {André Luís Mota Itaparica}\\
%\textbf{ilustração©}\quad			 		 {copyrightilustracao}\medskip

\textbf{título original}\quad			 	 {\emph{Vom Nutzen und Nachteil der Historie für das Leben}}\\
\textbf{edição consultada}\quad			 {\emph{Sämtliche Werke.} De Gruyter, 1988}\\
\textbf{primeira edição}\quad			 	 {1874 editora?, [Hedra, 2014]}\\
%\textbf{agradecimentos}\quad			 	 {agradecimentos}\\
%\textbf{indicação}\quad			 	      {indicacao}\medskip

\textbf{edição}\quad			 			 {Jorge Sallum}\\
\textbf{coedição}\quad			 			 {Suzana Salama}\\
\textbf{assistência editorial}\quad			 {Julia Murachovsky}\\
\textbf{revisão}\quad			 			 {}\\
\textbf{preparação}\quad			 		      {preparacao}\\
\textbf{iconografia}\quad			 		 {iconografia}\\
\textbf{capa}\quad			 				 {}\\
\textbf{imagem da capa}\quad			 	     {imagemcapa}\medskip

\textbf{ISBN}\quad			 				 {ISBN}\smallskip

\hspace{-5pt}\begin{tabular}{ll}
\textbf{conselho editorial}        & Adriano Scatolin,  \\
							& Antonio Valverde,  \\
							& Caio Gagliardi,    \\
							& Jorge Sallum,      \\
							& Ricardo Valle,     \\
							& Tales Ab'Saber,    \\
							& Tâmis Parron      
\end{tabular}
  
\begin{minipage}{6cm}
\textbf{Dados Internacionais de Catalogação na Publicação (CIP)\\
(Câmara Brasileira do Livro, SP, Brasil)}

\textbf{\hrule}\smallskip

Nietzsche, Friedrich, 1844-1900\\

Crime 1877-1879 / Luiz Gama; organização, introdução, estabelecimento de texto,
comentários e notas Bruno Rodrigues de Lima. 1. ed. São Paulo, \textsc{sp}: Editora Hedra, 2023.
(Obras completas; volume 7). Bibliografia.\\

\textsc{isbn} isbn aqui\\

1. Abolicionistas -- Brasil 2. Crimes (Direito penal) 3. Escravidão 4. Falsificação 5. Homicídio I. Lima, Bruno Rodrigues de. II. Título. III. Série.\\

23-164687 \hfill \textsc{cdu}: 343.232

\textbf{\hrule}\smallskip

\textbf{Elaborado por Tábata Alves da Silva (CRB-8/9253)}\\

\textbf{Índices para catálogo sistemático:}\\
1. Crimes: Direito penal 343.232
\end{minipage}

\vfill 

\bigskip
\textit{Grafia atualizada segundo o Acordo Ortográfico da Língua\\
Portuguesa de 1990, em vigor no Brasil desde 2009.}\\

\vfill
\textit{Direitos reservados em língua\\ 
portuguesa somente para o Brasil}\\

\textsc{editora hedra ltda.}\\
Av.~São Luís, 187, Piso 3, Loja 8 (Galeria Metrópole)\\
01046--912 São Paulo \textsc{sp} Brasil\\
Telefone/Fax +55 11 3097 8304\\\smallskip
editora@hedra.com.br\\
www.hedra.com.br\\
\bigskip
Foi feito o depósito legal.

\endgroup
\pagebreak


% CREDITOS ANTIGOS
%    \newcommand\copyrightlivro{Hedra}
%\newcommand\copyrighttraducao{André Luís Mota Itaparica} % Copy de tradução
%%\newcommand\copyrightorganizacao{}
%%\newcommand\copyrightilustracao{}
%%\newcommand\copyrightintroducao{}
%\newcommand\titulooriginal{Vom Nutzen und Nachteil der Historie für das Leben}
%\newcommand\edicaoconsultada{\emph{Sämtliche Werke.} De Gruyter, 1988}
%\newcommand\primeiraedicao{1874}
%%\newcommand\agradecimentos{}
%%\newcommand\indicacao{}
%\newcommand\ISBN{978-85-7715-545-3}
%\newcommand\ano{2014}
%\newcommand\edicao{Jorge Sallum}
%\newcommand\coedicao{José Eduardo Góes}
%\newcommand\assistencia{Equipe Hedra}