\vspace*{7cm}

\noindent\textbf{Friedrich Nietzsche} (Röcken, 1844---Weimar, 1900), filósofo 
e filólogo alemão, foi crítico mordaz da cultura ocidental 
e um dos pensadores mais influentes da modernidade. Descendente de pastores 
protestantes, opta no entanto por seguir carreira acadêmica. 
Aos 25 anos, torna-se professor de letras clássicas na Universidade 
da Basileia, onde se aproxima do compositor Richard Wagner. Serve 
como enfermeiro voluntário na guerra franco-prussiana, mas contrai 
difteria, a qual prejudica sua saúde definitivamente. Retorna a 
Basileia e passa a frequentar mais a casa de Wagner. Em 
1879, devido a constantes recaídas, deixa a universidade e passa a 
receber uma renda anual. A partir daí assume uma vida errante, 
dedicando-se exclusivamente à reflexão e à redação de suas obras, 
dentre as quais se destacam: \textit{O nascimento da tragédia} (1872), 
\textit{Considerações Extemporâneas} (1873--1874), \textit{Assim falava Zaratustra} 
(1883--1885), \textit{Para além do bem e mal} (1886), \textit{A genealogia da moral} 
(1887) e \textit{O anticristo} (1895). Em 1889, apresenta os primeiros 
sintomas de problemas mentais, provavelmente decorrentes de sífilis. Falece em 1900.


\noindent\textbf{Sobre a utilidade e a desvantagem da história para a vida}
 (1874), a segunda das quatro considerações extemporâneas, foi definida pelo
 autor, em sua autobiografia, \textit{Ecce Homo}, como sendo o tratado que:
 ``traz à luz o que há de perigoso, corrosivo e envenenador da vida na nossa
 forma de prática científica: a vida \emph{doente} dessa engrenagem e mecanismo 
 desumanos, da `\emph{im}pessoalidade' do trabalhador, da falsa economia da 
 `divisão do trabalho'''. Nessa obra, ``o `sentido histórico', do qual nosso século se
 orgulha, foi pela primeira vez reconhecido como doença, como signo típico da
 decadência''. 


\textbf{André Luis Mota Itaparica} é doutor em filosofia pela Universidade  
de São Paulo (\textsc{usp}) e professor da Universidade Federal do Recôncavo da Bahia 
(\textsc{ufrb}). É autor de \textit{Nietzsche: Estilo e moral} (Discurso/Unijuí, 2001),  
\textit{Verdade e linguagem em Nietzsche} (Edufba, 2014), numerosos artigos e contribuições 
a obras sobre Nietzsche, Crítica da Moral, Idealismo, Realismo, Natureza, Cultura etc.

