%!TEX root=LIVRO.tex 

\part[Sobre a utilidade e a desvantagem]{Sobre a utilidade e a desvantagem da história para a vida}


\chapter*{Prefácio}{}{}\label{prefuxe1cio}

``Aliás,\footnote{Esta tradução baseia-se na edição crítica
  Colli-Montinari. \emph{Sämtliche Werke.} Kritische Studienausgabe. Bd.
  1. Berlim/Munique: Walter de Gruyter, 1988. Quando necessário,
  comparamos nossa tradução com as seguintes traduções:
  \emph{Unfashionable Observations}. Trad. Richard Gray. Standford,
  California: Standford University Press, 1995; \emph{Untimely
  meditations}. Trad. R. J. Hollingdale. Cambridge: Cambridge University
  Press, 1997; \emph{Segunda consideração intempestiva}. Da
  utilidade e desvantagem da história para a vida. Trad. Marco
  Antônio Casanova. Rio de Janeiro:
  Relume Dumará, 2003; \emph{Considérations inactuelles \textsc{i} et \textsc{ii}}. Trad.
  Pierre Rusch. Paris: Gallimard, 1990. Para as notas do tradutor, foram
  úteis o aparato crítico da edição Colli-Montinari (Bd. 14) e as notas
  das edições mencionadas.} odeio tudo aquilo que apenas me instrui, sem
aumentar ou estimular diretamente minha ação''.\footnote{\emph{Carta de Goethe
  a Schiller,} de 19/12/1798. Nessa passagem, Goethe faz um comentário
  sobre a \emph{Antropologia} de Kant.} Com essas palavras de Goethe,
que expressam um resoluto \emph{ceterum censeo},\footnote{Palavras com
  que Catão, o Velho, finalizava seus discursos durante as Guerras
  Púnicas: ``\emph{Aliás, sou da opinião} de que Cartago deve ser
  destruída''.} pode ter início nossa consideração sobre o valor e
desvalor da história. Nela se mostra por que o ensinamento sem vivência,
por que o saber que entorpece a ação, por que a história como fútil
excesso de conhecimento e luxo devem, nas palavras de Goethe, ser odiados
-- porque ainda nos falta o mais necessário, e porque o supérfluo é
inimigo do necessário. É certo que precisamos da história, mas de
maneira diferente do que dela precisa o ocioso mimado no jardim do
saber, que pode nobremente olhar com desdém para nossas toscas
necessidades e nossas rudes carências. Isto é, precisamos da história
para a vida e para a ação, e não para uma cômoda renúncia da vida e da
ação, ou ainda para a edulcoração da vida egoísta ou do ato covarde e
vil. É apenas na medida em que a história serve à vida que queremos a
ela servir; mas existe um grau, no exercício e na valorização da história,
em que a vida fenece e se degenera: um fenômeno que experimentamos
agora, tão necessário quanto doloroso possa ser, como um estranho
sintoma de nossa época.\label{sintomadenossaepoca}

Esforcei-me em descrever uma sensação que me torturava com frequência;
vingo-me dela tornando-a pública. Talvez essa descrição leve alguém a me
dizer que também conhece essa sensação, mas que eu não a tinha sentido
de forma pura e originária o suficiente, e por isso não me tenha
expressado com a necessária segurança e maturidade da experiência. É o
que talvez me diriam um e outro; a maioria, contudo, me dirá que essa é
uma sensação distorcida, inatural, execrável e simplesmente ilícita, e
ainda que eu, com essa sensação, mostrei-me indigno da tendência
histórica atual, observada, como se sabe, há duas gerações entre os
alemães. Agora, de todo modo, a fisiografia de minha sensação antes
auxilia que prejudica o decoro geral, por oferecer a muitos a
oportunidade de defender a tendência atual. De minha parte, ganho algo
que para mim tem mais valor do que o decoro --- instruir-me e ser
corrigido publicamente sobre nossa época.

Extemporânea é esta consideração também porque nela procuro compreender
a cultura histórica, da qual nossa época, com razão, se orgulha, como
infortúnio, privação e carência; porque eu, além disso, acredito que
todos nós padecemos de uma febre histórica devastadora e devemos, ao
menos, reconhecer que dela padecemos. Mas, se Goethe tinha razão em dizer que
com nossas virtudes construímos ao mesmo tempo nossos erros,
e se, como todos sabem, uma virtude hipertrofiada --- como o sentido histórico
de nossa época parece ser --- pode corromper um povo tanto quanto um
vício hipertrofiado. Então, deixem que eu me manifeste. Para meu alívio,
também não devo esconder que cheguei por mim mesmo às experiências que
me provocavam aquela sensação torturante, só recorrendo a outros para
efeito de comparação; e que eu, somente na medida em que sou pupilo de
uma época mais antiga, ou seja, da grega, posso vir a ter, como um filho
da época atual, experiências tão extemporâneas. Para tanto, permito-me
confessar, até pela minha profissão de filólogo clássico: não saberia
que sentido teria a filologia clássica em nossos dias senão o de
intervir extemporaneamente, isto é, contra a época, sobre a época e a
favor de uma época futura.\label{epocafutura}

\chapter*{1}

Observe o rebanho a pastar: ele nada sabe do que é o ontem e o hoje;
saltita aqui e acolá, come, descansa, digere, novamente saltita, noite e
dia, dia após dia. Em resumo, preso ao seu prazer e desprazer, estancado
no instante, não se entristece nem se enfastia. Ver isso é difícil para
o homem, que se vangloria de sua humanidade perante o animal, mas
contempla enciumado a sorte deste --- pois o homem apenas quer, como o
animal, viver sem fastio e sem dor; mas o quer em vão, por não querer
como aquele. O homem pergunta ao animal: ``por que nada me diz de sua
sorte e apenas me fita?'' O animal quer responder e dizer: ``acontece que eu
sempre esqueço o que quero dizer'' --- mas já esquece essa resposta e
silencia, e o homem se espanta.\footnote{Passagem inspirada no poema
  ``Canto noturno de um pastor da Ásia'', de Giacomo Leopardi: ``Oh, meu
  rebanho, que repousa, eu louvo/ Tua alegria, que te poupa de
  conhecer/ Tua miséria. Ah, como te invejo!/ Não apenas porque pareces
  livre/ De quase toda dor/ Labuta, perda, o pior temor são esquecidos no
  instante/ Mais ainda porque o fastio nunca te tortura!/ Quando tu
  escolhes, na grama e na sombra, onde cochilar, e te sentes feliz.
  (\ldots{}) Se tu soubesses falar, te perguntaria:/ Diz-me, por que o
  animal,/ No descanso, na indolência,/ Tem alegria, enquanto eu me
  enfastio/ Assim que descanso?''. Tradução baseada na alemã de
  Robert Hamerling (\emph{Leopardi's Gedichte}. Hildburghausen, 1866),
  versão que, como mostrou Otto Friedrich Bollonow, Nietzsche
  provavelmente conheceu (``Nietzsche und Leopardi''. \emph{Zeitschrift
  für philosophische Forschung,} 26, 1972).}

No entanto, ele se espanta consigo mesmo, por não poder aprender a
esquecer e por sempre estar pendurado no passado: por mais distante e
rápido que possa correr, com ele corre um grilhão. É um milagre: o
instante, deslizando aqui e acolá, um nada antes e um nada depois,
retorna como um fantasma e tira a paz de um instante posterior.
Continuamente, uma folha se solta do papiro do tempo, cai e flutua --- e
de repente volteia e pousa no colo do homem. Então diz o homem: ``eu me
lembro'', e inveja o animal, que logo esquece e vê cada instante
efetivamente fenecer, afundar na noite e na névoa, extinguindo-se para
sempre. O animal vive de forma \emph{aistórica}: pois ele se absorve no
presente, como um número, sem restar uma estranha fração;\footnote{Erwin
  Rohde, ao ler as provas do livro, não gostou dessa expressão, que é,
  contudo, uma referência a Goethe: ``Eram homens sensatos, espirituosos
  e cheios de vida, que compreendiam muito bem que a soma de nossa
  existência, dividida pela razão, nunca é exata, restando sempre uma
  estranha fração'' (\emph{Os anos de aprendizado de Wilhelm Meister,}
  Livro 4. Rio de Janeiro: Ed. 34, 2006, p.\,266).} ele não sabe
dissimular, nada esconde e aparece em cada momento inteiramente como
aquilo que é; ele não sabe ser outra coisa senão sincero. O homem, ao
contrário, luta contra a crescente e pesada carga do passado: esta o
pressiona ou o enverga, sopesa seu passo como um fardo invisível e
obscuro, que ele pode negar como ilusório e que, entre seus semelhantes,
negaria com prazer para provocar inveja. Por isso ele se comove ao ver
o rebanho pastar ou, de forma mais próxima, vê a criança, que nada tem a
negar do passado, brincando entre a cerca do passado e do futuro em uma
cegueira abençoada, como se recordasse de um paraíso perdido. Mas a
brincadeira tem de acabar: já cedo, a criança é convocada a sair do
esquecimento. Então aprende a entender a palavra ``era'', aquela senha
que, junto com a luta, a dor e o fastio, leva o homem a se lembrar do
que, no fundo, é sua existência --- um imperfectivo que nunca se
perfaz.\footnote{Em alemão: \emph{ein nie zu vollendendes Imperfectum}.
  Traduzimos por imperfectivo para explicitar que Nietzsche faz uso aqui
  de um termo gramatical, referente aos tempos dos verbos que expressam
  ações passadas inacabadas.} Quando a morte enfim traz o seu ansiado
esquecimento, rouba também o presente e a existência, selando com
isso aquele conhecimento de que a existência é apenas um ininterrupto
``ter sido'', uma coisa que vive de negar-se e consumir-se, de
contradizer-se a si mesma.

Se uma felicidade, se a ânsia por uma nova felicidade tiver o sentido de
manter o vivente na vida e estimulá-lo a viver, então talvez nenhum
filósofo tenha mais razão que o cínico: pois a felicidade do animal,
como o cínico perfeito, é a prova viva da veracidade do cinismo. A menor
felicidade, quando ininterrupta e faz feliz, é incomparavelmente mais
feliz do que a maior felicidade episódica, que, como um capricho, como
uma ideia súbita desvairada, surge entre o desprazer, o desejo e a
carência. Tanto na maior como na menor felicidade, só uma coisa faz a
felicidade ser felicidade: a capacidade de esquecer ou, expresso de
forma erudita, a faculdade de sentir aistoricamente durante a
felicidade. Quem não sabe alojar-se no umbral do instante,
esquecendo-se de tudo que passou, quem não é capaz de manter-se em pé, 
como uma deusa Vitória, sem vertigem ou temor, nunca saberá o que
é a felicidade; e ainda pior: nunca fará algo que deixará outro feliz.
Pensem num exemplo extremo de um homem que não possuísse a faculdade de
esquecer, que fosse condenado a ver um devir em tudo: ele não acredita
mais no seu próprio ser, não acredita mais em si, vendo tudo fluir de um
ponto móvel a outro e se perdendo nessa correnteza do devir; por fim,
como o íntegro discípulo de Heráclito, ele quase sequer ousará apontar o
dedo.\footnote{O discípulo de Heráclito em questão é Crátilo. (Cf. Livro
  \textsc{iv} da \emph{Metafísica} de Aristóteles).} A toda ação pertence o
esquecimento: assim como pertence à vida de todo organismo não somente a
luz, mas também a escuridão. Um homem que sentisse tudo unicamente de
forma histórica seria parecido com alguém que tivesse abdicado do sono,
ou com o animal que devesse viver apenas em repetitiva ruminação.
Portanto, é possível viver, e até mesmo viver feliz, quase sem
lembranças, como mostra o animal; mas é totalmente impossível viver sem
o esquecimento. Ou, para me expressar sobre meu tema de forma mais
clara: \emph{existe um grau de insônia, de ruminação, de sentido
histórico, que prejudica o vivente e por fim o destrói, seja um homem,
um povo ou uma cultura}.

A fim de determinar esse grau e, por meio dele, o limite do que
deve ser esquecido, para que o passado não se torne o coveiro do
presente, se deveria saber exatamente quão grande é a \emph{força
plástica} de um homem, de um povo, de uma cultura, quero dizer, aquela
força que cresce a partir de si mesma, de transformar e incorporar o
passado e o estranho, de curar feridas, de substituir o que se perdeu e
reconstituir a partir de si formas arruinadas. Há homens que possuem tão
pouco dessa força que fenecem por uma única experiência, por uma única
dor, frequentemente até por uma única leve injustiça, como se sangrassem
até a morte por causa de um pequeno arranhão; há, de outro lado, aqueles
que pouco se abalam pelos mais violentos e tristes infortúnios da vida,
e mesmo pelas próprias ações malévolas, de sorte que no momento, ou logo
depois, alcançam uma bonança e uma espécie de consciência tranquila.
Quanto mais fortes são as raízes da natureza interior de um homem, mais
ela se apropriará do passado e o submeterá; e se se pensasse na natureza
mais poderosa e descomunal, então se reconheceria que para ela não
haveria limite do sentido histórico que lhe pudesse sobrepujar e
prejudicar; todo passado, próprio e alheio, seria recriado a partir de
si e introjetado no próprio sangue. Tal criatura sabe esquecer o que ela
não subjuga; tudo se esvanece, o horizonte fica completamente fechado, e
ela não é capaz de lembrar que existe, além desse horizonte, homens,
paixões, doutrinas, finalidades. E isto é uma lei universal: todo
vivente só pode tornar-se sadio, forte e fértil no interior de um
horizonte; ele é incapaz de trazer um horizonte para si e é muito
egoísta, por sua vez, para inserir seu olhar no interior de um horizonte
alheio, pois isso o adoece, debilitando-o, levando-o ao declínio. A
alegre serenidade, a boa consciência, o ato feliz, a confiança no
vindouro, tudo depende --- seja para um indivíduo como para um povo --- de
que haja uma linha que separe o visível e claro do obscuro e sombrio; de
que se saiba tanto esquecer direito e no tempo certo, quanto lembrar no
tempo certo; de que se perceba, com instinto forte, quando é necessário
sentir historicamente ou aistoricamente. Esta consideração convida o
leitor à seguinte sentença: \emph{o histórico e o aistórico são
igualmente necessários para a saúde de um indivíduo, de um povo e de uma
cultura.}

De início, façamos aqui uma observação: o saber e o sentido histórico de
um homem podem ser bastante limitados; seu horizonte, estreito como o de
um habitante de um vale alpino; ele pode julgar injustamente e cometer o
erro de considerar-se o primeiro a ter cada experiência e, apesar de
toda injustiça e de todo erro, permanecer com energia e saúde
insuperáveis, tirando proveito dessa visão; enquanto ao seu lado alguém
mais justo e instruído adoece e sucumbe, porque as linhas de seu
horizonte se expandem constante e incessantemente, porque ele não pode
livrar-se da rede suave de sua justiça e de sua verdade em direção ao
firme querer e desejar. Vimos, ao contrário, o animal que é totalmente
aistórico e que habita em um horizonte quase pontual, mas vive em uma
certa felicidade, ao menos sem fastio e dissimulação. Teremos de tomar a
capacidade de sentir, em um determinado grau, aistoricamente, como o
mais importante e originário, na medida em que nisso repousar o
fundamento sobre o qual pode crescer algo justo, sadio e excelso, algo
realmente humano. O aistórico é como uma atmosfera envolvente em que a
vida se reproduz, para de novo desaparecer com o aniquilamento dessa
atmosfera. É verdade: somente quando o homem pensa, reflete, compara,
discrimina e limita o elemento aistórico é que surge, no meio daquela
névoa envolvente, um brilho claro e luminoso; portanto, somente com a
força de utilizar o passado para a vida e fazer do ocorrido novamente
história, o homem tornou-se homem. Mas um excesso de história paralisa
de novo o homem, e sem o manto do aistórico ele nunca teria surgido nem
ousado surgir. Onde se encontram ações que os homens foram capazes de
realizar sem antes adentrar aquela camada de névoa do aistórico? Ou,
deixando de lado as imagens e tomando um exemplo para ilustração:
imaginem um homem assaltado e impulsionado por uma forte paixão, seja
por uma mulher ou por um grande pensamento; como seu mundo se
transforma! Olhando retrospectivamente, ele se sente cego; de sua parte, 
escuta mal os outros, como se ouvisse um barulho abafado e sem
sentido; ele sente como jamais havia sentido, ele sente tudo próximo,
colorido, sonoro, luminoso, como se percebesse simultaneamente por todos
os sentidos. Todas as estimativas de valor se modificaram e se
desvalorizaram; tanta coisa ele não consegue mais estimar, porque ele já
mal pode senti-las: ele se pergunta se ele não era estupidificado
pelas palavras e pensamentos alheios; ele se admira que sua memória gire
sem descanso em um círculo e, contudo, esteja tão fraca e cansada para
realizar um salto para fora deste. É a condição mais injusta do
mundo: estreita, ingrata com o passado, cega para os perigos, surda para
advertências, um pequeno redemoinho vivo em um mar morto de noite e
esquecimento: e contudo é essa condição --- totalmente aistórica e
anti-histórica --- o útero não apenas do ato injusto, mas, ao contrário,
de todo ato justo; e nenhum artista alcançará sua obra, nenhum general,
sua vitória, nenhum povo, sua liberdade, sem antes ter querido e ansiado
tal estado aistórico. Assim como todo homem de ação, segundo as palavras
de Goethe, é inescrupuloso, ele também é leviano;\footnote{Nietzsche se refere ao seguinte
  aforismo de Goethe: ``O homem de ação (\emph{Handelnde}) é sempre
  inescrupuloso; ninguém possui mais consciência que o homem
  contemplativo (\emph{Betrachtende}).'' (``Maximen und Reflexionen''.
  In: \emph{Gesammelte Werke}, Herausgegeben von Erwin Laaths, Bd. 6,
  p.\,281). Düsseldorf: Deutscher Bücherbund, 1952. Não foi possível
  manter o jogo de palavras entre inescrupuloso (\emph{gewissenlos}) e
  leviano (\emph{wissenlos}).} para realizar algo
ele esquece a maioria das coisas, ele é injusto com aquilo que repousa
atrás dele e conhece apenas um direito, o direito daquilo que agora deve
vir a ser. Assim, todo homem de ação ama seu ato infinitamente mais do
que ele mereceria: e os melhores atos ocorrem em tal superabundância de
amor, amor que eles, em todo caso, não deveriam merecer, mesmo se seu
valor fosse inestimavelmente alto.

Se alguém pudesse estar em condições, em diversos casos, de inalar e
respirar essa atmosfera aistórica, na qual todo evento histórico
grandioso surge, esse alguém poderia talvez, enquanto um ser que
conhece, elevar-se a um ponto de vista \emph{supra-histórico}, tal como
o descreveu Niebuhr,\footnote{Barthold G. Niebuhr (1776--1831),
  historiador prussiano. Até o momento não se identificou a fonte da
  citação que Nietzsche faz a seguir.} como resultado possível da
consideração histórica. Diz ele: ``Entendida de maneira clara e precisa,
a história é útil ao menos para uma coisa --- para que se saiba como também
os maiores e superiores espíritos da espécie humana não sabem quão
fortuitamente eles adquirem a forma pela qual veem e obrigam, à força,
que todos vejam; à força, porque a intensidade de sua consciência é
excepcionalmente grande. Quem não soube e não percebeu isso com clareza e em diversas
circunstâncias se submeterá ao surgimento de um espírito poderoso que
imprima a maior paixão a uma forma dada''. Tal ponto de vista poderia
ser chamado de supra-histórico, porque não se poderia perceber, naquele
que o defende, nada que o seduza a continuar vivendo e a participar da
história, pois ele reconheceria a condição única de todo evento, aquela
cegueira e injustiça na alma de quem age. Ele mesmo evitaria levar a
história demasiadamente a sério: ele teria aprendido, em toda
experiência, entre gregos ou turcos, seja no século \textsc{i} ou no século \textsc{xix},
a responder à questão de como e para que se viveu. Quem perguntar a
conhecidos se eles desejariam viver novamente os últimos dez ou vinte
anos, perceberá facilmente qual deles está cultivado para aquele ponto
de vista supra-histórico: todos bem responderão que ``Não!'', embora
esse ``Não!'' será distintamente justificado. Um talvez porque se
consolou: ``mas os próximos vinte anos serão melhores''; como aqueles
sobre os quais David Hume comentou com ironia:

\begin{quote}
\begin{verse}
\emph{And from the dregs of life hope to receive,}\\
\emph{What the first sprightly running could not give.}\\

\bigskip

E, do que sobrou no copo da vida, esperam tomar\\
Aquilo que o primeiro gole de vigor não pôde dar.\footnote{Segundo
  Walter Kaufmann (\emph{Nietzsche} --- Philosopher, psychologist,
  antichrist. Princeton: Princeton University Press, 1974),
  citação da peça ``Aureng-zebe'', de John Dryden, incluída no
  \emph{Diálogo sobre a religião natural}, de Hume. No original consta
  ``pensam'' em vez de ``esperam''.}
\end{verse}
\end{quote}

Vamos chamá-los de homens históricos; o olhar para o passado empurra-os
para o futuro, inflama sua coragem de perseverar ainda mais longamente
na vida, acende a esperança de que a justiça ainda advirá, que a
felicidade se esconde atrás da montanha que escalarão. Esses homens
históricos creem que o sentido da existência sairá à luz paulatinamente
no decurso de um \emph{processo}; por isso eles só olham para trás, a
fim de entender o presente pela consideração do processo até o momento,
e aprendem a desejar ansiosamente o futuro; não sabem como eles, apesar
de sua história, pensam e agem aistoricamente, e como sua ocupação com a
história não está a serviço do conhecimento puro, mas da vida.

Mas aquela pergunta, cuja primeira resposta ouvimos, pode mais uma vez
ser respondida. Mais uma vez com um ``Não!'', só que com um ``Não''
distintamente justificado. Com o ``Não'' do homem supra-histórico, que não
vê no processo a salvação; para quem, ao contrário, o mundo está pronto
em cada instante singular e nele alcança seu fim. O que dez novos anos
poderiam ensinar que dez passados não puderam ensinar!

Se o sentido do seu ensinamento é a felicidade, a resignação, a virtude
ou a penitência, é uma coisa sobre a qual os homens supra-históricos
nunca tiveram de acordo; mas, contrariamente a todas as formas de
considerações históricas do passado, eles chegam ao total consenso nesta
sentença: o passado e o presente são uma e mesma coisa, ou seja, em toda
multiplicidade, são tipicamente iguais e, como uma onipresença de tipos
perpétuos, são uma imagem paralisada de valor invariável e de
significado eternamente idêntico.\label{eternamenteidentico} Como às centenas de diferentes línguas
correspondem as mesmas necessidades típicas dos homens, de tal modo que
quem entendesse essas necessidades nada de novo aprenderia de todas
essas línguas: assim o homem supra-histórico explica todas as histórias
dos povos e dos indivíduos a partir de dentro, adivinhando
profeticamente o sentido originário dos diversos hieróglifos e aos
poucos, até a exaustão, esquiva-se da correnteza incessante dos símbolos
gráficos: pois como poderia ele, na infinita abundância do que acontece,
não chegar à saciedade, ao empanzinamento, ou mesmo ao nojo? De modo tal
que o mais ousado, por fim, talvez esteja pronto para dizer a seu coração,
junto com Giacomo Leopardi:

\begin{quote}
\begin{verse}
Nada que vive\\
é digno de tua aflição, e nem um suspiro\\
a Terra merece.\\
Dor e tédio é nosso ser, e o mundo\\
é um lodo --- e nada mais.\\
Assossega-te.\footnote{Nietzsche cita um trecho do poema ``A se
  stesso'' (Para si mesmo), de Leopardi.}
\end{verse}
\end{quote}

Mas deixemos para os homens supra-históricos seu nojo e sua sabedoria.
Queremos hoje, ao contrário, tonarmo-nos alegres de coração por nossa
ignorância e, como homens de ação e progressistas, como veneradores do
processo, ganhar nosso dia. Nossa estima do histórico pode ser apenas um
preconceito ocidental; imersos nesse preconceito, pelo menos progredimos
e não nos imobilizamos! Aprendendo melhor a praticar a história em
proveito da \emph{vida}! Então concedamos aos homens supra-históricos
que possuam mais sabedoria que nós; podemos estar certos de que
possuímos mais vida do que eles: em todo caso, nossa ignorância terá
mais futuro que sua sabedoria. E para não restar dúvida a respeito do
sentido dessa oposição entre vida e sabedoria, apresentarei algumas
teses com o auxílio de um procedimento preservado desde a Antiguidade.

Um fenômeno histórico, conhecido de forma pura e completa, e diluído em
um fenômeno de conhecimento, é para aquele que conhece algo morto: pois
reconheceu nele a insânia, a injustiça, a paixão cega e em geral todo o
horizonte obscuro e mundano daquele fenômeno e, ao mesmo tempo,
reconheceu sua força histórica. Essa força tornou-se impotente para o
homem que conhece, mas talvez não para o que vive.

A história, pensada como ciência pura e soberana, seria para a
humanidade uma espécie de balanço contábil da vida. A cultura histórica
é, ao contrário, apenas em consequência de uma nova e poderosa corrente
vital, de uma cultura em transformação, por exemplo, algo salutar e
alvissareiro, portanto apenas quando dominada e conduzida por uma força
superior, e não quando domina e conduz.

A história, na medida em que está a serviço da vida, está a serviço de
uma força aistórica e por isso, por essa submissão, nunca pode nem deve
se tornar uma ciência pura, como a matemática. Contudo, a questão de que
até que grau a vida precisa da história é uma das maiores questões e
preocupações no que diz respeito à saúde de um homem, de um povo, de uma
cultura. Pois o excesso de história destrói e degenera a vida,
degenerando, por fim, a própria história.

\chapter*{2}\label{capuxedtulo-2}

Que a vida precisa do serviço da história é algo que deve ser entendido
com tanta clareza quanto esta sentença, que posteriormente deverá ser
provada: que o excesso de história prejudica o vivente. Em três
aspectos a história pertence ao vivente: ela lhe pertence enquanto
indivíduo atuante e determinado, enquanto conservador e reverente, e
enquanto sofredor e carente de libertação. A essa tríade de relações
corresponde uma tríade de espécies de história: na medida em que ela
permite diferenciar uma espécie de história \emph{monumental}, uma
\emph{antiquária} e uma \emph{crítica}.

A história pertence sobretudo ao homem de ação e forte, que luta uma
grande luta, que precisa de modelos, mestres, consoladores, não logrando
encontrá-los entre seus contemporâneos e no presente. Assim era com
Schiller: pois nossa época é tão ruim, disse Goethe, que o poeta não
encontra, entre os homens a sua volta, nenhuma natureza
aproveitável.\footnote{Cf. Goethe. \emph{Conversações com Eckermann}.
  Entrada datada de 21/07/1827.} Políbio, por exemplo, tendo em vista o
homem de ação, chama a história política de justa preparação para o
governo de um Estado e mestra suprema, por meio da qual a lembrança
dos infortúnios alheios nos orienta a suportar altivamente os revezes da
sorte. Quem aprendeu a reconhecer aqui o sentido da história deve
irritar-se por ver viajantes curiosos ou micrólogos detalhistas galgando
as pirâmides do passado; lá, onde ele encontra estímulo para imitar e
melhorar, não deseja encontrar o ocioso, ávido por diversão e sensação,
que age como se vagasse, dentro de uma galeria, entre conhecidos
tesouros da pintura. O homem de ação, em meio a ociosos fracos e
desesperançados, em meio a contemporâneos aparentemente ativos, quando
na verdade são apenas ansiosos e inquietos, não sente náusea nem
esmorece, olha para trás de si e só interrompe o passo em direção a seu
objetivo para respirar. Seu objetivo, contudo, é uma felicidade
qualquer, talvez não a sua própria, com mais frequência a de um povo ou
a do conjunto da humanidade; ele foge da resignação e utiliza a história
como remédio contra a resignação. Na maioria das vezes, não espera
vantagem alguma; quando muito, espera a fama, isto é, aspira a um posto
de honra no templo da história, onde mais uma vez ele poderá ser, para
os tardios, mestre, consolador e voz da advertência. Pois seu mandamento
é: o que foi capaz de expandir e tornar mais belo o conceito ``homem''
deve estar presente pela eternidade, para eternamente realizar esse
feito. O pensamento fundamental da crença na humanidade expresso pela
exigência de uma história \emph{monumental} é o de que os grandes
momentos na luta dos indivíduos formam uma corrente que os une, no
decorrer dos séculos, na cordilheira da humanidade; que, para mim, o
mais elevado de cada momento há muito ocorrido ainda é vivo, claro e
grandioso. Mas é justamente essa exigência de que o grandioso seja
eterno que deflagra a luta mais terrível. Pois todo o resto que ainda
vive grita ``Não''. O monumental não deve surgir --- esse é o lema contrário.
A rotina embrutecida, o que há de menor e mais baixo, ocupando todos os
cantos do mundo, enfumaçando, como um ar pesado, tudo o que é grande,
lança-se, impedindo, ludibriando, sufocando, asfixiando o caminho que o
grandioso deve percorrer em direção à imortalidade. Mas esse caminho
passa pelo cérebro humano! Através do cérebro dos animais mais aflitos e
menos longevos, que revelam sempre as mesmas necessidades e, com
esforço, evitam perecer durante um curto espaço de tempo. Pois eles
querem, antes de tudo, apenas uma coisa: viver a qualquer preço. Quem
poderia supor neles aquela pesada tocha olímpica da história monumental,
por meio da qual tudo o que é grandioso continua a viver! E contudo há
sempre aqueles poucos que acordam para o que foi grandioso no passado e,
fortalecidos por sua contemplação, sentem-se tão bem-aventurados, como
se a vida humana fosse uma coisa magnífica, e como se o mais belo fruto
dessa planta amarga fosse saber que antes alguém já se tornou, no
decorrer desta existência, orgulhoso e forte; um outro, melancólico, um
terceiro, compassivo e solícito --- mas todos deixando um ensinamento:
que vive de forma mais bela quem não se preocupa com a existência. Se o
homem comum toma esse período de tempo de forma seriamente triste e
cobiçosa, aqueles saberiam oferecer-lhe, em seu caminho para a
imortalidade e para a história monumental, uma gargalhada olímpica ou ao
menos um escárnio sublime; com frequência, descem com ironia para sua
sepultura --- pois o que havia neles a sepultar! Apenas aquilo que a
história monumental tinha pulverizado como fraqueza, despojo, vaidade,
bestialidade é agora lançado ao esquecimento, depois de lhe ter
dispensado seu desprezo. Mas algo viverá, um monograma de sua essência
mais íntima, uma obra, um ato, uma rara iluminação, uma criação: viverá
porque nenhuma posteridade pode renunciá-lo. Nessa forma transfigurada,
contudo, a fama é ainda algo mais que a degustação de nosso
amor-próprio, como Schopenhauer a chamou;\footnote{Cf. o capítulo ``Von
  Dem, was Einer vorstellt'' (Sobre aquilo que alguém representa), de
  ``Aphorismen zur Lebensweisheit'' (Aforismos de sabedoria de vida).
  In: \emph{Parerga e Paralipomena}. Zurique: Haffmans Verlag, 1999.} ela é a crença na correlação e
continuidade do grandioso de todas as épocas, é um protesto contra a
mudança de gerações e do passado.

Em que a consideração monumental do passado é útil ao homem atual,
quando lida com o clássico e o raro de épocas anteriores? Ele conclui
que, em todo caso, o grandioso que um dia existiu foi \emph{possível}
uma vez e por isso será possível novamente; ele toma com mais coragem o
seu rumo, pois agora uma dúvida que o assaltava nas horas mais difíceis
é vencida: se ele talvez não quisesse o impossível. Supondo-se que
alguém acredite que a tarefa de exterminar a espécie de cultura que
agora se tornou moda na Alemanha cabe a não mais que uma centena de
homens produtivos, ativos e cultivados em um novo espírito; isso lhe
fortaleceria a percepção de que a cultura da Renascença apoiava-se nos
ombros de um bando de centenas de tais homens.

E contudo --- para aprender com o mesmo exemplo algo novo --- quão fluida e
pendente, quão imprecisa seria essa comparação! Quanta diferença se teve
de omitir para que ela tivesse aquele efeito vigoroso, quão
violentamente se teve de comprimir a individualidade do passado em uma
forma universal e aparar arestas em proveito da conformidade! No fundo,
aquilo que uma vez foi possível só poderia ocorrer uma segunda vez se os
pitagóricos estivessem certos em acreditar que, dada uma constelação
idêntica de corpos celestes, as mesmas coisas deveriam repetir-se também
na Terra, nos mínimos detalhes, de tal modo que, sempre que as estrelas
estiverem numa certa posição em relação às outras, um estoico e um
epicurista se aliarão e assassinarão César\footnote{Referência à conspiração entre
  Caio Cássio e Marco Bruto para assassinar César.} e, num outro arranjo, Colombo
novamente descobrirá a América. Apenas quando a
Terra começasse novamente sua peça teatral depois do quinto ato, quando
se estabelecesse que em determinados intervalos de tempo se repetiriam o
mesmo encadeamento de motivos, o mesmo \emph{deus ex machina}, a mesma
catástrofe, os poderosos poderiam desejar a história monumental
revestida da \emph{veracidade} de um ícone, ou seja, desejar todo fato
em sua exata peculiaridade e singularidade: provavelmente isso não
acontecerá até que os astrônomos voltem a ser astrólogos. Até lá, a
história monumental poderá não precisar daquela veracidade toda: ela
sempre aproximará, universalizará e, enfim, igualará o desigual, sempre
enfraquecerá a diversidade dos motivos e ocasiões, a fim de tomar, de
forma monumental, o \emph{effectus} às custas das \emph{causae}, isto é,
como algo exemplar e digno de imitação: por não se importar com as
causas, ela poderia chamar-se, com um pouco de exagero, de um conjunto
de ``efeitos em si'', como eventos que provocarão efeitos em todas as
épocas. O que será festejado em uma festa popular, um dia santo ou um
desfile militar, será propriamente esse ``efeito em si'': é ele que não
deixa os ambiciosos dormirem, é ele que repousa, como um amuleto, no
coração dos empreendedores, e não a verdadeira conexão histórica entre
causa e efeito, que, inteiramente conhecida, só provaria que nada de
idêntico poderia surgir no lance de dados do futuro e do acaso.

Quando a alma da historiografia repousa no grande \emph{estímulo} que um
indivíduo poderoso dela extrai, quando ela tem de descrever o passado
como algo digno de imitação, imitável e possível por uma segunda vez,
ela arrisca-se, em todo caso, a contrabandear algo, a edulcorar o
passado, aproximando-se assim da livre poetização; aliás, há épocas em
não se consegue distinguir o passado monumental da ficção mítica: porque
os mesmíssimos estímulos podem ser extraídos de um mundo ou de outro.
Portanto, se a consideração monumental do passado \emph{reina} sobre as
outras espécies de consideração, quero dizer, sobre a antiquária e a
crítica, então o próprio passado sofre \emph{prejuízo}: grandes partes
são totalmente esquecidas, desprezadas, e escorrem como uma enchente
terrível e interminável, da qual emergem, como ilhas, apenas alguns
fatos embelezados: para algumas pessoas de boa visão, salta aos olhos
algo de antinatural e sobrenatural, como a coxa dourada com que os
discípulos de Pitágoras diziam reconhecer seu mestre.\footnote{Anedota
  presente no livro \textsc{viii} de \emph{Vida e opiniões dos mais eminentes
  filósofos}, de Diógenes Laércio.} A história monumental ilude por
meio de analogias: com semelhanças sedutoras, ela estimula os corajosos
à temeridade, os entusiastas ao fanatismo; e se pensarmos essa história
nas mãos e mentes de egoístas talentosos e facínoras delirantes,
impérios serão destruídos, príncipes serão assassinados, guerras e
revoluções serão fomentadas e aumentará novamente o número de ``efeitos
em si'', ou seja, dos efeitos sem causa suficiente. Isso para lembrar o
estrago que a história monumental pode provocar nas mãos de homens
poderosos e ativos, sejam eles bons ou maus. O que pode causar quando
dominada e utilizada pelos impotentes e inativos!

Tomemos o exemplo mais simples e frequente. Pensemos nas naturezas mais
inartísticas e debilmente artísticas armadas e valorizadas pela história
da arte monumental: contra quem elas dirigirão suas armas? Contra seus
inimigos contumazes, os grandes espíritos artísticos; portanto, contra
os únicos que tornam tal história veraz, isto é, capaz de ensinar a
viver e transformar em prática o que foi aprendido. O caminho destes é
obstruído, o ar escurece, quando aqueles dançam, com idolatria e zelo,
em torno de um monumento, entendido pela metade, de um passado grandioso
qualquer, como se quisessem dizer: ``Vejam, isto é a arte verdadeira e
real: não nos importam os que se transformam e os que têm querer!''
Aparentemente, essa turba dançante tem até o privilégio do ``bom
gosto'': pois aquele que cria esteve sempre em desvantagem diante
daquele que só observa e não executa com as próprias mãos, assim como,
em todas as épocas, os políticos de botequim foram mais prudentes,
corretos e reflexivos do que o estadista que governa. Mas se se
transpuser ao âmbito da arte o costume do plebiscito e da maioria
numérica e o artista precisar deles para sua defesa, diante do tribunal
dos homens inativos, pode-se assegurar de antemão que ele será
condenado: não apesar de, mas justamente \emph{porque} seus juízes
proclamaram festivamente o cânone da arte monumental, isto é, a
mencionada declaração de que a arte de todos os tempos ``provocou
efeitos'': enquanto para eles isso não ocorre com toda arte
não monumental, pois, para eles, à arte contemporânea falta, em primeiro
lugar, a necessidade, em segundo, o anseio, em terceiro, aquela
autoridade da história. Ao contrário, seu instinto lhes revela que a
arte poderia ser assassinada pela própria arte: o monumental não deve
surgir, e para isso se utilizam justamente da autoridade que o
monumental extrai do passado. Assim eles são os especialistas em arte,
porque em geral eles gostariam de deixar de lado a arte; eles agem como
médicos, quando no fundo pretendem envenenar, cultivando sua língua e
seu paladar para atribuir a essa deseducação a recusa de pratos
artísticos nutritivos. Porque eles não querem que o grandioso surja, seu
remédio é dizer: ``veja, o grandioso já está aí!'' Na verdade, esse
grandioso que já está aí lhes importa tão pouco quanto o que surge:
disso sua vida dá testemunho. A história monumental é a máscara pela
qual dão vazão a seu ódio dirigido contra os homens grandes e poderosos
de sua época, através de seu maravilhamento exagerado diante dos homens
grandes e poderosos
%Como esta "homens grandes e poderosos", haverá outras repetições, provavelmente como no original alemão, mantemos?
 de épocas passadas; dissimuladamente, eles
transformam em seu contrário o autêntico sentido daquela espécie de
consideração histórica. Se eles sabem ou não com clareza o que fazem, o
fato é que eles agem assim, como se sua divisa fosse: deixemos que os
mortos enterrem os vivos.

Cada uma das três espécies de história existentes tem seu lugar em um
determinado solo e sob um determinado clima: em outros casos alastram-se
como ervas daninhas. Se um homem quiser criar algo grandioso, e 
precisar do passado, então se apoderará do passado por meio da
história monumental; aquele que, ao contrário, quiser preservar o
costume e a reverência pelo que é antigo, cultivará o passado como
historiador antiquário; e apenas aquele em quem a carência do presente
aperta o peito, querendo livrar-se a qualquer preço do seu fardo, tem
necessidade da história crítica, isto é, da história que julga e
condena. A transposição descuidada de vegetais produz danos: o crítico
sem necessidade, o antiquário sem piedade, o conhecedor do grandioso sem
a capacidade do grandioso são tais plantas degeneradas, que se alastram
como ervas daninhas quando afastadas de seus solos naturais.

\chapter*{3}\label{capuxedtulo-3}

Em segundo lugar, a história também pertence ao conservador e reverente que, 
com fidelidade e amor, olha para trás, de onde veio e de
onde veio a ser; ele traz igualmente, por meio dessa piedade, a gratidão
por sua existência. Cuidando com zelo do que é antigo e permanente, quer preservar
 as condições sob as quais surgiu e outros deverão
surgir depois dele --- e assim ele serve à vida. Em tal espírito, a posse
dos utensílios ancestrais\footnote{Referência ao \emph{Fausto}, de
  Goethe, \textsc{i}, linha 408.} modifica suas ideias: ele passa a ser por eles
possuído. O que é pequeno, estreito, podre e envelhecido adquire honra
pelo fato de que a alma conservadora e reverente do homem antiquário se
transmigra para essas coisas e constrói para si um ninho oculto. A
história de sua cidade se torna a sua própria história; ele entende a
muralha, o pórtico fortificado, os éditos municipais e festas populares
como seu diário, reencontrando nisso tudo sua força, seu denodo, seu
prazer, seu julgamento, sua loucura e sua travessura. Aqui se viveu, ele
diz a si mesmo, pois aqui se vive, aqui se viverá, já que somos
resistentes e não desmoronamos durante a noite. Então ele vê esse
``nós'' através de vidas singulares pretéritas e maravilhosas e se sente
como o espírito do lar, da espécie, da cidade. Nisso ele saúda, através
de séculos obscuros e confusos, a alma de seu povo como a sua própria
alma; um sentimento transpassa e daí surge uma nostalgia, um farejar de
rastros apagados, uma leitura correta de um passado rasurado, um
entendimento ágil do palimpsesto, até mesmo do \emph{polipsesto}\footnote{Neologismo
  de Nietzsche, em analogia a ``palimpsesto''. Depreende-se que o
  sentido seja o de um pergaminho cujo manuscrito foi apagado e
  reutilizado muitas vezes.} --- estes são seus talentos e virtudes. Com
eles, Goethe quedou-se diante do monumento de Erwin von
Steinbach;\footnote{Arquiteto alemão que construiu a catedral de
  Estrasburgo. Na passagem seguinte, Nietzsche faz alusão ao ensaio de
  ``Von deutscher Baukunst'' (Sobre a arquitetura alemã), de Goethe.}
na tempestade de seu sentimento rasgou o véu nebuloso que a história
estendeu entre eles: ele viu, pela primeira vez, a obra alemã
``realizar-se, provinda da forte e rouca alma alemã''. Tal sentido e
ímpeto conduziram os italianos da Renascença e refizeram despertar, em
sua densidade, o gênio italiano antigo, ``o eco maravilhoso da lira
arcaica'', como disse Burckhardt.\footnote{Citação extraída de \emph{Die
  Kultur der Renaissence in Italien} (\emph{A cultura do Renascimento na
  Itália}).} Mas aquele sentido reverente, histórico-antiquário, tem
seu maior valor onde ele estende um simples e pungente sentimento de
prazer e satisfação sobre aquelas condições modestas, rudes, até mesmo
miseráveis em que um homem ou um povo vive; como, por exemplo, Niebuhr
assumiu, com uma credulidade ingênua e honesta: no pântano e na
charneca, entre camponeses livres, que possuem uma história, basta
viver; não se sente falta de arte alguma. Como poderia a história melhor
servir à vida do que ligando gerações e populações menos favorecidas a
sua pátria e a costumes pátrios, tornando-as nacionalistas e
impedindo-as de correr atrás do que há de melhor no estrangeiro, para
conquistá-lo em disputas? Do mesmo modo, parece ser teimosia e desatino
fixar o indivíduo nessa comunidade e localidade, nesses hábitos
extenuantes, nesse cume árido --- mas esse desatino é o que há de mais
salutar e exigente para a comunidade; como bem sabe aquele que conheceu
claramente os efeitos terríveis do prazer aventureiro de emigrar de uma
população inteira ou o estado de um povo que viu de perto a perda da
confiança em sua época precedente, sacrificando-a em nome de uma busca e
escolha cosmopolitas e incessantes pelo novo. A sensação contrária, o
bem-estar da árvore com suas raízes, uma alegria não de todo arbitrária
e fortuita em saber que vicejou, como legado, flores e frutos de um
passado que absolve ou mesmo justifica a sua existência --- isto é o que
se designa preferencialmente como o autêntico sentido histórico.

Essa não é certamente a condição em que o homem, na maioria das vezes,
seria capaz de dissolver o passado em um conhecimento puro; assim,
percebemos aqui também o que havíamos percebido na história monumental,
que o próprio passado sofre na medida em que a história serve à vida e é
dominada por um impulso vital. Falando com alguma liberdade poética, a
árvore mais sente suas raízes do que as vê: mas esse sentimento mede sua
grandeza pelo tamanho e força de seus ramos visíveis. Pudesse a árvore
nisso equivocar-se: como estaria ela equivocada sobre toda a floresta em
volta! Sobre aquilo que ela só sabe e sente na medida em que a impede ou
a fomenta---mas nada além disso. O sentido antiquário de um homem, de
uma comunidade, de todo um povo possui sempre um campo de visão
extremamente estreito; ele não percebe a maioria das coisas e, do pouco
que vê, vê de forma muito próxima e isolada; não é capaz de medir, e por
isso toma tudo como igualmente importante e todo indivíduo como
demasiado importante. Não existem, então, para as coisas do passado,
diferenças de valor e proporções que de fato fizessem justiça às coisas
em relação entre si; ao contrário, há apenas medidas e proporções das
coisas para o indivíduo ou povo que olha retrospectivamente de maneira
antiquária.

Aqui, um perigo está sempre próximo: de, ao fim, tomar-se tudo o que for
antigo e pretérito, tudo que se encontra em seu campo de visão, como
igualmente digno de honra; enquanto o que é novo e em transformação, o
que não se dirige ao antigo com veneração, é recusado e hostilizado.
Assim os gregos toleravam o estilo hierático de suas artes plásticas, em
contraposição a estilos livres e grandiosos; depois, não só toleravam os
narizes empinados e os risos mordazes, mas neles encontravam um gosto
refinado. Quando o sentido de um povo se enrijece, quando a história
serve assim a uma vida passada, de tal forma que sepulta a continuação
da vida e justamente a vida superior, quando o sentido histórico não
mais conserva, mas mumifica: desse modo a árvore morre de uma forma
antinatural, de cima para baixo, gradualmente --- e, por fim, a própria
raiz fenece. A história antiquária degenera no mesmo instante em que o
frescor da vida atual não mais anima e entusiasma. Agora, a piedade se
resseca, o hábito erudito permanece sem ela e se volta, de forma egoísta
e vaidosa, para seu próprio âmago. Então, assiste-se ao espetáculo
repugnante de uma mania cega de colecionar, de uma acumulação incessante
de tudo o que já existiu. O homem se envolve em um odor cadavérico; ele
logra encontrar um lugar mais significativo, uma necessidade mais nobre,
através do modo antiquário da curiosidade insaciável, da reta ânsia de
tudo o que é antigo; frequentemente se afunda tanto que, por fim, se
satisfaz com qualquer bocado e devora com prazer a poeira das
quinquilharias.

Mas mesmo quando não ocorre essa degeneração, quando a história
antiquária não perde o único fundamento sobre o qual a redenção da vida
pode enraizar-se, sempre resta o perigo nada irrisório de que ela se
torne demasiado poderosa e suplante as outras espécies de consideração
do passado. Ela só sabe \emph{preservar} a vida, mas não produzi-la; por
não possuir nenhum instinto divinatório para o devir --- como possui, por
exemplo, a história monumental ---, ela sempre o subestima. Assim, ela
impede aquela decisão forte pelo novo, assim ela lamenta o homem de
ação, que, como aquele que age, sempre deve e terá de ferir qualquer
piedade. O fato de que algo se tornou antigo traz agora à luz a
exigência de que deva ser imortal; pois se alguém calcular tudo o que
essa antiguidade experimentou no decorrer de sua existência --- um antigo
costume paterno, uma crença religiosa, um privilégio político herdado
---, aquele montante de piedade e reverência por parte do indivíduo e das
gerações, parecerá algo arrogante e inescrupuloso substituir tal
antiguidade por uma nova criação e contrapor algo transformador e atual
ao acúmulo numérico de piedade e reverência.

Com isso fica claro como o homem, com frequência, tem necessidade de
considerar o passado, além de uma forma monumental e antiquária, de uma
\emph{terceira} forma, a \emph{crítica}, e mais uma vez a serviço da
vida. De tempos em tempos, ele deve utilizar a força de destruir e
dissolver um evento passado para que possa viver: ele alcança isso
levando esse passado ao tribunal, interrogando-o minuciosamente e, enfim,
condenando-o; mas todo passado merece ser condenado --- pois assim são as
coisas humanas: sempre nelas existiriam a violência e a fraqueza
humanas. Não é a justiça que se senta aqui no tribunal; muito menos é a
clemência que anuncia o julgamento: mas somente a própria vida, aquele
poder obscuro, impulsionador, insaciável, que deseja a si mesmo. Seu
veredito é sempre inclemente, sempre injusto, pois ele nunca jorra da
pura fonte do conhecimento; mas, em todo caso, o veredito sempre falha,
mesmo se enunciado pela justiça. ``Pois tudo o que surge
\emph{merece} extinguir-se. Melhor seria que não surgisse''. Há bem mais
força de vida e esquecimento quando viver e ser injusto são uma única
coisa. Lutero pensou, certa feita, que o mundo só surgiu graças a um
esquecimento de Deus, isto é, se Deus tivesse pensado nos ``armamentos
pesados'', não teria criado o mundo.\footnote{Segundo o tradutor francês,
  Nietzsche faria referência aqui à seguinte passagem dos discursos
  (\emph{Tischreden}) de Lutero: ``Se Adão pudesse ter visto os
  artefatos fabricados por seus filhos, ele teria morrido de desgosto''.}
Mas a mesma vida que precisa do esquecimento precisa, de vez em quando,
da destruição desse esquecimento; então deve tornar-se claro como, por
exemplo, é injusta a existência de um privilégio, de uma casta, de uma
dinastia quaisquer e como essas coisas deveriam perecer. Então, seu
passado passa a ser considerado criticamente, suas raízes são
golpeadas com um facão, a piedade é cruelmente pisoteada. É sempre
um processo perigoso, isto é, perigoso para a própria vida: e homens ou
épocas que servem à vida dessa forma são sempre épocas e homens 
perigosos. Pois lá onde somos resultado de gerações anteriores, somos
também resultado de seus desvios, paixões, erros e até mesmo crimes; não
é possível se livrar dessa cadeia. Se condenarmos aqueles desvios e nos
tomarmos como libertos deles, isso não elimina o fato de que deles
descendemos.\label{delesdescendemos} No melhor caso, reduzimos isso a uma disputa entre a
natureza herdada e atávica e nosso conhecimento, ou bem a uma luta de um
novo e duro disciplinamento contra um antigo, impregnado e inato;
plantamos um novo hábito, um novo instinto, uma segunda natureza que
apodrece a primeira. É uma tentativa, igualmente, de se fornecer um
passado \emph{a posteriori}, do qual se gostaria de descender, em
contradição com aquilo do que se descende --- sempre uma tentativa
perigosa, porque é muito difícil encontrar um limite para a negação do
passado e porque as segundas naturezas são, na maioria das vezes, mais
fracas que as primeiras. Muito frequentemente se permanece em um
conhecimento do bem sem realizá-lo, porque se pode conhecer o que há de
melhor sem poder realizá-lo. Mas vez por outra se alcança a vitória, e
também há, para os lutadores que servem à vida por meio da história
crítica, um consolo suspeito: saber que aquela primeira natureza já foi
uma vez uma segunda natureza e que aquela segunda natureza vitoriosa se
tornará uma primeira.

  \chapter*{4}\label{capuxedtulo-4}

    Esses são os serviços que a história pode prestar à vida; todo homem
    e todo povo precisam, segundo seus objetivos, forças e necessidades,
    de um certo conhecimento do passado, às vezes monumental, às vezes
    antiquário, às vezes crítico. Mas não como um bando de pensadores
    puros que só observam a vida, não como indivíduos ávidos de
    conhecimento, que só se satisfazem com o saber que tem como objetivo
    o aumento do conhecimento, e sim com fins vitais, e portanto sob o
    domínio e a condução desses fins. Que essa seja a relação natural de
    uma época, de uma cultura, de um povo com a história --- provocada
    pela fome, regulada pelo grau de necessidade, limitada pela força
    plástica interior ---, que o conhecimento de todos os tempos seja
    desejável apenas a serviço do futuro e do presente, não para o
    enfraquecimento do presente, não para a extirpação de um futuro
    revigorante: tudo isso é simples como a verdade é simples, e
    convence de imediato aquele que não se deixa levar pela prova
    histórica.

    Lancemos agora um rápido olhar para a nossa época! Nós nos
    apavoramos, fugimos: para onde foi toda clareza, naturalidade e
    pureza na relação entre vida e história? Como esse problema, diante
    de nossos olhos, nos parece confuso, exagerado e inquietante! A
    culpa é nossa, que observamos? Ou a constelação constituída de vida
    e história realmente mudou quando um astro rival se colocou entre
    elas? Outros podem sugerir que enxergamos errado: queremos dizer o
    que pensamos ter visto. Foi contudo tal astro ali presente,
    resplandecente e magnífico, que realmente modificou a constelação ---
    \emph{através da ciência, através da exigência de que a história
    devesse ser científica}. Agora a vida não mais impera sozinha e
    conduz o conhecimento sobre o passado: em contrapartida, todas as
    barreiras são eliminadas e tudo o que já existiu desmorona sobre os
    homens. Quanto mais houver um devir retroativo, tanto mais todas as
    perspectivas são empurradas para o infinito. Nenhuma geração
    assistiu a esse espetáculo de forma tão explícita como a ciência do
    devir universal, a história, hoje o apresenta; certamente ela o
    apresenta com a perigosa temeridade de seu lema: \emph{Fiat veritas
    pereat vita}.\footnote{Que a verdade se realize e que o mundo pereça.}

    Imaginemos agora o processo mental presente na alma do homem
    moderno. O saber histórico flui, continuamente e em diversas
    direções, de fontes inesgotáveis, o estranho e disparatado o
    pressiona, a memória abre todas suas portas, entretanto ainda não o
    suficiente; a natureza se esforça ao máximo em receber esses
    hóspedes estranhos, em organizá-los e venerá-los; estes, contudo,
    estão em luta uns com outros, parecendo ser necessário
    constrangê-los e coagi-los para que eles não pereçam nessa luta. A
    habituação a essa moradia desorganizada, tempestuosa e beligerante
    torna-se paulatinamente uma segunda natureza, estando igualmente
    fora de questão se essa segunda natureza é muito mais fraca, muito
    mais inquieta e completamente mais doentia do que a primeira. Enfim,
    o homem moderno carrega consigo uma quantidade descomunal de
    indigestas pedras de conhecimento, que então, em certo momento e em
    sua ordem, estrepitam na barriga, como no conto de fadas.\footnote{Referência
      ao conto ``O lobo e os sete cabritinhos'', dos Irmãos Grimm.}
    Esse estrépito revela a caraterística mais própria desse homem
    moderno, que os povos antigos não conheciam: a estranha contradição
    de um interior que não corresponde a um exterior e um exterior que
    não corresponde a um interior. O saber que se empanturra, sem fome e
    mesmo sem necessidade, não mais produz um motivo que transfigura e
    se dirige para o exterior, e permanece oculto em um certo mundo
    interior caótico, que aquele homem moderno, com raro orgulho,
    denomina como sua mais própria ``interioridade''. Bem que se diz que
    ele teria um conteúdo, faltando-lhe apenas a forma; mas em todo
    vivente isso é uma contradição totalmente imprópria. Por isso
    nossa cultura moderna não é algo vivo, ela não pode ser compreendida
    sem aquela contradição, isto é: ela não é uma verdadeira cultura,
    mas uma espécie de saber em torno da cultura; ela permanece sendo
    uma ideia de cultura, um sentimento de cultura que não resulta em
    uma definição cultural. Ao contrário, o motivo real, que se
    apresenta em ato, com frequência não significa muito mais que uma
    convenção indiferente, uma imitação lamuriosa ou mesmo uma tosca
    caricatura. No interior está a sensação daquela cobra que engoliu um
    coelho inteiro e então descansa tranquilamente ao sol e evita
    qualquer movimento que não seja necessário. O processo interior é
    agora a própria coisa, isto é, a autêntica ``cultura''. Todos os que
    cruzam com tal cultura desejam que ela não morra de indigestão.
    Imagine, por exemplo, um grego que cruzasse com tal cultura; ele
    perceberia que, para os homens modernos, ``ser culto'' e ``ser culto
    em assuntos históricos'' parecem tão vinculados que é como se fossem
    uma coisa, cuja diferença só residiria no número de palavras. Se ele
    expressasse sua sentença: alguém pode ser bastante culto e
    totalmente inculto em assuntos históricos, todos acreditariam não
    tê-lo ouvido direito e balançariam a cabeça em sinal de
    desaprovação. Aquele pequenino povo já mencionado, não tão distante
    de nós, quero dizer, os gregos, tinham preservado com zelo, no
    período de sua maior força, um sentido aistórico; se um homem
    contemporâneo retornasse por mágica àquele mundo, ele provavelmente
    acharia os gregos bastante ``incultos''; isso certamente revelaria,
    para o escárnio público, o segredo da cultura moderna, tão
    penosamente dissimulado; pois nós, modernos, nada somos; somente
    quando nos preenchemos e nos abarrotamos de épocas, costumes, artes,
    filosofias, religiões e conhecimentos de outrem é que nos tornamos
    algo digno de atenção, isto é, enciclopédias ambulantes, como
    poderia nos chamar um heleno maldoso. Mas todo o valor das
    enciclopédias reside apenas naquilo que nela consta, no conteúdo,
    não naquilo que é capa e invólucro; e assim é toda a cultura
    moderna, interior; por fora, o encadernador imprimiu algo como:
    \emph{Manual de cultura interior para bárbaros da exterioridade}. Aliás,
    essa contradição entre interior e exterior torna o exterior mais
    bárbaro do que ele devia ser, quando um povo rude cresce a partir de
    suas necessidades grosseiras apenas. Pois que artifício resta à
    natureza para coagir o que se expande além da medida? Apenas o
    artifício de aceitá-lo o mais facilmente possível, a fim de logo
    eliminá-lo e afastá-lo. Daí surge o hábito de não mais levar a sério
    as coisas reais, daí surge a ``personalidade fraca'', a qual a
    realidade, o existente, pouco impressiona; ela se torna mais
    negligente e comodista com a exterioridade, aumentando o abismo
    entre conteúdo e forma até a insensibilidade para a barbárie; a
    memória é ininterruptamente estimulada, novas coisas dignas de
    conhecer borbotoam e podem ser colocadas, com apuro, nas caixas da
    memória. A cultura de um povo, como o oposto dessa barbárie, foi uma
    vez designada --- com algum direito, como penso --- como unidade do
    estilo artístico em todas as expressões vitais de um povo;\footnote{Nietzsche
      refere-se aqui a sua formulação de cultura presente na primeira
      extemporânea, \emph{David Strauss, o devoto e o escritor}.}\label{vitaisdeumpovo} essa
    designação não deve ser mal-entendida como se se tratasse de uma
    oposição entre barbárie e \emph{belo} estilo; o povo a que se
    prescreve uma cultura deve ser, em toda efetividade, apenas uma
    unidade viva e não se dividir penosamente em interior e exterior,
    entre conteúdo e forma. Quem quer incentivar e fomentar a cultura de
    um povo incentiva e fomenta essa unidade superior e contribui com o
    aniquilamento da aculturação moderna em favor de uma cultura
    verdadeira; ele ousa refletir sobre como a saúde de um povo,
    prejudicada pela história, pode ser recuperada, como ele pode
    reencontrar seus instintos e, com eles, sua honra.

    Quero falar, justamente agora, sobre nós, alemães da atualidade, que
    sofremos mais que qualquer outro povo daquela fraqueza de
    personalidade e da contradição entre conteúdo e forma. Para nós,
    alemães, a forma é comumente uma convenção, uma vestimenta e um
    disfarce, sendo, por isso, se não odiada, em todo caso não amada;
    mais precisamente, se poderia dizer que há um medo extraordinário da
    palavra convenção e ainda mais da própria coisa convenção. Nesse
    medo o alemão abandonou a escola francesa: pois ele queria tornar-se
    natural e, desse modo, alemão. Contudo, ele parece ter errado nas
    contas nesse ``desse modo'': desviado da escola da convenção, ele se
    deixou levar como e para onde ele bem entendeu, e imitou, no fundo,
    de modo inseguro e arbitrário, meio avoado, aquilo que outrora
    imitava detalhadamente e frequentemente com sucesso. Assim se vive
    hoje, em comparação a épocas anteriores, preguiçosamente, em uma
    convenção francesa incorreta: como mostra todo nosso modo de andar,
    de nos portar, de conversar, de vestir-se e de morar. Na medida em
    que se acredita em um retorno ao natural, são opções apenas o se
    deixar ir, o conforto e a medida mínima de superação de si. Ao
    caminhar por uma cidade alemã --- toda convenção, comparada às
    características nacionais de cidades estrangeiras, se mostra em
    negativo: tudo é desbotado, desgastado, mal copiado, todos anseiam
    por suas coisas preferidas, que não são vigorosas, criativas; ao
    contrário, seguem as regras que prescrevem a afobação geral e assim
    a mania de conforto. Uma roupa cuja invenção não é nenhum
    quebra-cabeça, cujo projeto não leva tempo, ou seja, uma imitação
    emprestada do exterior, da forma menos onerosa possível, é para os
    alemães uma contribuição ao traje típico alemão. O sentido da forma
    foi, ironicamente, recusado --- pois sem dúvida se tem \emph{o
    sentido do conteúdo}: o alemão é, no fim das contas, o célebre povo
    da interioridade.

    Mas há também um célebre perigo nessa interioridade: o próprio
    conteúdo, que se considera não poder ser visto de fora, pode,
    eventualmente, dissipar-se; de fora, ele não seria percebido, nem em
    sua dissipação nem em sua presença prévia. Mas se ao menos pensarmos no
    povo alemão o mais afastado possível desse perigo: o estrangeiro
    está certo em nos objetar que nosso interior é muito fraco e
    desordenado para agir exteriormente e se dar forma. Nisso o alemão
    pode mostrar-se, em graus excepcionais, solícito, sério, forte,
    introspectivo, bom e talvez até mais rico interiormente que outros
    povos: mas no todo permanece fraco, pois todos os belos fios não são
    atados em nós fortes: de tal modo que o ato visível não é o ato em
    totalidade e a revelação para si desse interior, mas sim uma
    tentativa fraca e tosca de querer que qualquer fio que apareça valha
    como totalidade. Por isso os alemães não podem ser julgados a partir
    de uma ação e, como indivíduo, pode permanecer oculto após esse ato.
    Como se sabe, deve-se medir o alemão segundo seus pensamentos e
    sentimentos, os quais ele expressa em seus livros. Quando esses
    livros não despertam nada de novo senão a dúvida se a célebre
    interioridade não repousa em seu templinho inacessível: seria um
    pensamento terrível que a interioridade um dia desaparecesse e
    aquela exterioridade arrogante, desajeitada e de uma preguiça
    desprezível restasse como marca do alemão. Quase tão terrível quanto
    aquela interioridade ali sentasse, sem poder ser vista, falsificada,
    colorida, pintada, fazendo as vezes de atriz, quando não coisa pior:
    como parece concordar, por exemplo, Grillparzer,\footnote{Franz
      Grillparzer (1791--1872), poeta e dramaturgo austríaco.} que
    observa, de modo marginal e discreto, a partir de sua experiência
    dramático-teatral: ``Nós sentimos com abstrações'', diz ele, 
    ``não mais sabemos como nossos contemporâneos expressam seus sentimentos; 
    fazemo-los agir como  hoje ninguém mais age. Shakespeare nos arruinou''.

    Esse é um caso singular, significativo, talvez muito rapidamente
    generalizável: mas como seria terrível sua legítima generalização
    se os casos singulares não devessem, frequentemente, fazer o
    observador perceber quão desesperada soa a frase: nós, alemães,
    sentimos com abstração; fomos todos arruinados pela história --- uma
    frase que destruiria pela raiz toda esperança em uma cultura
    nacional futura, pois essa esperança cresce a partir da crença na
    autenticidade e imediatidade do sentimento alemão, da crença na
    interioridade íntegra; o que ainda deve ser esperançado, acreditado,
    quando a fonte da crença e da esperança maculou-se, quando a
    interioridade aprendeu a realizar saltos, dançar, maquiar-se,
    expressar-se em abstração e cálculo e paulatinamente se perder de si
    mesma! E como pode um espírito produtivo sobreviver entre um povo
    que não está mais seguro de sua interioridade peculiar e que se
    desfaz, nos cultos, em uma interioridade deseducada e seduzida, e
    nos incultos, em uma interioridade inacessível? Como pode ele
    sobreviver quando a unidade do sentimento do povo se perde, e além
    disso uma parte, justamente a que se denomina a porção culta de um
    povo, e que atribui para si o direito ao espírito artístico
    nacional, sabe que seu sentimento é falsificado e colorido. De vez
    em quando, o julgamento e o gosto do próprio indivíduo poderiam
    tornar-se mais refinados e sublimes --- isso não lhe traz nenhuma
    vantagem: é-lhe igualmente torturante ter de professar uma seita e
    não ser mais necessário para o seu povo. Talvez ele preferisse
    enterrar seu tesouro, porque sente náusea de tornar-se,
    pretensiosamente, o protetor de uma seita, enquanto seu coração se
    enche de compaixão por tudo. O instinto do povo não mais o atinge; é
    inútil estender-lhe nostalgicamente os braços. O que lhe resta agora
    é direcionar seu ódio exaltado contra aquele feitiço inibidor,
    contra as barreiras erguidas na chamada cultura de seu povo, para
    que, como juiz, ao menos condenar aquilo que é, para ele, vivente e
    procriador, aniquilamento e desonra: assim, ele troca a intuição
    profunda de seu destino pelo prazer divino da criação e do auxílio e
    acaba como conhecedor solitário, como sábio mais que saturado. É o
    espetáculo mais doloroso: quem o vê em sua totalidade conhece aqui
    uma necessidade sagrada: ele diz a si mesmo que aqui se precisa de
    ajuda, que aquela unidade superior na natureza e na alma de um povo
    deve ser novamente construída, aquela cratera entre o interno e o
    externo deve desaparecer sob as marteladas da necessidade. Que
    artifícios ele deve usar? Novamente, nele só permanece o seu
    profundo conhecimento, ele tenta plantar uma necessidade nesse
    conhecimento que se expressa, se expande: e, assim, da necessidade
    vigorosa surgirá o ato vigoroso. E com isso não deixo nenhuma dúvida
    de onde tomo exemplo daquela carência, daquela necessidade, daquele
    conhecimento: então devo, expressamente, deixar aqui meu testemunho
    de que é a \emph{unidade alemã} em seu sentido superior que
    ansiamos, e que ansiamos com mais fervor do que a unidade política,
    \emph{a unidade da vida e espírito alemães pelo aniquilamento da
    contradição entre forma e conteúdo, entre interioridade e
    convenção}.

  \chapter*{5}\label{capítulo-5}


A saturação de uma época com a história me parece ser adversa e perigosa,
em relação à vida, em cinco aspectos: com tal excesso cria-se aquele
contraste já mencionado entre interior e exterior, enfraquecendo assim a
personalidade; por conta desse excesso, uma época imagina possuir a mais
rara virtude, a justiça, em maior grau do que outras épocas; por meio
desse excesso, o instinto de um povo é destruído, impedindo o
amadurecimento tanto do indivíduo quanto da totalidade; através desse
excesso, planta-se, a qualquer momento, a crença nociva na velhice da
humanidade, a crença de ser tardio e epígono; graças a esse excesso, uma
época adquire uma perigosa disposição à ironia sobre si mesma e, com
ela, uma disposição ainda mais perigosa ao cinismo; mas, neste caso,
nela amadurece uma práxis egoísta e astuta, que debilita as forças
vitais e por fim as destrói.

E retornando à nossa sentença inicial: o homem moderno padece de uma
personalidade enfraquecida. Assim, como o romano do império tornou-se
não romano no que diz respeito ao que o mundo contemporâneo lhe tinha a
servir, assim como ele se perdeu no influxo de estrangeiros e se
degenerou no carnaval cosmopolita de deuses, costumes e artes, o mesmo
acontece com o homem moderno, que continuamente prepara para si a festa
de uma exposição universal através de seus artistas históricos; ele se
tornou o espectador que aprecia e perambula, chegando a um estado em que
mesmo grandes guerras e revoluções mal podem modificar algo, mesmo que
em um instante. Mal a guerra acabou, já se transformou, aos milhares,
em papel impresso, sendo logo servida como o mais novo aperitivo para o
paladar enfastiado do ávido pela história. Parece quase impossível que
um som potente e cheio se produza, mesmo quando se tocam as cordas com
força: logo ele se amortece, no instante seguinte ele já soa suavemente
histórico, fugidio e fraco. Falando moralmente, vocês não mais conseguem
instituir o sublime, seus atos são toques súbitos, não trovões
retumbantes. Mesmo que se realize o que há de maior e maravilhoso:
tem-se de descer ao Hades, apesar da calada e do silêncio. Pois a arte
foge quando vocês cobrem seus atos com a tenda da história. Quem quer
entender, calcular, compreender, no instante em que deveria suportar,
numa longa convulsão, o incompreensível e o sublime, pode ser chamado de
racional, mas apenas no sentido em que Schiller fala do senso dos
racionais:\footnote{Referência ao poema ``As palavras da fé'', de
  Schiller. Em alemão: ``Verstand der Verständige''.} ele não vê o
mesmo que uma criança vê, não ouve o mesmo que uma criança ouve; mas
esse ``mesmo'' é o mais importante: pois, ao não entendê-lo, seu senso é
mais infantil do que o da criança e mais ingênuo do que a
ingenuidade\footnote{Ao traduzir \emph{Einfalt} (ingênuo) e
  \emph{Einfaltheit} (ingenuidade) perdemos em português a imagem que
  Nietzsche fará em seguida com a palavra \emph{Falte} (dobra).} ---
apesar das pequenas e hábeis dobras que realiza no pergaminho e no
exercício virtuoso de seus dedos ao desembaraçar o embaraçado.
Conclusão: ele destruiu e perdeu seu instinto; não pode mais confiar e
soltar as rédeas do ``animal divino'', se seu entendimento claudica e
seu caminho conduz ao deserto. Assim, o indivíduo se torna temeroso e
inseguro e não pode mais acreditar em si mesmo, ele afunda em si mesmo,
no interior, o que aqui quer dizer apenas: na mixórdia acumulada do que
aprendeu, que não produz efeitos exteriores, do aprendizado que não se
tornou vida. Se se olha para o exterior, percebe-se como o exorcismo dos
instintos recriou o homem quase que numa forma pura de
\emph{abstractis} e sombra. Ninguém mais ousa apresentar-se em sua
pessoa; mascara-se de homem culto, erudito, escritor, político.
Arrancam-se essas máscaras acreditando tratar-se de algo sério, e não
apenas teatro de bonecos --- já que todos eles exibem seriedade ---, e de
repente só se tem nas mãos trapos e farrapos coloridos. Por isso ninguém
deve mais se deixar enganar, por isso se lhes deve recriminar: ``Tirem
suas vestes ou sejam o que parecem ser''. Quem possui aquela seriedade
de nascença não mais deve se tornar um Dom Quixote, já que ele tem mais
o que fazer, em vez de lidar com pretensas realidades. Mas em todo caso
deve observar com atenção cada máscara e dar seu grito de ``Alto lá,
quem vem!'' e tirá-la do rosto. Estranho. Pode-se pensar que a história
humana encoraja sobretudo a se ser \emph{sincero} --- mesmo que uma
tolice sincera; e sempre foi esse seu efeito, mas agora não mais! A
cultura histórica e a vestimenta burguesa universal reinam ao mesmo
tempo. Enquanto ainda não se tenha falado, em tom solene, da
``personalidade livre'', não se veem de fato personalidades, muito menos
livres, mas apenas homens universais escondidos com medo. O indivíduo se
retraiu para o interior: de fora nada se sabe dele; daí a dúvida se é
possível haver causas sem efeitos. Ou seria necessária, para vigiar o
grande harém da história universal, uma geração de eunucos? Certamente a
pura objetividade lhes cai bem. Quase parece como se a tarefa de
preservar a história não fosse senão proteger, que dela se esperasse
estórias, mas não acontecimentos! Que, através dela, se preveniria que
as personalidades se tornassem ``livres'', quer dizer, verazes para si
mesmas, verazes para os outros, sobretudo em palavra e ação. Somente com
essa veracidade viria à luz a carência, a miséria interior do homem
moderno, e, no lugar da convenção e do mascaramento ocultos com temor,
poderiam surgir então, como verdadeiras assistentes, a arte e a
religião, para, juntas, semear uma cultura que corresponda às
verdadeiras necessidades, e não à atual cultura geral, que ensina a
mentir sobre essas necessidades e assim tornar-se uma mentira ambulante.

Em que condições inaturais, artificiais e em todo caso indignas se
encontra, em uma época que padece da cultura geral, a mais veraz de
todas as ciências, a honesta e nua deusa filosofia! Ela permanece,
naquele mundo de uniformidade superficial e forçada, como um monólogo
erudito do passeante solitário, butim fortuito do indivíduo, oculto
segredo de alcova ou conversinha frívola entre criançolas e velhotes
acadêmicos. Ninguém pode ousar seguir, em si, as leis da filosofia,
ninguém vive filosoficamente, com aquela hombridade singela que um
antigo exigia de alguém que se comportasse estoicamente, onde quer que
fosse ou o que fizesse, caso já tivesse jurado lealdade ao estoicismo.\label{lealdadeaoestoicismo}
Todo o filosofar moderno é político e policial, limitado à aparência
erudita por governos, igrejas, academias, costumes e covardias humanas:
ele permanece num suspiro ``Mas se'' ou num conhecimento ``Era uma
vez''. A filosofia não tem direitos no interior de uma cultura
histórica, caso ela queira ser mais que um saber interior e tímido, que
não produz efeitos; fosse o homem moderno corajoso e decidido, não fosse
ele, mesmo em suas inimizades, apenas uma existência interior: ele a
baniria; mas ele se satisfaz em cobrir, envergonhado, sua nudez. Aliás,
pensa-se, escreve-se, publica-se, fala-se, ensina-se filosoficamente ---
até aí quase tudo é permitido; apenas na ação, na assim chamada vida, a
coisa é diferente: aqui sempre só uma coisa é permitida e todo o resto é
simplesmente impossível --- assim o quer a cultura histórica. Ainda são
homens --- pode-se perguntar --- ou talvez apenas máquinas de pensar, escrever
e falar?

Certa vez, Goethe disse de Shakespeare:\footnote{Ensaio ``Shakespeare und
  kein Ende'' (Shakespeare para sempre), de Goethe.} 

\begin{quote}
Ninguém desprezou o figurino material tanto quanto ele; pois conhecia muito bem
o figurino humano interior, e neste todos são iguais. Diz-se que ele
representou os romanos perfeitamente; não acho isso; eles nada mais são
que ingleses encarnados, mas certamente são homens, fundamentalmente
homens, e também as togas romanas lhes caem bem. 
\end{quote}

Agora eu pergunto se
também seria possível representar nossos atuais literatos, cidadãos,
funcionários públicos, políticos, como romanos; não seria possível, pois
eles não são homens, e sim compêndios encarnados e igualmente abstrações
concretas. Se tivessem caráter e natureza própria, enterrariam tudo isso
fundo o suficiente para que não mais emergisse à luz do dia: se fossem
homens, só o seriam para Aquele ``que sonda os rins''.\footnote{Ironia
  que utiliza as passagens do Salmo 7, 10 e de Jeremias 11, 20, nas quais Deus
  é descrito como Aquele que ``sonda o coração e os rins'', no sentido
  de que Ele examina interior e profundamente o caráter de uma
  pessoa.} Para todos os outros, seriam algo diferente, nem homens, nem
deuses, nem animais, e sim produtos de cultura histórica, inteiramente
formação, imagem, forma sem conteúdo comprovável, infelizmente má forma,
e além disso uniforme.\footnote{Não foi possível manter aqui todo o jogo
  que Nietzsche faz com os radicais \emph{Bild} e \emph{Form}: ``Für
  jeden Anderen sind sie etwas Anderes, nicht Menschen, nicht Götter,
  nicht Thiere, sondern historische \emph{Bild}ungsge\emph{bild}e, ganz
  und gar \emph{Bild}ung, \emph{Bild}, \emph{Form} ohne nachweisbaren
  Inhalt, leider nur schlechte \emph{Form}, und überdies
  Uni\emph{form}''.} E assim minha sentença pode ser compreendida e
ponderada: \emph{a história é suportada apenas por personalidades
fortes; ela extingue completamente as fracas}. Nisso repousa o fato de
que ela confunde o sentimento e a sensação naqueles que não são fortes o
suficiente para medir em si mesmos o passado. Aquele que não mais ousa
confiar em si, mas que involuntariamente recorre à história, pedindo-lhe
conselho para seu sentir: ``como devo sentir?'', torna-se
paulatinamente, por medo, um ator que representa um papel, na maioria
das vezes muitos papéis, todos mal e superficialmente. Aos poucos se
perde totalmente a coerência entre o homem e sua esfera histórica; vemos
rapazotes desinibidos lidarem com os romanos como se estes fossem da sua
laia: e eles soterram e enterram o que sobrou dos autores gregos, como
se esses \emph{corpora} lá estivessem para ser dissecados e fossem
\emph{vilia},\footnote{\emph{Corpora vilia} (corpos vis, sem valor).
  Corpos considerados sem valor e por isso utilizados em experimentos.} 
como são os seus próprios \emph{corpora} literários. Supondo que alguém
trabalhe com Demócrito, sempre me vem uma pergunta: por que não
Heráclito? Ou Fílon? Ou Bacon? Ou Descartes, e assim por diante? Por que
não um escritor, um orador? E ainda: por que afinal um grego, por que
não um inglês, um turco? Não é o passado grande o suficiente para que se
encontre algo que não os faça parecer tão ridiculamente arbitrários?
Mas, como dissemos, é uma geração de eunucos; para o eunuco toda mulher
é igual, apenas a mulher, a mulher em si, a eterna inacessível --- assim,
não importa o que fazem, se a história mesma permanecer conservada bela
e ``objetivamente'' por aqueles que nunca poderão por si mesmos fazer
história. E como nunca serão elevados pelo eterno feminino,\footnote{Menção
  ao final do segundo livro do \emph{Fausto}, de Goethe: ``O eterno feminino
  nos eleva''.} afastam-se dele e tomam, neutros, também a própria
história como neutra. Com isso, não se acredite que comparo seriamente
a história com o eterno feminino; quero, ao contrário, expressar com
clareza que a considero o eterno masculino, que, para aqueles que são
inteiramente ``cultivados em assuntos históricos'', é indiferente ser
homem ou mulher, ou mesmo a comunhão de ambos; eles são sempre neutros
ou, em termos cultos, eternos objetivos.

As personalidades são assim apagadas, da forma apontada, até se tornarem
a eterna falta de subjetividade ou, como se diz, objetividade: não mais
adianta provocá-las; se algo bom e justo acontece, como ação, obra
literária ou música, logo o homem soterrado pela cultura desvia os olhos
da obra e pergunta pela história do autor. Caso este já tenha realizado
mais obras, logo devem ser interpretados os passos anteriores e os
prováveis passos posteriores de seu desenvolvimento, logo ele é
comparado a outros autores, a escolha de seus temas e seu tratamento
devem ser dissecados, desmembrados, sabiamente reunidos de uma nova
maneira e censurados e repreendidos em sua totalidade. A coisa mais
impressionante pode acontecer, sempre o bando das neutralidades
históricas estará, de longe, pronto para observar o autor. Num instante
ressoa o eco: mas sempre ``crítico'', enquanto, pouco antes, o crítico
nem sonhasse com a possibilidade de tal acontecimento. Em nenhum lugar
se chega a algum efeito, mas sempre a uma ``crítica''; e a própria
crítica não produz efeito, apenas experimenta a crítica novamente. Daí o
acordo em considerar muitas críticas como um sucesso
{[}\emph{Wirkung}{]}, poucas como um fracasso.\footnote{\emph{Wirkung},
  normalmente traduzido por ``efeito'', recebeu aqui outra acepção
  possível (``sucesso'', no sentido de ``bom resultado''), para
  esclarecer o jogo de palavras aqui presente.} Mas no fundo permanece,
mesmo em tal ``efeito'' {[}\emph{Wirkung}{]} obtido, algo de antigo:
embora se tagarele tanto tempo sobre o novo, nesse ínterim novamente se
faz o que sempre foi feito. A cultura histórica de nossos críticos não
mais permite que ocorra um efeito no seu sentido próprio, ou seja, um
efeito na vida e na ação; eles passam seu mata-borrão na tinta mais
escura dos escritos, passam seu pincel grosso sobre as palavras mais
graciosas, como se fossem correções: aquilo já é novamente passado. A
pena dos críticos nunca para de escrever, pois eles perderam o controle
sobre ela; eles não a conduzem, são por ela conduzidos. Justamente nessa
falta de medida de sua efusão crítica, nessa ausência de domínio sobre
si mesmos, que os romanos chamavam \emph{impotentia}, revela-se a
fraqueza da personalidade moderna.

\chapter*{6}\label{capuxedtulo-6}

Mas deixemos de lado essa fraqueza. Façamos uma questão incômoda a uma
notória força do homem moderno: tem ele o direito de se designar, pela
sua conhecida ``objetividade'' histórica, como forte, isto é,
\emph{justo}? E justo em maior grau do que os homens de outras épocas? É
verdade que essa objetividade tem sua origem numa crescente necessidade
e exigência por justiça? Ou ela desperta, como efeito de outras causas,
a aparência de que a justiça seja a verdadeira causa desse efeito? Ela
não seduz, talvez, a uma presunção danosa, excessivamente lisonjeira, a
respeito das virtudes dos homens modernos? --- Sócrates considerava uma
enfermidade próxima da loucura imaginar possuir uma virtude que não se
possui: e certamente tal delírio é mais perigoso do que o desvario
oposto de pensar cometer um pecado, um vício. Pois por meio desse
desvario talvez seja possível se tornar melhor; mas aquele delírio torna
os homens ou sua época, a cada dia, piores, ou seja --- nesse caso, injustos.

Certamente, ninguém tem aquela pretensão à reverência em maior grau do
que aquele que possui a força e o anseio de justiça. Pois nela unem-se e
ocultam-se as maiores e mais raras virtudes, como um mar abissal que
produz e absorve correntes por todos os lados. A mão do justo foi feita
para julgar; não treme quando segura a balança; decidido, pondera,
malgrado a si mesmo, peso por peso; seu olhar não se turva quando os
pratos da balança sobem e descem, e sua voz não soa nem forte nem
alquebrada quando anuncia o veredicto. Se fosse um demônio do
conhecimento, ele transmitiria a atmosfera gélida de uma majestade
sobrenatural e terrível, que teríamos de temer e não venerar; mas como
ele é humano, tenta galgar de uma dúvida perdoável a uma certeza
rigorosa, de uma brandura tolerante a um imperativo ``tu deves'', da
rara virtude da magnanimidade à mais que rara virtude da justiça, ele
agora se assemelha àquele demônio, sem ser, desde início, nada mais que
um pobre homem, que em cada instante cobra a si mesmo sua humanidade e
tragicamente se consome por uma virtude impossível --- isso tudo o coloca
em uma altura solitária, como \emph{o mais venerável} exemplar da
espécie humana; pois ele quer a verdade, mas não como um frio
conhecimento sem consequências, mas como uma juíza ordenadora e
punitiva; verdade não como posse egoísta de indivíduo, mas como a
licença sagrada de mover todas as barreiras da posse egoísta; verdade,
em uma palavra, como justiça universal e não algo como caça e satisfação
de um caçador. Assim, na medida em que aquele que é veraz possui a
vontade incondicionada de ser justo, esse anseio glorificado e impensado
pela verdade é geralmente algo grandioso: enquanto diante do olhar
apático aflui um grande número de diversos impulsos, como a curiosidade,
o medo do tédio, a inveja, a vaidade, o prazer do jogo, impulsos que
nada têm a ver com a verdade, com aquele anseio pela verdade. É assim
que, embora o mundo esteja repleto de tais ``serviçais da verdade'', é
raro a virtude da justiça estar presente; ela é pouco conhecida e quase
sempre odiada de morte: ao contrário, o bando dos aparentes virtuosos de
toda época é venerado e amado ostensivamente. Na verdade, poucos servem
à verdade, pois apenas poucos possuem a vontade pura de ser justos e,
mesmo entre estes, pouquíssimos possuem a força de poder ser justos.
Para isso, não basta apenas possuir a vontade: e os mais terríveis
sofrimentos dos homens provêm justamente do impulso de justiça aliado à
incapacidade de julgar. Por isso, para o bem-estar geral, nada mais se
exigiria senão espalhar as sementes da faculdade de julgar da forma mais
ampla possível, para diferenciar o fanático do juiz, o desejo cego de
ser juiz da força consciente de poder sê-lo. Mas onde se encontraria
o solo para plantar a faculdade de julgar! Por isso, quando falam de
verdade e justiça para os homens, estes sempre ficam indecisos, sem
saber se quem fala com eles é um fanático ou um juiz. Por essa razão, deve-se
desculpá-los por sempre saudarem, com particular benevolência, aqueles
``serviçais da verdade'', que não possuem nem a vontade nem a força de
julgar e assumem a tarefa de procurar o conhecimento ``puro, sem
consequências'' ou, mais precisamente, a verdade que nada produz.
Existem muitas verdades indiferentes; existem problemas cujo correto
julgamento não depende de superação, muito menos de sacrifício. Nessas
esferas indiferentes e inofensivas, ocorre de um homem se tornar um frio
demônio do conhecimento, apesar disso! Se mesmo em épocas
particularmente favorecidas legiões inteiras de eruditos e pesquisadores
se transformaram nesses demônios, infelizmente, ainda é possível que
nesta época careça-se da justiça grande e rigorosa, em resumo, do núcleo
mais nobre do assim chamado impulso à verdade.

Que se coloque diante dos olhos o virtuose histórico da atualidade: é
ele o mais justo dos homens de sua época? É verdade que ele cultivou
aquela delicadeza e suscetibilidade da sensação de que nada humano lhe é
distante: as épocas e pessoas mais diversas ecoam, em sua lira, em tons
harmoniosos; ele se tornou um passivo ressonador que ecoa, que com seu
eco provoca outros passivos similares, até que esses ecos vibrantes,
delicados e harmoniosos preencham confusamente o ar de uma época. Mas me
parece que se escutam apenas os harmônicos mais agudos da sonoridade
histórica original: não se pode mais adivinhar, a partir do som agudo e
estridente das cordas, a força e o poder do original. O tom original
despertava, na maioria das vezes, atos, necessidades, horrores; agora
ele nos entorpece e nos transforma em apreciadores indolentes: é como se
a sinfonia \emph{Eroica} tivesse sido arranjada para duas flautas, para
o desfrute de opiômanos delirantes. Daí se pode medir como as mais altas
pretensões do homem moderno para a justiça superior e pura se apresentam
para esses virtuoses; essa virtude nada tem de agradável, não conhece
nenhuma oscilação estimulante; é rígida e terrível. Em sua régua, a
magnanimidade está na parte inferior da escala da virtude, a
magnanimidade que é a característica de um historiador raro e singular!
Mas muitos alcançam a tolerância, a indiferença e a edulcoração
massificada e benevolente, na arguta suposição de que o homem
inexperiente interpreta como virtude da justiça a narração do passado
sem uma entonação áspera e sem expressão de ódio. Mas apenas a força
superior pode julgar; o fraco tem de tolerar, caso não finja ser forte e
não queira fazer a justiça posar de comediante na cadeira do juiz. Resta
ainda uma temível espécie de historiador, de caráter bravo, rigoroso e
honesto --- mas de cabeça estreita ---, aqui a boa vontade é justa e se
apresenta como o \emph{páthos} da magistratura, mas suas sentenças são
equivocadas, mais ou menos como, pelos mesmos motivos, são equivocadas
as sentenças de júris comuns. Como é também improvável a ocorrência do
talento histórico! Deixando de lado, aqui, os egoístas e partidários
disfarçados que, fingindo, fazem feições justas e objetivas. Excluindo
também as pessoas totalmente desarrazoadas, que ingenuamente escrevem,
como historiadores, que sua época tem razão em todos os pontos de vista
populares e que escrever de acordo com sua época significa ser justo;
uma fé que anima toda religião e sobre a qual, nas religiões, nada mais
temos a dizer. Esses historiadores ingênuos chamam ``objetividade''
medir as opiniões e os fatos passados a partir das opiniões difundidas
no momento; aqui eles encontram o cânone de toda verdade: seu trabalho é
ajustar o passado à trivialidade atual. Caso contrário, chamam de
subjetivista toda historiografia que não toma a opinião popular como
canônica.

Não poderia mesmo subjazer, no sentido mais elevado da palavra
objetividade, uma ilusão? Pois se entende com essa palavra um estado em
que o historiador enxerga, em um evento, todos seus motivos e
consequências, de forma tão pura que não afeta sua subjetividade.
Pense-se naquele fenômeno estético, naquela libertação de interesse
pessoal com que o pintor vê, em uma paisagem tempestuosa, um mar revolto
sob raios e trovões, sua imagem interior, quer dizer, a imersão completa
nas coisas: é contudo uma superstição dizer que a imagem com que as
coisas se apresentam a esse homem, assim forjado, reproduzisse a
existência empírica das coisas. Ou as coisas deveriam, em todo momento,
em sua atividade, serem igualmente desenhadas, retratadas, fotografadas em
uma passividade pura?

Isso seria uma mitologia, e das ruins além disso. Nela se esqueceria que
todo momento é, no interior do artista, justamente o momento criador
mais forte e ativo, um momento de composição do tipo mais superior, cujo
resultado será uma pintura artisticamente verdadeira, e não
historicamente verdadeira. Desse modo, pensar a história objetivamente é
o trabalho silencioso do dramaturgo; ou seja, pensar tudo em correlação,
tecer o particular em um todo --- com o pressuposto geral de que se deveria
colocar a unidade do plano nas coisas, caso já não esteja nelas. Assim o
homem inventa o passado e o exorciza, assim seu impulso artístico se
exterioriza --- mas não o seu impulso à verdade e à justiça. Objetividade
e justiça não têm nada a ver uma com a outra. Seria possível pensar em
uma historiografia que não tivesse em si nenhuma gota de verdade
empírica comum e, contudo, pretendesse receber o predicado de objetividade
em seu mais alto grau. Aliás, Grillparzer ousa explicar:

\begin{quote} 
O que é a história senão a forma como o espírito do homem assimila os
\emph{eventos que lhe são impenetráveis}; aquilo que, Deus sabe como,
faz correlações. O incompreensível é substituído pelo compreensível;
impõe seu conceito de finalidade externa a uma totalidade que só conhece
a finalidade interna; admite sempre o acaso, onde atuaram milhares de
pequenas causas. Todo homem tem igualmente sua necessidade individual,
de tal modo que milhões de direções correm, paralelamente, em linhas retas
e oblíquas, que se cruzam, se impedem, se impelem para frente e para
trás e assim admitem, entre si, o caráter de acaso e de tal modo que,
excetuando as intervenções dos eventos naturais, tornam impossível
provar uma necessidade apreensível e abrangente daquilo que ocorre.
\end{quote}

Mas é justamente essa necessidade que aquele olhar ``objetivo'' das
coisas deve trazer à luz! Isso é um pressuposto que, quando é expresso
como profissão de fé do historiador, só pode ser admitida como uma forma
estranha; Schiller expôs com toda clareza e propriedade a subjetividade
dessa admissão, quando diz do historiador: ``um fenômeno começa, um
atrás do outro, a se soltar da imprecisão cega e da liberdade sem regras
e a se organizar em um todo coerente --- \emph{que certamente só existe
em sua imaginação} {[}\emph{Vorstellung}{]} --- como um elo que se
enfileira''.\footnote{Schiller, ``O que significa e para que fim se
  estuda a história universal''. \emph{Vorstellung}, normalmente
  traduzido por representação, foi vertido aqui por imaginação.} O que
dizer, contudo, desta afirmação de um célebre virtuose da história,
expressa com fé, oscilando artificialmente entre a tautologia e o
absurdo: ``não é um fato inquestionável que toda ação e impulso humanos
são submetidos ao silencioso e nem sempre percebido, mas violento e
ininterrupto curso das coisas?'' Em tal frase se percebe tanto uma
verdade enigmática quanto uma inverdade patente; como o dito do
jardineiro goethiano: ``a natureza se deixa forçar, mas não se
coagir'',\footnote{\emph{Carta de Goethe a Schiller,} de 21/02/1798.} ou na
inscrição de uma barraca de quermesse, contada por Swift: ``Aqui se
encontra, com exceção de si mesmo, o maior elefante do mundo''. Qual é,
contudo, a oposição entre ação e impulso humanos e o curso das coisas?
Ocorre-me que historiadores como aqueles cuja frase citamos não mais
ensinam, assim que universalizam, e mostram o sentimento de sua fraqueza
no escuro. Em outras ciências, a universalidade é o mais importante, uma
vez que ela contém as leis: se a frase por nós mencionada valesse como
lei, então se teria de retrucar que o trabalho historiográfico
desapareceria; pois o que sobra de verdade na frase, com a retirada do
resto obscuro e indissolúvel, sobre o qual falamos, é algo conhecido e
mesmo trivial; pois ocorre a todos na pequena esfera da experiência. Por
isso, o incômodo de povos inteiros e anos de trabalho exaustivo nada
significaram, nas ciências naturais, senão o acúmulo de experimento
atrás de experimento, muito depois que a lei foi deduzida do tesouro da
experiência; aliás, segundo Zöllner,\footnote{Johann Karl Friedrich
  Zöllner (1834--82), astrônomo e físico alemão.} a ciência natural
atual padece dessa absurda desmesura do experimento. Se o valor de um
drama residisse no desfecho e nas ideias principais, o próprio drama
seria possivelmente um caminho longo e sinuoso em direção a um fim; e,
assim espero, a história não deve encontrar seu significado nas ideias
universais, como uma espécie de florada e fruto; seu valor reside
justamente em reescrever, de forma engenhosa, um tema conhecido e mesmo
habitual, uma melodia ordinária, erguê-lo, alçá-lo a um símbolo
abrangente, percebendo, no tema original, toda profundeza, poder e
beleza.

Para isso, é preciso sobretudo uma grande potência artística, um pairar
criador sobre as coisas, uma imersão apaixonada nos dados empíricos, uma
poetização de tipos dados, a isso pertence a objetividade, mas como
qualidade positiva. No entanto, frequentemente a objetividade é
tão somente um chavão. No lugar daquela serenidade, internamente
lampejante e externamente imóvel e obscura, do olhar artístico, surge a
afetação de serenidade; como a ausência de \emph{páthos} e de força
moral costuma se travestir de frieza penetrante da observação. Em certos
casos, a banalidade do pensamento, a sabedoria banal, que apenas por seu
tédio causa a impressão de serenidade, ousa apresentar-se como
tranquilidade, a fim de valer como aquele estado artístico em que o
sujeito silencia e se torna completamente imperceptível. Então, é ansiado
tudo o que não intranquiliza e a mais seca palavra é justa. Aliás,
chega-se ao ponto de se consentir que se tenha como ocupação a
representação de um momento do passado que \emph{em nada lhe importa}.
Assim comportam-se frequentemente os filólogos em relação aos gregos: a
eles nada interessa --- isso então é chamado de ``objetividade''! Onde o
mais elevado e raro deve ser representado, há o desinteresse intencional
e explícito, a arte superficial da motivação, sobriamente escolhida;
justamente isso revolta --- quando a \emph{vaidade} do historiador o
impele a ostentar essa indiferença como objetividade. Além disso, no
trato com esse autor, deve-se julgar segundo o princípio de que todo
homem tem justamente tanta vaidade quanto lhe falta entendimento. Não,
sejam ao menos honestos! Vocês não procuram a ilusão da força artística
que se pode efetivamente chamar objetividade, não procuram a ilusão de
justiça, se não são consagrados para a terrível vocação do justo. Como
se a tarefa de toda época fosse a de ser justo com tudo que já existiu!
Nunca épocas ou gerações tiveram direito a ser juízas de todas as épocas
e gerações passadas: ao contrário, somente aos indivíduos, justamente
aos mais raros, ocorreu missão tão desconfortável. Quem os coage a
julgar? E então, provem apenas que podem ser justos quando quiserem!
Como juízes, devem estar acima do réu; contudo, só chegam tardiamente.
Os convidados que chegam por último à mesa devem, com razão, receber os
últimos lugares; e querem ter os primeiros? Façam ao menos o mais
elevado e grandioso; talvez assim lhes sejam oferecidos os primeiros
lugares, mesmo quando chegarem por último.

\emph{Apenas da força superior do presente lhes é permitido interpretar
o passado}: apenas sob a pressão de suas qualidades mais nobres
adivinharão o que do passado é digno de conhecer e preservar. De igual
para igual! Senão rebaixarão o passado a si mesmos. Não creiam em uma
historiografia se ela não provier do maior dos espíritos mais raros; mas
sempre perceberão que característica tem seu espírito quando ele precisa
expressar algo universal ou repetir algo conhecido por todos: o
historiador atento deve ter a força de transformar o conhecido por todos
no inaudito e expressar o universal de forma tão simples e profunda que
não se veja nem a simplicidade acima da profundidade nem a profundidade
acima da simplicidade. Ninguém pode ser ao mesmo tempo um grande
historiador, um homem artístico e um cabeça-oca: ao contrário, não se
deve subestimar os trabalhadores braçais que carregam, extraem e
examinam pela certeza de que eles não podem tornar-se grandes
historiadores; muito menos confundi-los com estes últimos, mas
compreendê-los como aprendizes e ajudantes braçais a serviço do mestre,
assim como os franceses, com mais ingenuidade do que os alemães,
costumam falar dos \emph{historiens de} M. Thiers.\footnote{Lous A.
  Thiers (1797--1877), historiador francês. O sentido da expressão em
  francês é: ``os historiadores que trabalham para M. Thiers''.} Esses
trabalhadores braçais devem paulatinamente tornar-se eruditos, mas por
isso nunca poderão ser mestres. Um grande erudito e um grande cabeça-oca
-- duas coisas que combinam.

Portanto, o homem experimentado e superior escreve a história. Quem
nunca viveu algo elevado e grandioso não saberá interpretar o elevado e
o superior do passado. A sentença do passado é sempre a sentença de um
oráculo: a qual, apenas como arquitetos do futuro, sábios do presente,
vocês poderão entender. O efeito de Delfos, extraordinariamente profundo
e vasto, explica-se pelo fato de que seus sacerdotes eram conhecedores
acurados do passado; agora não convém saber que apenas aquele que
constrói o futuro tem o direito de julgar o passado. Vejam adiante,
imponham-se um grande objetivo e domem ao mesmo tempo aquele impulso
analítico exagerado que desertifica o presente e impossibilita toda
tranquilidade, todo crescimento pacífico e amadurecimento. Puxem para si
a cerca de uma esperança grande e abrangente de um impulso esperançoso.
Formem em si uma imagem que deva corresponder ao futuro e esqueçam a
superstição de serem epígonos. Os senhores têm bastante sobre que refletir
e inventar quando refletem sobre a vida futura; mas não perguntem à
história se ela mostra o ``como?'', o ``porquê?''. Se os senhores, ao
contrário, viverem a história dos grandes homens, aprenderão o
mandamento superior de se tornar maduro e de fugir daquele feitiço
paralisante da pedagogia da época, cuja utilidade reside em não deixar
que se amadureça, a fim de dominar e pilhar os imaturos. E desejem
biografias que não tragam na capa o bordão ``Senhor fulano de tal e sua
época'', mas sim ``um guerreiro contra seu tempo''. Alimentem sua alma
com Plutarco e ousem acreditar em si mesmos como acreditam em seus
heróis. Com uma centena de homens educados dessa maneira antimoderna,
isto é, maduros e habituados ao heroico, pode-se calar toda a cultura
baixa e barulhenta desta época.

\chapter*{7}\label{capuxedtulo-7}

O sentido histórico, quando reina \emph{de forma incontrolada} e extrai
todas as suas consequências, extingue o futuro, pois destrói as ilusões
e rouba das coisas existentes a atmosfera sem a qual não podem viver. A
justiça histórica, mesmo quando é realmente exercida e em pura
conscienciosidade, é por isso mesmo uma virtude terrível, por sepultar e
sabotar o que vive: seu julgar é sempre um aniquilar. Se atrás do
impulso histórico não imperar nenhum impulso construtivo; se não
destruir e se dispuser a construir, com esperança, sua casa, seu futuro
sobre um solo livre; se a justiça reinar sozinha, então o instinto
criador é enfraquecido e desanimado. Por exemplo, uma religião que, sob
o reino da justiça, se tornasse saber histórico, uma religião que fosse
conhecida cientificamente em sua totalidade se destruiria no fim do
caminho. A razão disso reside no fato de que surgem, na contabilidade da
história, tanta falsidade, crueza, inumanidade, absurdo e violência, que
necessariamente se dissipa o ânimo ilusório e piedoso somente no qual
tudo o que quer viver pode viver: mas apenas no amor, apenas nas sombras
da ilusão do amor é que o homem cria, ou seja, na incondicional crença
no perfeito e justo. Quem é obrigado a não mais amar tem as raízes de
sua força arrancadas de si: ele apodrece, ou seja, torna-se ímprobo.
Nesses efeitos a história contrapõe-se à arte: talvez somente quando a
história suportar transformar-se em obra de arte, ou seja, em pura forma
artística, ela poderá conservar ou mesmo despertar os instintos. Essa
historiografia, contudo, contradiz o ímpeto analítico e inartístico de
nossa época, e é até mesmo sentida como falsificação. Contudo, a
história que apenas destrói, sem ter um impulso criador a conduzi-la,
torna-se, com o tempo, pedante e inatural: pois esses homens destroem as
ilusões e ``a natureza, como a mais dura tirana, pune quem destrói as
ilusões em si e nos outros''.\footnote{Goethe, ``Fragmento sobre a
  natureza''.} Pode-se até lidar com a história de maneira inofensiva
e discreta, como se fosse uma ocupação como qualquer outra. Os novos
teólogos, particularmente, parecem aprovar a história pura, devido ao
seu caráter inofensivo, mal percebendo que com isso, e provavelmente
contra a própria vontade, estão a serviço do \emph{écrasez}
voltairiano.\footnote{Referência ao ``destruam a infame'', ou seja, a
  Igreja.} Ninguém supunha atrás disso um instinto construtor forte;
poder-se-ia então considerar a dita União Protestante\footnote{Instituição
  alemã, fundada no século \textsc{xix}, que unia diversos segmentos da igreja
  reformada.} como o útero de uma nova religião, e um jurista como
Holtzendorf\footnote{Franz von Holtzendorf (1829--89), jurista alemão.}
(como editor e prefaciador da dita Bíblia protestante), o João do Rio
Jordão. Talvez essa época auxilie a filosofia hegeliana, que ainda
tortura as velhas cabeças, na propagação daquela inocuidade,
diferenciando a ``Ideia do cristianismo'' de suas múltiplas e
imperfeitas formas de manifestação e convencendo como a ``paixão'' da
Ideia se revela de forma cada vez mais pura, transparente e até pouco
visível aos cérebros dos atuais \emph{theologus liberalis
vulgaris}.\footnote{Teólogo liberal comum.} Se ouvir o que esses
cristianismos puríssimos têm a falar dos antigos cristianismos impuros,
um ouvido imparcial terá a impressão de que o assunto não é o
cristianismo, mas o que devemos pensar disso? Quando vemos o
cristianismo ser designado pelos ``maiores teólogos do século'' como a
religião que se pode sentir no interior de todas as religiões existentes
e ainda nas meramente possíveis e quando a ``Igreja verdadeira'' é
aquela que ``se torna uma massa fluida e sem limites, onde cada parte se
encontra aqui e acolá e tudo se mistura pacificamente'', mais uma vez,
o que devemos pensar disso?

O que se pode aprender do cristianismo é que ele, sob o efeito de um
tratamento histórico, torna-se pedante e inatural, até o ponto em que um
tratamento inteiramente histórico, isto é, justo, dissolve-o em um saber
puro sobre o cristianismo e o destrói, o que pode ser estudado em tudo o
que é vivo: que deixa de viver quando é dissecado e vive enfermo e
lacerado quando aplica em si a dissecação histórica. Há homens que
acreditam em um poder curador, transformador e reformador da música
alemã: eles se enraivam, acham injusto e um crime contra o que há de
mais vivo em nossa cultura, quando homens como Mozart e Beethoven são
assolados por todo um amontoado das biografias eruditas e, graças ao
método de tortura da crítica histórica, são constrangidos a responder a
milhares de perguntas impertinentes. Não chegando a levar à exaustão
seus efeitos vitais extemporâneos ou ao menos imobilizá-los, o fato é
que a curiosidade ávida se dirige aos inúmeros micrólogos da vida e obra
e procura problemas de conhecimento onde se devia aprender a viver e a
esquecer todos os problemas. Se imaginássemos transportar meia dúzia
desses biógrafos modernos para o nascedouro do cristianismo ou da
reforma luterana, sua curiosidade ávida, sóbria e pragmática teria
bastado para tornar impossível qualquer \emph{actio in
distans}\footnote{Ação a distância.} espiritual: como o mais miserável
dos animais pode evitar o surgimento de um carvalho ao engolir seus
frutos. Tudo o que é vivo precisa ter em torno de si um círculo de névoa
e mistério; se lhe tomam esta névoa, se uma religião, uma arte, um gênio
é condenado a girar como um astro sem uma atmosfera: não se deve admirar
do seu rápido apodrecimento, tornando-o duro e estéril. Assim ocorre com
todas as grandes coisas, ``que não prosperam sem alguma ilusão'', como
afirma Hans Sachs em \emph{Os mestres-cantores}.

Mas mesmo um povo, um homem que queira \emph{amadurecer} necessita dessa
ilusão nevoenta, dessa nuvem envolvente e protetora; hoje em dia se
odeia o amadurecimento porque se venera mais a história do que a vida.
Aliás, hoje é vangloriado o fato de que ``a ciência começa a dominar a
vida'': é possível que se chegue a isso, mas a vida assim dominada não
tem muito valor, pois é menos \emph{vida} e garante menos vida para o
futuro do que outrora, quando se dominava a vida não pelo saber, mas por
instintos e fortes alucinações. Mas esta não deve ser, como dissemos,
uma época de personalidades harmoniosas, perfeitas e maduras, mas a do
trabalho mais ordinário e mais útil possível. Isso significa que os
homens devem direcionar-se aos propósitos da época para trabalhar o mais
cedo possível. Eles devem trabalhar na fábrica das utilidades universais
antes de se tornar maduros --- porque seria um luxo dispensar do
``mercado de trabalho'' uma grande quantidade de força. Cegam-se alguns
pássaros para que eles cantem melhor; não acredito que os homens de hoje
cantem melhor do que os de outrora, mas sei que se cegam na atualidade.
Mas o instrumento, o terrível instrumento que utilizam para cegar é
\emph{uma luz por demais rútila, súbita e cambiante}. O homem jovem é
chicoteado por séculos: jovens que não entendem de guerra, de ação
diplomática, de política comercial, são considerados aptos à introdução
à história política. Assim como os jovens passeiam na história, nós,
modernos, passeamos pelas galerias de arte e ouvimos concertos. Bem se
sente que algo soa diferente, que algo provoca coisas diferentes: perder
esse estranhamento, não mais se surpreender excessivamente, enfim, tudo
tolerar --- isso se chama sentido histórico, cultura histórica.
Expressando-me sem floreios: a massa de afluentes é tão grande, o
estrangeiro, o bárbaro e o violento, ``espremidos em um monstruoso
torrão'',\footnote{Citação de ``O mergulhador'', de Schiller.} 
pressionam com tanta força a alma juvenil, que ela só sabe se salvar com
uma estupidez calculada. Onde repousa uma consciência refinada e forte,
encontra-se também uma outra sensação: o nojo. O jovem se tornou um
apátrida, duvidando de todos os costumes e conceitos. Agora ele sabe: em
todos os tempos as coisas eram diferentes, não importando como você
seja. Em sua insensibilidade apática, ele deixa passar por si
pensamentos após pensamentos e entende as palavras de Hölderlin sobre a
leitura da vida e obra dos filósofos de Diógenes Laércio:
``Experimentei, novamente aqui, o que às vezes já me havia sucedido, a
saber, que o que há de passageiro e mutável dos pensamentos e sistemas
humanos me tinha atingido de forma quase mais trágica do que os
destinos, que frequentemente são considerados a única
realidade''.\footnote{\emph{Carta de Hölderlin a Isaak v. Sinclair,} de 24/12/1798.}
Não, tal história transbordante, ensurdecedora e violenta certamente não
é necessária para a juventude, como mostravam os antigos, sendo até
mesmo perigosa, como mostram os modernos. No entanto, observem os
estudantes de história, são herdeiros de um pedantismo precoce,
perceptível desde garotos. Para eles, o ``método'' é seu próprio
trabalho: a apreensão correta e o tom solene que deve copiar dos
mestres; um pequeno capítulo isolado do passado é sacrificado por sua
perspicácia e pelo método ensinado; ele já produziu ou, usando palavras
mais vaidosas, ele já ``criou''; ele se tornou um serviçal da história
por intermédio dos fatos e um senhor no campo da história. Ele já estava
``pronto'' desde garoto, agora recebe apenas o acabamento; basta
sacudi-lo para a verdade cair-lhe ruidosamente no colo; mas a verdade
está podre e toda maçã carrega consigo o seu bicho. Acreditem em mim: se
um homem deve trabalhar na fábrica da ciência e ser útil antes de
amadurecer, logo também a ciência se arruinará, como os modernos
escravos dessa fábrica. Lamento o jargão e o uso de palavras como senhor
de escravos e empregador para designar essas relações, que deviam ser
pensadas como livres de toda utilidade e carência: mas involuntariamente
me escapam palavras como ``fábrica, mercado de trabalho, oferta,
utilidade'' --- que soam como verbos auxiliares do egoísmo --- quando
esboço a mais jovem geração de eruditos. A mediocridade sólida torna-se
cada vez mais medíocre; a ciência, em sentido econômico, cada vez mais
útil. De fato, os novíssimos eruditos são sábios em apenas um ponto, no
qual são mais sábios do que todos os homens do passado; nos outros
pontos, são infinitamente diferentes --- dito com cautela --- em relação a
todos os eruditos de antiga linhagem. Apesar disso, exigem honras e
vantagens para si, como se o Estado e a opinião pública fossem obrigados
a aceitar as novas moedas tanto quanto as antigas. Os trabalhadores
braçais fecharam um contrato e decretaram o gênio como dispensável --
determinando que todo trabalhador braçal é um gênio: provavelmente um
tempo posterior verá que suas construções foram mais ajuntadas que
construídas. Pode-se dizer àqueles que incansavelmente entoam as
convocações à guerra e ao sacrifício, ``Participação no trabalho! Em
fila!'', em alto e bom som: se quiserem fomentar a ciência o mais rápido
possível, então que a destruam o mais rápido possível; como destruíram a
galinha que obrigam a pôr ovos artificial e rapidamente. É certo que a
ciência recebeu um incentivo surpreendentemente rápido nas últimas
décadas: mas observem os eruditos, são galinhas exaustas. Não são
verdadeiramente naturezas ``harmônicas'': apenas cacarejam mais do que
antes, por porem seus ovos publicamente; certamente os ovos são cada vez
menores (embora os livros sejam cada vez mais grossos). Como resultado
último e natural, alcança-se a amada ``popularização'' (ao lado da
``afeminação'' e da ``infantilização'') da ciência, isto é, o infame
corte do tecido da ciência de acordo com as medidas do ``público
diversificado'': a fim de nos utilizarmos de um alemão de alfaiataria para
atividade própria à alfaiataria. Goethe via nisso um mau uso e exigia
que a ciência só devesse exercer efeito no mundo exterior através de uma
\emph{práxis superior}. Além disso, para as gerações mais antigas de
eruditos, esse mau uso parece, por boas razões, ser algo difícil e
perturbador: igualmente por boas razões, ele é fácil para os jovens
eruditos, pois eles mesmos, exceto por uma pequena faceta do
conhecimento, pertencem ao público diversificado e compartilham suas
necessidades. Eles precisam se acomodar confortavelmente, e assim
conseguem adequar seu pequeno campo de estudo àquela curiosidade popular
geral. Para esse ato de acomodação forjam o nome de ``modesta
condescendência do erudito para com seu povo''; enquanto, no fundo, o
erudito apenas se rebaixa a si mesmo, na medida em que não é um erudito,
mas um plebeu. Criem para si o conceito de ``povo'': nunca poderão
pensar nele de forma nobre e suficientemente elevada. Se vocês tivessem
grandeza ao pensar no povo, seriam compassivos para com ele e se
guardariam de oferecer-lhe sua água-forte como bebida vital e
fortificante. Mas no fundo vocês pensam nele de forma estreita, pois não
podem ter nenhum respeito verdadeiro e seguro do futuro e se comportam
como pessimistas práticos, quero dizer, como homens que têm a percepção
de um declínio e com isso se tornam indiferentes e relapsos em relação
ao estranho e mesmo ao próprio bem. Se o chão ainda nos suportar! E se
ele não mais nos suportar, tudo bem --- assim eles sentem e vivem uma
existência \emph{irônica}.

\chapter*{8}\label{capuxedtulo-8}

Embora possa parecer estranho, não é contudo contraditório quando
atribuo, a uma época que costuma manifestar-se de forma tão gritante e
insistente sobre o triunfo de sua cultura histórica, uma espécie de
\emph{autoconsciência irônica}, uma percepção pairando de que aqui não
há triunfo, de que em breve ela sucumbirá com todo o deleite do
conhecimento histórico. Um enigma semelhante, no que diz respeito a
personalidades particulares, nos é colocado por Goethe, em sua notável
caracterização de Newton: ele encontrou, na base (ou melhor, no ápice)
de seu ser, ``uma compreensão turva de seu erro'', como que, num
instante singular, a expressão perceptível de uma consciência reflexiva
e orientada tivesse alcançado uma certa visão panorâmica irônica da
natureza que lhe era necessariamente interior. Assim se encontra,
justamente nos maiores e mais desenvolvidos homens históricos, uma
consciência embotada, que beira o ceticismo generalizado, do quanto há
de absurdo e superstição na crença de que a educação de um povo deva ser
pesadamente histórica, como ocorre hoje; pois justamente os povos mais
fortes, sobretudo em atos e obras, viveram e educaram sua juventude de
forma diferente. Mas aquele absurdo e aquela superstição nos pertencem
-- assim nos objeta o cético --, nós, os últimos e pálidos rebentos de
gerações poderosas e exultantes, podemos interpretar a profecia de
Hesíodo de que os homens nasceriam vetustos e que Zeus destruiria essa
geração assim que esse sinal ficasse visível. A cultura histórica é de
fato uma espécie de vetustez inata, e aqueles que trazem consigo os
sinais da infância devem atingir a crença instintiva na
\emph{antiguidade humana}; mas agora a antiguidade é uma ocupação de
idosos, ou seja, ter lembranças, olhar para trás, fazer contas, concluir
e buscar consolo na convalescença; em resumo, a cultura histórica. A
espécie humana, contudo, é uma coisa perene e obstinada, não quer ter
seus passos --- para frente e para trás --- julgados depois de milhares,
de centenas de milhares de anos, ou seja, \emph{não} quer ser
considerada como um todo a partir de um infinitamente pequeno ponto
atômico, um homem singular. O que poucos milênios podem nos ensinar (ou,
para expressar de outra forma, trinta e quatro anos consecutivos de uma
vida de sessenta anos) para podermos falar, no início, de uma
``juventude'', e, no final, ``da velhice da humanidade''! Não se esconde
nessa crença paralisante em uma humanidade já murcha o mal-entendido de
uma concepção teológica cristã, herdada da Idade Média, a ideia
paralisante da aproximação de um fim do mundo, de um juízo final
temivelmente aguardado? Essa ideia, travestida pela crescente
necessidade de julgar da história, como se nossa época, a última das
possíveis, tivesse sido incumbida de assumir aquele juízo final, que a
crença cristã reserva não para o homem mas para ``o filho do homem''? Em
outras épocas, esse evocativo \emph{memento mori}\footnote{Lembra que
  morrerás.} foi sempre um espinho torturante e igualmente o ápice da
ciência e da consciência medieval. Para ser franco, o chamado contrário
de nossa época, \emph{memento} \emph{vivere},\footnote{Lembra que
  viverás.} ainda ressoa de forma bem acanhada, não sai a plenos
pulmões e tem algo de quase desonesto. Pois a humanidade ainda repousa
no \emph{memento mori} e revela isso por meio da necessidade histórica
universal: o saber não pôde, apesar da força de suas asas, levantar voo,
deixando um profundo sentimento de desesperança e assumindo aquela
coloração histórica que, com apatia, escurece toda educação e cultura
superiores. Uma religião que considera, de todas as horas da vida de um
homem, a sua última como a mais importante, que prevê o término completo
da vida na Terra e condena todos a viver no quinto ato da tragédia,
certamente estimula a força mais profunda e nobre, mas é adversária de
toda nova semeadura, de toda busca ousada, de todo desejo de liberdade;
ela se opõe àquele voo rumo ao desconhecido, pois não ama nem nutre
esperanças: só contra a vontade ela deixa que o devir a estimule e, no
momento certo, o desconsidera e o sacrifica, como uma sedução para a
existência, como uma mentira a respeito da vida. Aquilo que os
florentinos faziam, quando eles, impressionados com os sermões de
Savonarola,\footnote{Girolamo Savonarola (1452--98), frade italiano.} 
patrocinaram aquelas célebres fogueiras sacrificais, queimando quadros,
manuscritos, espelhos e máscaras, o cristianismo podia fazer com
qualquer cultura que estimula a perseverança e toma o \emph{memento
vivere} como lema; e quando não é possível fazer isso direta e
serenamente, isto é, com superioridade, ela alcança, em todo caso, seu
objetivo, quando se alia à cultura histórica, na maioria das vezes sem
cumplicidade e, falando a partir de si, rejeita todo devir com
indiferença e dissemina o sentimento de ser serôdio e epígono, em
resumo, de ter nascido vetusto. A consideração amarga e profundamente
séria sobre o desvalor de todo passado, sobre como o mundo está pronto
para ser julgado, afugentou-se na consciência cética de que em todo 
caso é bom conhecer o passado por completo, porque é muito tarde para
fazer algo melhor. Assim, o sentido histórico realiza seu serviço de
forma passiva e retrospectiva, e só por um esquecimento momentâneo, na
intermitência desse sentido, o paciente da febre histórica se torna
ativo, para que, assim que a ação acabe, disseque o ato com a observação
analítica, com o propósito de evitar sua proliferação e enfim tirar sua
pele em nome da ``história''. Nesse sentido, vivemos ainda como na Idade
Média e a história é uma teologia disfarçada: como também a veneração
com que os cientificamente leigos tratam a casta de cientistas é uma
veneração herdada do clero. O que a Igreja antes oferecia, hoje a ciência
oferece, embora tardiamente: mas a Igreja ainda oferecia algo que
produzia seus efeitos, ao contrário do espírito moderno, que, ao lado de
suas boas qualidades, são reconhecidas sua avareza e sua inaptidão para
a nobre virtude da generosidade.

Talvez não agrade a observação, muito menos a dedução do excesso de
história a partir do \emph{memento mori} medieval e da desesperança que
o cristianismo traz no coração a respeito de todo futuro da existência
terrena. Deve-se, ao menos, substituir essa minha explicação hesitante
por uma explicação melhor; pois a origem da cultura histórica --- e seu
radical desacordo com o espírito de uma ``época moderna'', de ``uma
consciência moderna'' dessa origem --, \emph{deve} ser ela mesma outra vez
conhecida historicamente, a história \emph{deve} resolver o próprio
problema da história, o saber \emph{deve} voltar seu ferrão contra si
mesmo --- esse triplo ``\emph{deve}'' é o imperativo do espírito da
``época moderna'', caso nele haja realmente algo de novo, poderoso,
afirmativo da vida e original. Ou deve ser verdade que nós alemães --
deixando os povos românicos fora do jogo --- temos de ser, nas questões
superiores da cultura, sempre tardios, porque só isso \emph{podemos}
ser, como expressa essa instigante frase de Wilhelm
Wackernagel:\footnote{Wilhelm Wackernagel (1806--69), filólogo alemão.
  Citação da obra \emph{Ensaios sobre a história literária alemã}.}

\begin{quote}
Nós, alemães, somos um povo de descendentes, somos sempre, com todo o
nosso saber superior, até mesmo com nossas crenças, herdeiros do mundo
antigo; e também aqueles que têm o mundo antigo como adversário não
deixam de respirar, além do espírito do cristianismo, o espírito imortal
da cultura clássica, e caso alguém conseguisse subtrair esses dois
elementos do sopro de vida que preenche o interior do homem, então não
restaria muita coisa para manter uma vida espiritual. 
\end{quote}

Mesmo com a
vocação de herdeiro da Antiguidade, nos acalmaria se decidíssemos
tomá-la firmemente como séria e grande, e reconhecêssemos nessa firmeza
nosso único e excelso privilégio --- então, apesar disso, seríamos
obrigados a perguntar se o nosso destino eterno deva ser o do
\emph{pupilo da} \emph{Antiguidade em declínio}: quando será permitido
atingir nosso alvo a passos altos e largos? Quando mereceremos o elogio
de termos recriado em nós o espírito da cultura alexandrino-romana --
também em nossa história universal --- de uma forma tão frutífera e
grandiosa, a fim de, como recompensa mais nobre, podermos nos propor
a tarefas intensas, entre elas retornar ao mundo alexandrino e
superá-lo, procurando nossos modelos de olhar intrépido nos gregos
antigos, no mundo originário da grandeza, do natural e do humano? 
\emph{Mas lá encontramos também a realidade de uma cultura
essencialmente aistórica e uma cultura, apesar ou talvez por isso mesmo,
indizivelmente rica e viva}. Não fôssemos nós, alemães, nada além de
herdeiros, não poderíamos, vendo essa cultura como herança a ser
apropriada, ser algo maior e orgulhoso que herdeiros.

Devemos única e exclusivamente dizer que, mesmo o pensamento,
frequentemente embaraçoso, de ser epígono, pensado altivamente, pode
garantir grandes resultados e um esperançoso anseio pelo futuro, tanto
para um indivíduo quanto para um povo; na medida em que nos entendermos
como herdeiros e sucessores de forças clássicas e surpreendentes, vendo
nisso nossa honra, nosso estímulo. Não como pálido e fraco rebento
tardio de gerações fortes, que leva uma vida acanhada de antiquário e
coveiro. Esses rebentos tardios vivem certamente uma existência irônica:
a destruição está nos seus calcanhares, seguindo os passos mancos de
suas vidas; eles tremem diante dela, quando desfrutam do passado, pois
são memórias vivas, e contudo suas recordações, sem herdeiros, são
absurdas. Assim são tomados por uma compreensão turva de que sua vida é
um erro e de que não têm direito a nenhuma vida futura.

Mas pensemos que repentinamente aqueles tardios antiquários tenham a
insolência de se opor àquela modéstia dolorosa e irônica; pensemos como
eles anunciarão, com uma voz estridente: o homem está no auge, pois
agora possui o saber sobre si e se tornou a si mesmo manifestamente --
assim teríamos um espetáculo em que, como em imagem, teria decifrado o
significado enigmático de uma filosofia muito renomada na cultura alemã.
Acredito que não houve, neste século, nenhum abalo ou mudança perigosa
na cultura alemã que não se tenha tornado algo mais perigoso através do
efeito, descomunal e até hoje influente, dessa filosofia, da filosofia
hegeliana. Na verdade, paralisante e mortificante é a crença de ser um
rebento tardio de sua época: mas ela deve parecer terrível e destrutiva
quando um dia essa crença, através de uma guinada atrevida, idolatra
esse rebento como o verdadeiro sentido e fim de todos os acontecimentos
anteriores, quando sua miséria erudita é igualada à consumação da
história universal. Essa forma de consideração acostumou os alemães a
falar sobre o ``processo universal'' e a justificar sua própria época
como o resultado necessário do processo universal; essa forma de
considerar colocou a história, no lugar de outras forças espirituais,
tais como a arte e a religião, como soberana única, na medida em que ela
é ``o conceito realizado em si mesmo'', na medida em que ela é ``a
dialética do espírito dos povos'' e ``o tribunal universal''.\footnote{Referência
  à famosa frase de Hegel, na sua \emph{Filosofia da história}:
  ``\emph{Die Weltgeschichte ist das Weltgericht}'' (A história
  universal/do mundo é o tribunal universal/do mundo'').
  \emph{Weltgerichte} também quer dizer Juízo Final.}

Com escárnio, chamou-se essa história hegeliana da marcha de Deus sobre
a Terra; um Deus, por sua vez, que é criado pela história. Mas esse Deus
se tornou, no interior da cabeça hegeliana, transparente e
compreensível, superando todos os estádios dialéticos possíveis de seu
devir, até sua autorrevelação: de sorte que, para Hegel, o ápice e o fim
último do processo universal coincidem em sua própria existência
berlinense. \label{existenciaberlinense} Aliás, ele deveria dizer que todas as coisas que viriam
depois dele deveriam ser avaliadas como uma coda musical do rondó da
história universal ou, de forma mais apropriada, como supérfluas. Isso
ele não disse: ele plantou nas gerações por ele fermentadas a admiração
pelo ``poder da história'', que envolve praticamente todo instante na
admiração nua do desfecho e conduz à idolatria do factual, para cujo
serviço agora se esmera no mote mitológico e, além disso, bem alemão de
universalmente ``levar em conta os fatos''. Mas isso é para quem
aprendeu a se curvar e a baixar a cabeça diante do ``poder da
história'', quem enfim a balança com seu ``sim'', mecanicamente, como
uma marionete chinesa, a todo poder, seja do governo, da opinião pública
ou da maioria numérica, e que movimenta seus membros no mesmo ritmo em
que um poder qualquer titereia.\footnote{No alemão, ``\emph{am Faden
  ziehen}'' que, como no inglês ``\emph{to pull the strings}'', provém
  da imagem do titereiro; literalmente significa ``puxar os fios'' e
  figurativamente quer dizer controlar, exercer influência. Seguimos
  aqui a sugestão de ambos os tradutores de língua inglesa de fazer
  referência às marionetes chinesas, o que está implícito no texto em
  alemão: ``\emph{der nickt zuletzt chinesenhaft-mechanisch sein `Ja' zu jeder
  Macht}'' (``que enfim balança a cabeça com seu `sim', de forma
  mecânico-chinesa, a todo poder'').} Contivesse aquele resultado uma
necessidade racional em si, fosse aquele evento a vitória da lógica ou
da ``Ideia'' --- então que todos se ajoelhassem logo nos degraus dos
``resultados''! Quê! Não existiria mais nenhuma mitologia dominante?
Quê! As religiões se extinguiriam? Reparem na religião do poder
histórico, deem atenção aos sacerdotes da mitologia das Ideias e seus
joelhos esfolados! Todas as virtudes não estão no séquito dessa nova
crença? Ou não é falta de individualidade quando o homem histórico se
deixa desvanecer até virar um espelho objetivo? Não é magnânimo abdicar
de toda violência no céu e na terra de modo que em cada violência a
violência seja cultuada em si mesma? A justiça não tem sempre a balança
nas mãos, apurando para que lado pende o mais forte e pesado? E que
escola do bom comportamento é tal consideração da história! Tomar tudo
objetivamente, sem ódio nem amor, tudo compreender, suave e
delicadamente: e mesmo quando alguém educado nessa escola se enfurece e
se enerva, satisfaz-se em saber que é artisticamente; é \emph{ira}
{[}ódio{]} e \emph{studium} {[}estudo{]}, mas completamente \emph{sine
ira et studio} {[}sem ódio e sem parcialidade{]}.\footnote{Frase de
  Tácito sobre como realizava seu trabalho de historiador.}

Mas que pensamentos antiquados contra tal complexo de mitologia e
virtude trago eu no coração! Mas eles devem ser expostos, mesmo que só
produza o riso. Também diria: a história sempre insiste, ``era uma
vez''; a moral, ``não devem'' ou ``não deveriam''. Assim a história se
torna um compêndio de uma imoralidade factual. Quão difícil se enganaria
aquele que visse a história ao mesmo tempo como juíza dessa imoralidade
factual! Ofende a moral, por exemplo, que um Rafael devesse morrer aos
trinta e seis anos: uma tal pessoa não deveria morrer. Querem agora o
auxílio da história, como apologistas do factual, então devem dizer: ele
expressou o que estava nele, ele teria, numa vida mais longa, podido
criar sempre o belo como o mesmo belo, não como um novo belo, etc. São
advogados do diabo, quando conseguem fazer do fato o seu ídolo: enquanto
o fato é sempre estúpido e em toda época ele se aproxima mais de um
bezerro do que de um deus. Além disso, como apologistas da história, é a
ignorância que lhes sopra as respostas: pois é não saber o que é uma
\emph{natura naturans} {[}natureza criadora{]} como Rafael que faz que
não ouçam com força que ela se foi e não mais existirá. Sobre Goethe,
alguém nos ensinou ultimamente que ele se esgotou aos oitenta e dois
anos: gostaria de comparar os poucos anos de um Goethe ``esgotado'' com
um vagão inteiro de currículos frescos e ultramodernos, para poder tomar
parte de debates como os mantidos entre Goethe e Eckermann, para, dessa
maneira, precaver-me de todo ensinamento atual dos legionários do
instante. Quão poucos vivos têm, diante desses mortos, o direito de
viver! O fato de que muitos vivem e aqueles poucos não mais vivem não é
nada mais que uma verdade brutal, isto é, uma estupidez insuperável, um
desajeitado ``era uma vez'' em contraposição à moral do ``não deveria
ser assim''. Isso mesmo, em contraposição à moral! Pois que se fale da
virtude que se queira, da justiça, da magnanimidade, da coragem, da
sabedoria e da compaixão do homem --- em todo lugar isso é virtude na
medida em que se indigna contra o poder cego dos fatos, contra a tirania
do real e se submete a leis que não são as leis daquelas flutuações
históricas. Ela sempre nada contra as ondas da história, seja lutando
contra suas paixões como a próxima facticidade burra de sua existência
ou se exigindo a honestidade, enquanto a mentira tece sua brilhante teia
em sua volta. Se a história não fosse nada além que ``o sistema
universal da paixão e do erro'', então o homem teria de nela ler, como
Goethe fazia ler o \emph{Werther}, igualmente como se gritasse: ``seja
um homem e não me siga!'' Felizmente, ela preserva também a memória das
grandes lutas \emph{contra a história}, ou seja, contra o poder cego do
real, e se coloca assim no pelourinho para destacar como autêntica
natureza histórica aquele que se preocupa menos com ``assim são as
coisas'', para, ao contrário, seguir um ``assim devem ser as coisas''.
Não levar sua geração à cova, mas fundar uma nova geração --- isso os
leva insistentemente adiante: e se eles mesmo nascerem como rebentos
tardios --- existe uma maneira de esquecer isso; as gerações futuras só
os conhecerão como primogênitos.

\chapter*{9}\label{capuxedtulo-9}

Talvez seja nossa época esse primogênito? --- De fato, a veemência de
seu sentido histórico é tão grande, sua expressão, tão universal e
ilimitada, que os tempos vindouros elogiarão, no mínimo, sua
primogenitura --- caso venha a haver, culturalmente falando, \emph{um
tempo vindouro}. Mas resta aqui uma séria dúvida. Junto ao orgulho
do homem moderno está sua \emph{ironia} sobre si mesmo, sua
consciência de viver em um estado de espírito historial e noturno,
em seu temor de futuramente não mais poder resguardar suas
esperanças e forças juvenis. Vez por outra, ele recai no
\emph{cinismo}, justificando o percurso da história segundo o cânone
do cinismo, adequando o desenvolvimento universal para o uso do
homem moderno: o que agora existe é exatamente o que devia ocorrer,
o homem não devia tornar-se outra coisa senão o que é hoje; contra
esse ``dever'' ninguém pode rebelar-se. Quem não consegue suportar a
ironia se refugia no bem-estar desse cinismo; além do mais, é
agraciado com a última e bela invenção da década passada, uma frase
perfeita para aquele cinismo: chama-se a forma atual de ser e viver
placidamente, ``a completa renúncia da personalidade no processo do
mundo''.\footnote{Citação da \emph{Filosofia do inconsciente}, de
Eduard von Hartmann. As citações seguintes de Hartmann provêm
desse livro.} A personalidade e o processo do mundo! O processo
do mundo e a personalidade de um pulgão! Se não se tivesse de ouvir
eternamente a hipérbole das hipérboles: a palavra mundo, mundo,
mundo, ele deveria, com honestidade, falar em homem, homem, homem!
Herdeiros dos gregos e romanos? Do cristianismo? Isso tudo não é
nada para esse cínico; herdeiros do processo do mundo, isso sim!
Ápice e alvo do processo do mundo. Sentido e solução de todos os
enigmas do devir, expressos no homem moderno como fruto mais maduro
    da árvore do conhecimento! --- Eu denomino tudo isso de exaltação
    inflada; nesse emblema se reconhecem os primogênitos de todos os
    tempos, mesmo que tenham chegado por último. Nunca a consideração
    histórica voou tão alto, nem mesmo em sonho: agora a história humana
    é apenas a continuação da história animal e vegetal; e até nas
    profundezas do mar o universalista histórico encontra, como muco
    vivo, rastros de si mesmo; o caminho descomunal que o homem já
    percorreu, como um milagre surpreendente, desvia os olhos do milagre
    ainda mais surpreendente, o do homem moderno que é capaz de abarcar
    com os olhos esse caminho. Ele está, altivo e orgulhoso, na pirâmide
    do processo do mundo, ele coloca sobre ela a derradeira pedra de seu
    conhecimento; parece berrar para a natureza que ouve à sua volta:
    ``alcançamos a meta, somos a meta, somos a natureza perfeita''.

    Europeu orgulhoso do século \textsc{xix}, você está perdendo o juízo!
    Seu saber não completa a natureza; ao contrário, mata-a. Meça, ao
    menos uma vez, sua altura como conhecedor com sua baixeza como
    realizador. Certamente galga os raios solares do conhecimento em
    direção ao céu, mas também desce em direção ao caos. Seu jeito de
    andar, isto é, de galgar como conhecedor, é sua fatalidade: a base e
    o solo recuam diante de você, em direção da incerteza; sua vida não
    possui mais sustentação, apenas teias de aranha que se rasgam cada
    vez que seu conhecimento nelas se agarra. --- Mas não falarei mais
    nada sério sobre o assunto, já que é possível dizer algo mais
    jovial.

    Seu dilaceramento e desfibramento colérico de todo fundamento, sua
    dissolução em um devir derretido e fluido, o incansável destecer e
    historiar tudo o que deveio através do homem moderno, o aranhão nos
    nós da teia cósmica --- isso pode ocupar e preocupar o moralista, o
    artista, o devoto, assim como o estadista; devemos nos alegrar por
    ver tudo isso no espelho mágico e reluzente de um \emph{parodista
    filosófico}, em cuja cabeça a época se tornou, para si mesma,
    consciência irônica, mais precisamente ``até a infâmia'' (para falar
    como Goethe). Hegel nos ensinou uma vez, ``se o espírito faz um
    desvio, estaremos lá também, nós, filósofos'': nossa época se
    desviou para a ironia, e vejam só! Lá também estava E. von Hartmann,
    que tinha escrito sua célebre \emph{Filosofia do inconsciente} --- ou
    melhor dizendo --- sua filosofia da ironia inconsciente. É raro ler
    invenção mais divertida e galhofa filosófica maior que a de
    Hartmann; quem com ela não se esclarece sobre o \emph{devir}, e até
    se prepara interiormente para ele, está realmente maduro para ter
    sido alguma coisa. O início e o fim do processo do mundo, dos
    primeiros estádios da consciência até o retorno para o nada,
    incluindo a tarefa de nossa geração quanto ao processo do mundo.
    Isso exposto a partir da fonte engenhosa da inspiração inconsciente
    e aclarado pela luz do apocalipse, tudo feito de forma bem
    enganadora e com uma seriedade circunspecta, como se tratasse de uma
    filosofia séria, e não de uma filosofia de entretenimento --- o
    conjunto da obra coloca seu criador como o primeiro parodista
    filosófico de todos os tempos: sacrifiquemo-nos então no seu altar,
    sacrifiquemos a ele, o inventor de uma panaceia, um cacho de cabelo
    --- para roubar uma expressão de deslumbramento schleiermacheriana.
    Pois que remédio seria mais salutar contra o excesso de cultura
    histórica que a paródia de Hartmann da história universal?

    Quem quisesse expressar, com justa rispidez, o que Hartmann nos
    anuncia do tripé enfumaçado e envolvente da ironia inconsciente,
    diria: ele afirma que nossa época deverá ser assim como ela é apenas
    quando a humanidade se satisfizer seriamente com sua existência, o
    que acreditamos de coração. Aquela terrível fossilização da época e
    aquele bater inquieto de ossos --- como David Strauss, ingenuamente,
    nos esboçava como a mais bela facticidade --- foram justificados por
    Hartmann, não retrospectivamente, \emph{ex causis efficientibus}
    {[}a partir de causas eficientes{]}, mas prospectivamente, \emph{ex
    causa finali} {[}a partir de causas finais{]}; deixe a galhofa do
    Dia do Juízo Final lançar luz sobre nossa época, e lá se achará que
    ela é muito boa, nomeadamente para aquele que quer padecer da mais
    forte dispepsia da vida possível e não pode desejar fortemente o Dia
    do Juízo Final. Embora Hartmann, segundo seu esboço, chame a era em
    que a humanidade se aproxima da ``idade do homem'', isto é, o estado
    mais feliz que há da ``pura mediocridade'', em que a arte é ``a
    farsa a que o negociante berlinense assiste à noite'', em que ``os
    gênios não são mais necessários, porque, como se diz, isso seria
    lançar pérolas aos porcos ou também porque já se avançou daquele
    estádio, no qual gênios eram necessários, para um mais importante'',
    para aquele estádio do desenvolvimento social em que cada
    trabalhador ``em uma jornada de trabalho, que lhe permite suficiente
    folga para sua formação intelectual, leva uma existência
    confortável''. Galhofeiro de todos os galhofeiros, você discorre
    sobre a nostalgia da época atual: sabe igualmente que tipo de
    fantasma estará aguardando no final dessa era da humanidade, como
    resultado daquela formação intelectual para a pura mediocridade --- o
    nojo. Vê-se que ela é péssima, mas vai piorar muito mais, ``é
    visível que o anticristo se alastra'' --- mas ele \emph{deve} estar,
    ele \emph{deve} vir, pois, com o todo, estamos no melhor caminho --
    de nojo diante de todo existente. ``Por isso, siga para o alto no
    processo do mundo como trabalhador no vinhedo do Senhor, pois o
    processo é aquilo que pode conduzir à redenção!''.

    O vinhedo do Senhor! O processo! Para a redenção! Quem não vê e ouve
    aqui a cultura histórica que só conhece a palavra ``devir''? Como
    ela, intencionalmente, se disfarça numa deformidade paródica, como
    ela, através de uma careta grotesca, diz as coisas mais pérfidas!
    Pois o que propriamente exige esse último chamado piadista aos
    trabalhadores em vinhedos? A que trabalho se deve ansiar ativamente?
    Ou para perguntar de outra forma: o que o cultivado em assuntos
    históricos ainda resta a fazer, o fanático moderno do processo, que
    nada e se afoga no fluir do devir, para mais uma vez plantar o nojo,
    aquela saborosa uva do vinhedo? --- Ele nada tem a fazer senão
    continuar a viver o que já viveu, continuar a amar o que já amou,
    continuar a odiar o que já odiou, continuar a ler o jornal que já
    leu; para ele existe apenas um pecado --- viver de forma diferente da
    que viveu. Mas como ele viveu, aquela célebre página nos diz,
    grafada na pedra, com excessiva nitidez, com frases em negrito,
    sobre as quais toda a hodierna fermentação cultural caiu em um
    entusiasmo cego e em uma cólera entusiasmada, porque acreditava ler
    nessa frase sua própria justificação e além disso sua justificação à
    luz apocalíptica. Pois o parodista inconsciente exige de cada
    indivíduo ``a completa renúncia da personalidade no processo do
    mundo, em nome de sua finalidade, em nome da redenção do mundo''
    ou, mais claro e evidente: ``a afirmação da vontade de vida é
    proclamada correta apenas preliminarmente; pois apenas o abandono da
    vida e de suas dores, mas não em uma abdicação pessoal covarde e
    renúncia, é algo a ser feito para o processo do mundo'', ``o ímpeto
    para a negação da vontade individual é tolo e inútil, mais tolo
    ainda do que o suicídio''. ``O leitor que reflete entenderá, sem
    maiores explicações, como desses princípios se construiria uma
    filosofia prática que não contém a cisão, mas a conciliação com a
    vida''.

    O leitor que reflete entenderá: como Hartmann pode ser
    mal-entendido! E como é indizivelmente divertido que o entenda mal!
    Os alemães atuais são muito refinados? Um inglês de coragem sente
    neles a falta de \emph{delicacy of perception} {[}delicadeza da
    percepção{]}, ousa até dizer: ``\emph{in the German mind there
    does seem to be something splay, something blunt-edged, unhandy and
    infelicitous}''\footnote{O espírito alemão parece ser algo oblíquo,
      embotado, inútil e inapropriado.} --- O grande parodista gostaria
    de retrucar isso? Embora nos aproximemos, segundo sua explicação,
    ``daquele estado ideal em que a espécie humana realiza
    conscientemente sua história'', talvez estejamos ainda bastante
    distantes daquele ideal em que a humanidade lê conscientemente o
    livro de Hartmann. Disso resulta que nenhum homem deixará escapar de
    sua boca a expressão ``processo do mundo'' sem que dessa boca saia
    um sorriso; pois ele se lembrará do tempo em que a ``\emph{german
    mind}'', com toda circunspeção, e mesmo com ``a seriedade
    desfigurada da coruja'', como dizia Goethe, escutava, entoava,
    discutia, honrava, disseminava e canonizava o evangelho parodista de
    Hartmann. Mas o mundo segue adiante, aquele estado ideal não pode
    ser sonhado, ele deve ser batalhado e conquistado, e o caminho da
    redenção só pode ser trilhado com serenidade, a redenção daquela
    mal-entendida seriedade de coruja. Virá o tempo que conterá
    sabiamente toda construção do processo universal ou também da
    história da humanidade, um tempo em que não se levará em
    consideração apenas a massa, mas novamente o indivíduo, que formará
    uma espécie de ponte sobre a corrente desordenada do devir. Estes
    não continuarão um processo, mas sim viverão na sincronicidade e na
    atemporalidade, graças à história, que permite esse efeito conjunto;
    eles viverão como uma república de gênios, como uma vez relatou
    Schopenhauer; um gigante chama outro através de intervalos de
    tempo desérticos, impassíveis aos anões pérfidos e barulhentos que
    rastejam sob eles, continuando um diálogo superior. A tarefa da
    história é ser a mediadora entre eles e sempre dar oportunidade à
    criação do grandioso, emprestando-lhe força. Não, o objetivo da
    humanidade não se encontra no fim, mas só nos seus exemplares
    superiores.

    Certamente nosso divertido parodista dirá, com aquela dialética
    admirável, que é tão autêntica quanto seus admiradores são
    admiráveis: ``Não está de acordo com o conceito de desenvolvimento a
    atribuição, ao processo do mundo, de uma duração infinita no
    passado, porque senão aquele desenvolvimento inteligível já teria
    sido percorrido, o que não é o caso'', (oh, que galhofeiro!), ``do
    mesmo modo, não podemos permitir ao processo uma duração infinita no
    futuro; ambos os casos suprimiriam o conceito do desenvolvimento em
    direção a um fim'' (oh, mais uma vez, que galhofeiro!) ``e
    comparariam o processo do mundo ao tonel das Danaides. A vitória
    completa do lógico sobre o ilógico'' (oh, galhofeiro dos
    galhofeiros!), ``deve contudo coincidir com o fim temporal do
    processo do mundo, o Dia do Juízo Final''. Não, espírito claro e
    zombeteiro, enquanto o ilógico reinar como nos dias atuais, enquanto
    se puder falar, com consentimento geral, como você fala, por
    exemplo, do ``processo do mundo'', o Dia do Juízo Final está ainda
    distante; pois ainda há muita alegria na Terra, algumas ilusões
    ainda vicejam, por exemplo, a ilusão dos contemporâneos com relação
    a você; não somos maduros o suficiente para sermos catapultados de
    volta ao seu nada, pois acreditamos que ainda podem acontecer coisas
    mais divertidas, quando se começar a entender você, inconsciente
    incompreensível. Mas se, apesar disso, o nojo vier com força, como
    assim o profetizou a seus leitores, se estiver correto no esboço que
    fez de seu presente e de seu futuro --- e ninguém desprezou a ambos
    como você, com nojo, os desprezou --- então estarei pronto para
    concordar, com a maioria, na forma que propôs, que no próximo
    sábado, pontualmente à meia-noite, seu mundo se aniquilará, e nosso
    decreto poderá rezar: a partir de amanhã cessará o tempo e os
    jornais não serão lançados.\footnote{Não mantivemos o jogo entre
      \emph{Zeit} (tempo) e \emph{Zeitung} (jornal).} Mas talvez não
    faça efeito e decretamos em vão: pois, em todo caso, não nos falta
    tempo para um belo experimento. Peguemos uma balança e coloquemos,
    em um prato, o inconsciente de Hartmann; no outro, o processo do
    mundo de Hartmann. Há homens que acreditam que ambos terão o mesmo
    peso: pois em cada prato repousaria uma palavra igualmente ruim e
    uma piada igualmente boa. Quando se compreende a piada de
    Hartmann, ninguém mais precisa usar a expressão ``processo do
    mundo'', a não ser em piadas. De fato, já é hora de avançar, com um
    exército de comentários sarcásticos, contra o excesso do sentido
    histórico, contra o prazer exagerado pelo processo, às custas do ser
    e da vida, contra o afastamento inconsequente de todas as
    perspectivas; e se pode sempre elogiar o autor da \emph{Filosofia do
    inconsciente} por ser o primeiro a ter conseguido, mordazmente,
    sentir o ridículo da ideia de ``processo do mundo'' e fazer que os
    outros também o sentissem, através da particular seriedade de sua
    apresentação. Que o ``mundo'' esteja aí, que a ``humanidade'' esteja
    aí, é algo com que não nos preocupamos nem por um instante, a não
    ser que queiramos fazer piada disso: pois a arrogância do pequeno
    verme humano é a mais alegre e divertida do palco da Terra; mas para
    que esteja aí, indivíduo, pergunto-lhe, e se não quiser dizer nada
    para ninguém, tente então, pelo menos uma vez, justificar igualmente
    \emph{a posteriori} o sentido de sua existência, ao lhe oferecer
    mesmo um fim, um objetivo, um ``para quê'', um ``para quê'' elevado
    e nobre. Pereça por ele --- não conheço melhor finalidade da vida do
    que perecer na busca do grande e impossível, \emph{animae magna
    prodigus}.\footnote{``Pródigo de grande alma''\textbf{.} Expressão de
      Horácio, \emph{Odes \textsc{i}}, significando ``sem cuidado com a vida''.} Se, ao
    contrário, as doutrinas do devir soberano, da fluidez de todos os
    conceitos, tipos e espécies, da ausência de toda diferença cardinal
    entre homem e animal, doutrinas que tomo por verdadeiras mas letais
    --- na sanha de ensinamentos hoje habituais, for lançada ao povo por
    mais uma geração, não é de admirar quando um povo decline por
    pequenez egoísta e miséria, por fossilização e egoísmo, primeiro
    desmoronando e deixando de ser povo: em seu lugar poderão surgir
    talvez, na arena do futuro, sistemas de egoísmos individuais,
    fraternizações para fins de pilhagem dos não irmãos e criações
    semelhantes de vulgaridade utilitária. Que se vá adiante para
    preparar essa criação de escrever a história do ponto de vista das
    \emph{massas} e procurar aquelas leis que se deduzem das
    necessidades dessas massas, ou seja, segundo as leis do movimento
    dos estratos mais baixos de barro e argila da sociedade. As massas
    parecem-me merecer um olhar em apenas três aspectos: primeiro como
    cópias borradas dos grandes homens, produzidas em papel de má
    qualidade e com matrizes já gastas, depois como resistência contra
    os grandiosos e finalmente como instrumento dos grandiosos; no mais,
    ao diabo com elas e com a estatística! Como se a estatística
    provasse que haveria leis na história? Leis? Sim, ela prova quão
    vulgar e nauseantemente uniforme é a massa: devem-se chamar de leis
    os efeitos das forças gravitacionais da estupidez, do macaquear, do
    amor e da fome? Queremos confessar, mas com isso que se fixe a
    frase: quanto mais houver leis na história, essas leis não terão
    valor e a história não terá valor. Mas justamente essa espécie de
    história é hoje estimada, essa que toma os impulsos da massa como a
    coisa mais importante e principal na história e considera todos os
    grandes homens apenas a expressão mais clara e, igualmente, como
    borbulhas visíveis na enchente. Daí a massa deve dar à luz, de si, a
    grandeza, do caos, a ordem; no fim, será entoado então o hino à
    massa genitora. Grandioso será chamado então tudo que moveu essa
    massa e, como se diz, possuiu ``um poder histórico''. Mas isso não
    quer dizer confundir intencionalmente quantidade com qualidade? Se a
    espessa massa encontrou de forma justa e adequada um pensamento
    qualquer, por exemplo, um pensamento religioso, defendeu-o duramente
    e o empurrou por anos: então só assim o descobridor e fundador desse
    pensamento será grandioso. Qual o quê! O mais nobre e o mais elevado
    não têm efeito sobre a massa; o sucesso histórico do cristianismo,
    seu poder histórico, sua dureza e sua estabilidade, felizmente, nada
    provam a respeito da grandeza de seu fundador, já que provariam algo
    contra ele: mas entre ele e o sucesso histórico repousa uma camada
    bem terrena e obscura de paixão, erro, ânsia por poder e honra, de
    uma força atuante do \emph{imperium romanum}
    %Traduzir imperium romanum?
    , uma camada da qual o
    cristianismo recebeu seu gosto telúrico e seu resto telúrico, que
    possibilitou sua continuação no mundo e igualmente fornecia sua
    conservação. O grandioso não deve depender do sucesso, e Demóstenes
    teve grandeza, embora não tivesse tido sucesso. Os seguidores mais
    puros e verdadeiros do cristianismo sempre colocaram em questão o
    seu sucesso mundano, o seu assim chamado ``poder histórico'',
    obstruindo-o mais que o fomentando; pois costumam se colocar fora
    ``do mundo'', sem se preocupar com ``o processo da ideia cristã'';
    por isso, em sua maioria permaneceram totalmente ignorados e
    anônimos na história. Expresso de forma cristã: o demônio é o
    regente do mundo e o mestre do sucesso e do progresso; em todo poder
    histórico ele é o próprio poder, e assim ele permanecerá
    essencialmente; isso pode soar bem lamentável aos ouvidos de uma
    época que se acostumou com a divinização do sucesso e do poder
    histórico. Ela se exercitou justamente em renomear as coisas e mesmo
    rebatizar o demônio. É certamente a hora de um grande perigo: os
    homens parecem perto de descobrir que o egoísmo do indivíduo, do
    grupo ou das massas foi, em todos os tempos, a alavanca do movimento
    histórico; mas, ao mesmo tempo, ninguém se incomoda com essa
    descoberta, mas decreta: o egoísmo deve ser nosso deus. Com essa
    nova crença se está prestes a construir, com clara intenção, a
    história futura no egoísmo; deve apenas ser um egoísmo prudente, que
    se obriga a algumas restrições a fim de manter-se de forma
    duradoura, que justamente por isso estuda a história, para tomar
    conhecimento do egoísmo imprudente. Nesse estudo se aprende que cabe
    ao Estado uma missão bem particular no sistema universal fundador do
    egoísmo: ele deve se tornar o patrono de todo egoísmo prudente, para
    se proteger, com força militar e policial, contra o terrível
    desencadeamento do egoísmo imprudente. Para os mesmos fins, a
    história --- tanto animal quanto humana --- vem sendo inculcada nas
    perigosas, porque imprudentes, massas populares e proletárias,
    porque se sabe que um grão de cultura histórica é capaz de
    desencadear os instintos e desejos mais toscos e brutos ou conduzir
    à trilha do egoísmo refinado. Em suma: falando com E. von Hartmann,
    o homem ``olha para o futuro em uma prática e confortável acomodação
    na pátria terrena''. O mesmo escritor denomina tal período de ``era
    viril da humanidade'', zombando assim daquilo que hoje se chama
    ``homem'', como se com isso se entendesse o sóbrio egocêntrico; como
    ele igualmente profetiza, após essa era da humanidade, uma era senil
    a ela correspondente, deixando contudo com isso visível a sua
    zombaria para com os idosos de hoje: pois ele fala de sua serenidade
    madura, como eles ``veem toda a paixão de sua vida pregressa
    irromper desordenadamente e compreendem a vaidade dos supostos fins
    de então''. Não, corresponde a uma era humana daquele egoísmo
    cultivado historicamente e batido, uma era senil, de uma ambição de
    vida repugnante e indigna e, assim, o último ato, que 

 \begin{quote}
 \begin{verse}
encerra a estranha história agitada,\\
em uma segunda infância e total esquecimento\\
sem olhos, sem dentes, sem paladar, sem nada.\footnote{Shakespeare,
  \emph{Como gostais}, Ato 2, Cena 7.}
 \end{verse}
 \end{quote}
  

  Quer os perigos de nossa vida e nossa cultura provenham desses velhos
  devastados, desdentados e sem paladar ou dos daqueles chamados
  ``homens'' de Hartmann: diante de ambos queremos cravar com os dentes
  o direito de nossa \emph{juventude} e não cansaremos de defender o
  futuro de nossa juventude contra aqueles iconoclastas da imagem do
  futuro. Mas nessa luta devemos ter uma percepção particularmente ruim:
  \emph{que, intencionalmente, se fomenta, anima --- e utiliza --- o
  excesso do sentido histórico que o momento sofre}.

  Mas se o utiliza contra a juventude, para esta se adestrar àquela
  maturidade de um egoísmo almejado em todos os lugares, utiliza-o
  para romper a indisposição natural da juventude ao egoísmo
  viril-inviril por meio de uma luz transfiguradora, quer dizer,
  mágico-científica. Aliás, do que o sobrepeso de história é capaz se
  sabe muito bem: desenraizar os instintos mais fortes da juventude,
  como fogo, repúdio, autoesquecimento e amor, abrandar o calor do seu
  sentimento de justiça, murchar lentamente seus desejos com os desejos
  contrários de estar rapidamente pronta, útil, fértil, pressioná-la
  para baixo ou para trás, adoecer a honestidade e polidez das sensações
  por meio da dúvida; ele é mesmo capaz de enganar a juventude em
  relação ao seu próprio belo privilégio de poder plantar, com inteira
  fé, um grande pensamento, e permitir que dele saiam outros ainda
  maiores. Um certo excesso de história é capaz de tudo, já vimos: ainda
  mais que não mais se permite ao homem que ele sinta e aja
  \emph{aistoricamente}, por meio de um crescente afastamento de
  perspectivas e horizontes e do afastamento da atmosfera envolvente.
  Ele então puxa a infinitude do horizonte para si mesmo, de volta ao
  círculo estreito do egoísmo e deve por isso apodrecer e ressecar-se:
  talvez ele alcance a prudência; nunca a sabedoria. Ele se deixa falar,
  calcular e se pacificar com os fatos, não se destempera, pisca e sabe
  procurar a própria vantagem e partido na vantagem e desvantagem
  alheia; ele desaprende a vergonha supérflua e se torna
  progressivamente o ``homem'' e ``idoso'' hartmanniano. Mas é isso que
  ele \emph{deve} se tornar, é justamente isso o sentido da agora tão
  cinicamente exigida ``completa renúncia da personalidade no processo
  do mundo'' --- em nome de seu objetivo, da redenção do mundo, como nos
  assegura E. von Hartmann, o galhofeiro. Agora, é difícil que a vontade
  e o objetivo daquele ``homem'' e ``idoso'' hartmanniano sejam
  justamente a redenção do mundo: certamente o mundo estaria redimido se
  se redimisse desses homens e idosos. Aí então adviria o reino da
  juventude!

\chapter*{10}\label{capuxedtulo-10}

Neste momento, pensando na \emph{juventude}, eu brado: terra à
    vista! Terra à vista! Satisfeito e mais que satisfeito desta viagem
    emotiva, exploradora e errante em direção a mares estranhos e
    sombrios! Finalmente, agora surge uma costa: para saber como ela é,
    preciso nela aportar, e o pior porto de emergência é melhor do que
    voltar a balançar na infinitude desesperançada e cética. Paremos
    primeiro em terra; depois encontraremos os bons portos e
    facilitaremos a chegada dos que virão.

    Essa viagem foi perigosa e emocionante. Quão distantes estamos do
    olhar tranquilo com que de início víamos nosso navio navegar.
    Presumindo os perigos da história, encontramo-nos fortemente unidos;
    nós mesmos trazemos à luz os rastros daqueles sofrimentos que, em
    consequência do excesso de história, acometem o homem da nova era, e
    justamente este tratado mostra como não procuro enganar-me sobre o
    seu caráter moderno, o caráter da personalidade fraca, que está na
    desmedida de sua crítica, na imaturidade de sua humanidade, na
    passagem constante entre orgulho e ceticismo. E contudo confio na
    força inspiradora que, ao contrário do gênio, desvia o navio; confio
    na \emph{juventude} que me conduziu, quando agora necessito de um
    \emph{protesto contra a educação histórica da juventude do homem
    moderno} e quando o protestador exige que o homem necessite da
    história sobretudo para aprender a viver; apenas \emph{a serviço da
    vida é que se aprende}. É preciso ser jovem para entender este
    protesto; aliás se pode, dada a vetustez da juventude atual, quase
    não ser jovem o suficiente para perceber contra quem se protesta
    aqui. Como auxílio, tomarei um exemplo. Na Alemanha, há menos que um
    século, surgiu em alguns jovens o instinto natural para aquilo que se
    chama poesia. Pensam que as gerações anteriores, naquela época, nada
    falaram daquela arte, para eles internamente estranha e inatural?
    Sabe-se do contrário: que elas refletiram, escreveram, discutiram
    intensamente sobre a ``poesia'', mas com palavras, palavras e
    palavras. Aquele despertar de uma palavra para vida não foi
    igualmente a morte dos seus criadores, em certo sentido eles ainda
    vivem; pois quando, como disse Gibbon,\footnote{Edward Gibbon
      (1737--94), historiador inglês.} é preciso muito tempo para que um
    mundo decline, do mesmo modo é preciso muito tempo para que, na
    Alemanha, ``a terra do processo paulatino'', um falso conceito
    decline. Pelo menos talvez haja centenas de homens, distantes há
    mais de um século, que sabiam o que é a poesia; talvez, depois de
    centenas de anos, haverá novamente mais centenas de homens que
    entrementes tenham aprendido o que seja a cultura, algo que os
    alemães até agora não possuem, mesmo que falem ou se orgulhem disso.
    O contentamento geral dos alemães com sua cultura lhes parecerá tão
    incrível e medíocre quanto, para nós, foi o reconhecimento de
    Gottsched\footnote{Johann Christoph Gottsched (1700--66), filósofo
      alemão e crítico literário.} como um clássico e Ramler\footnote{Karl
      Wilhelm Ramler (1725--98), poeta e tradutor alemão.} como o Píndaro
    alemão. Talvez eles julgarão que essa cultura seja apenas uma
    espécie de saber acerca da cultura e por isso seja certamente um
    saber falso e superficial. Falso e superficial porque carrega a
    contradição entre vida e saber, porque não vê o característico em
    todo povo de cultura verdadeira: que a cultura só pode medrar e
    florescer a partir da vida; enquanto ela é, entre os alemães,
    plantada como uma flor de papel ou regada com açúcar, e por isso
    deve sempre permanecer como mendaz e estéril. A educação da
    juventude alemã, contudo, parte desse conceito falso e estéril de
    cultura: seu objetivo, pensado de forma pura e elevada, não é o
    homem culto e livre, mas o erudito, o homem científico e ainda o
    mais precoce homem científico possível, que se afasta da vida para
    conhecer de forma certa e precisa; seu resultado, vendo de forma
    justa, empírica e comum, é o filisteu da cultura histórico-estética,
    que tagarela, esperto e novidadeiro, sobre o Estado, a Igreja e
    arte, o sensor de milhares de sensações, o estômago insaciável, mas
    que não sabe o que é uma fome e uma sede justas. Que são inaturais
    os objetivos de tal educação e seu resultado, isso sente aquele que
    ainda não foi feito para ele, que só sente o instinto da juventude,
    porque ainda possui o instinto da natureza, que se quebrou, de forma
    artificial e violenta, através daquela educação. Quem contudo quer
    romper com essa educação, que quer ajudar a juventude a se
    pronunciar, esse deve iluminar, com a claridade dos conceitos, a
    recusa que lhes é inconsciente, tornando-a consciente a uma
    consciência que fala alto. Como pode ele alcançar esse inusitado
    objetivo?

    Sobretudo destruindo uma superstição, a crença na \emph{necessidade}
    daquela forma de educação. Porém, pensa-se que não haveria outra
    possibilidade senão a da nossa tão lamentável realidade. Basta
    alguém examinar a literatura das últimas décadas produzida por
    nossas escolas e estabelecimentos de ensino superiores: ele
    verificará, para seu espanto e desgosto, como o objetivo geral do
    ensino é pensado uniformemente, em toda mudança de sugestões, em
    toda sofreguidão de contradições; como temerosamente se admite o
    resultado atual, o ``homem culto'', como hoje é entendido, como o
    fundamento necessário e racional de um ensino ulterior. Mas aquele
    cânone monótono soaria assim: o jovem deve começar com um saber
    acerca da cultura, não com um saber acerca da vida e muito menos com
    um saber acerca da própria vida e vivência. Ainda mais, esse saber
    acerca da cultura, como saber histórico, é misturado e administrado
    ao jovem; isto é, sua cabeça é entupida com um número descomunal de
    conceitos extraídos, no máximo, do conhecimento indireto de épocas e
    povos pretéritos, não da observação direta da vida. Seu anseio é
    entorpecido e igualmente inebriado pelo grande teatro de que seria
    possível sumarizar em si as mais altas e mais marcantes experiências
    das épocas antigas, justamente as maiores épocas. É o mesmo método
    absurdo que conduz nossos jovens artistas plásticos a museus e
    galerias, e não ao ateliê de um mestre e, sobretudo, ao ateliê da
    mestra única, a natureza. Como se se pudesse prever, como um
    passeante fugidio, na história das coisas passadas, seus pendores e
    artes, seu produto vital! Como se a própria vida não fosse um
    ofício, que se aprende profunda e firmemente, e que se exerce com
    labor, quando não impede que incompetentes e falastrões saiam do
    ovo!

    Platão considerava necessário que a primeira geração de sua nova
    sociedade (no Estado perfeito) fosse educada com a ajuda de uma
    forte \emph{mentira necessária}; as crianças deviam aprender a
    acreditar que tinham habitado por muito tempo, sonhando, sob a
    terra, onde foram prensadas e conformadas pelo artesão-mestre da
    natureza. Impossível se rebelar contra esse passado! Impossível se
    contrapor à obra dos deuses. Isso deve valer como uma cláusula
    pétrea da natureza: quem nasceu como filósofo tem ouro em seu corpo;
    como guardião, apenas prata; como artesão, ferro e bronze. Como não
    é possível misturar esses metais, esclarece Platão, não deve ser
    possível alterar e trocar a ordem das castas; a crença nessa
    \emph{aeterna veritas} {[}verdade eterna{]} é o fundamento da nova
    educação e portanto do novo Estado. --- Assim também o alemão moderno
    acredita na \emph{aeterna veritas} de sua educação: e contudo essa
    crença ruirá, como ruiria o Estado platônico, quando a essa mentira
    necessária se opuser uma \emph{verdade necessária}:\footnote{Jogo de
      palavras entre \emph{Notlüge} (uma mentira que procura acalentar
      ou prevenir do pior, em uma situação de emergência) com o que
      seria seu antônimo, o neologismo \emph{Notwahrheit}.} que o
    alemão não possui cultura alguma, porque ele, graças a sua educação,
    não pode ter cultura alguma. Ele quer a flor sem caule e raiz:
    portanto a quer em vão. Essa é a simples verdade, uma justa verdade
    necessária, desagradável e áspera.

    Mas é com essa verdade necessária que \emph{nossa primeira geração}
    deve ser educada; é certo que ela sofrerá mais, pois deverá
    ensiná-la a si mesma e até contra si mesmo, para chegar a um novo
    hábito e uma nova natureza a partir de uma natureza e de um hábito
    anteriores e envelhecidos: de tal modo que ela possa falar entre si,
    em espanhol antigo, ``\emph{Defienda me Dios de my}'', ``Deus,
    defendei-me de mim'', ou seja, de minha natureza já educada. Ela
    pode provar daquela verdade gota a gota, como um remédio amargo e
    forte, e cada indivíduo dessa geração deve se superar, julgar a si
    mesmo, o que ele suportaria ainda mais facilmente do que um juízo
    geral sobre toda sua época: somos sem cultura, mais ainda, fomos
    destituídos da vida, do simples e correto ver e ouvir, da apreensão
    feliz do que é próximo e natural, e não possuímos até hoje o
    fundamento de uma cultura, porque não estamos, nós mesmos,
    convencidos de ter em nós uma vida verdadeira. Despedaçado e
    destroçado, dividido meio mecanicamente, em seu conjunto, em um
    interior e um exterior, semeando conceitos como quem semeia dentes
    de dragão, criando dragões conceituais, sofrendo de uma doença da
    palavra e sem confiança na própria sensação, quando não é carimbada
    com uma palavra: como uma morta e contudo assustadoramente
    perturbadora fábrica de conceitos e palavras tenho talvez o direito
    de dizer para mim mesmo \emph{cogito ergo sum}, mas não \emph{vivo
    ergo sum}. O ``ser'' vazio me é afiançado, não a inteira e verde
    ``vida''; minha sensação originária garante apenas que eu sou um ser
    pensante, não um vivente, que não sou um animal, mas um
    ``\emph{cogital}''. Deem-me apenas vida que eu criarei, a partir
    dela, uma cultura! --- Assim clama o indivíduo dessa primeira
    geração, e todos os indivíduos entre eles se reconhecerão nesse
    clamor. Quem lhes dará essa vida?

    Nenhum deus e nenhum homem, apenas sua própria \emph{juventude}:
    esta, estando livre, terá libertado a vida. Pois ela repousava
    escondida, na prisão, não está estragada e morta --- perguntem a si
    mesmos!

    Mas essa vida libertada está doente e tem de ser curada. Adoece de
    tanta miséria e não sofre apenas por lembrar o sofrimento em suas
    amarras --- ela sofre, no que nos diz respeito principalmente aqui,
    da \emph{doença histórica}. O excesso de história agrediu a força
    plástica da vida, ela não sabe mais se servir do passado como um
    alimento poderoso. A miséria é terrível, mas em vão! Se a juventude
    não tivesse o dom previdente, ninguém saberia que ela está na
    miséria e que o paraíso da saúde foi perdido. Essa mesma juventude
    adivinha contudo, com um instinto curador da mesma natureza, como
    recobrar esse paraíso; ela conhece o bálsamo e o medicamento contra
    a doença histórica, contra o excesso do histórico: como eles se
    chamam?

    Não se admire que sejam nomes de venenos: o antídoto contra o
    histórico se chama --- \emph{o aistórico e o supra-histórico}. Com
    esses nomes retornamos ao início de nossa consideração e à sua
    bonança.

    Com a palavra ``aistórico'' designo a arte e a força de poder
    \emph{esquecer} e se fechar em um horizonte delimitado; chamo de
    ``supra-histórico'' o poder de desviar a visão do devir em direção
    daquilo que dá à existência o caráter da eternidade e identidade, a
    \emph{arte} e a \emph{religião}. A \emph{ciência} --- pois é ela que
    falaria de venenos --- veria naquela força, nesses poderes, poderes e
    forças adversários; pois ela considera verdadeiro e correto apenas a
    reflexão objetiva, portanto a reflexão científica, que vê em todo
    lugar algo que veio a ser, algo histórico, e em nenhum lugar um ser,
    algo eterno; ela vive do mesmo modo em uma contradição eterna contra
    os poderes perpetuadores da arte e da religião, como se ela odiasse
    o esquecimento, a morte do conhecimento, como se ela procurasse
    suprimir a limitação de horizontes para inserir o homem em um mar
    infinito e ilimitado de ondas luminosas do conhecimento do devir.

    Se ele pudesse aí viver! Assim como as cidades desmoronam e ficam
    desertas em um terremoto e o homem só sai de sua casa, no solo
    vulcânico, tremendo e fugidio, a vida colapsa, tornando-se fraca e
    temerosa, quando o \emph{terremoto de conceitos} que a ciência
    provoca toma do homem a crença no fundamento de toda segurança e
    tranquilidade, a crença no permanente e no eterno. Deve a vida
    imperar sobre o conhecimento, sobre a ciência, deve o conhecimento
    imperar sobre a vida? Qual das forças é a superior e decisiva?
    Ninguém duvidará: a vida é superior, a força imperante, pois um
    conhecimento que destruísse a vida seria destruído por si mesmo. O
    conhecimento pressupõe a vida, tendo portanto o mesmo interesse na
    vida que qualquer criatura tem na sua sobrevivência. A ciência
    necessita, assim, de uma observação superior e vigilância; \emph{uma
    higiene da vida} acerca-se da ciência; e a proposição dessa higiene
    seria: o aistórico e o supra-histórico são os antídotos naturais
    contra a vigilância da vida pelo histórico, contra a doença
    histórica. É provável que nós, os doentes históricos, também
    tenhamos de sofrer com esse antídoto. Mas esse sofrimento não é uma
    prova contra a adequação do tratamento.

    E aqui reconheço a missão daquela \emph{juventude}, daquela primeira
    geração de guerreiros e caçadores de serpentes, que antecipa uma
    cultura e humanidade felizes e belas, sem ter das alegrias futuras e
    da primeira beleza não mais que uma ideia positiva. Essa juventude
    sofrerá, ao mesmo tempo, do mal e do antídoto; apesar disso acredita
    poder se celebrizar por uma saúde forte e uma natureza mais natural,
    mais que sua geração anterior, os ``homens'' cultos e ``vetustos''
    da atualidade. Sua missão, contudo, é abalar os conceitos de
    ``saúde'' e ``cultura'' da atualidade e fomentar escárnio e ódio
    contra esses híbridos monstros de conceitos; o indício e maior
    garantia de sua forte saúde deve ser justamente o de que essa
    juventude não precisa utilizar nenhum conceito, nenhuma palavra de
    ordem tirada da inundação de moedas verbais e conceituais da
    atualidade para nomear sua essência; é convencida por um poder
    atuante, batalhador, discriminador e distintivo presente nela e por
    um sentimento vital crescente em cada momento. Pode-se discutir se
    essa juventude já possui cultura --- mas isso é uma objeção para que
    juventude? Pode-se apontar sua aspereza e desmedida --- mas ela ainda
    não é idosa e sábia o bastante para se satisfazer com isso;
    sobretudo, ela não precisa de nenhuma cultura pronta para fingir e
    defender, e aproveita todas as consolações e prerrogativas da
    juventude, sobretudo a prerrogativa da honestidade corajosa e
    impulsiva e a consolação entusiástica da esperança.

    Sei que esses esperançosos conhecem todas essas generalidades de
    perto e que se traduzirão, com sua experiência mais própria, em uma
    doutrina pensada pessoalmente; os outros, às vezes, não poderão
    perceber senão tigelas cobertas, que bem podem estar vazias; até que
    eles, surpresos com os próprios olhos, vejam que as tigelas estão
    cheias, e que agressões, exigências, impulsos vitais, paixões estão
    empacotados e espremidos nessas generalidades, que não poderiam
    ficar por muito tempo cobertos. Volto-me, por fim, a esses céticos da
    época que mostra tudo à luz, em direção daquela sociedade de
    esperançosos, para lhes contar, por meio de símbolos, o passo e o
    percurso de sua cura, sua salvação da doença histórica, e com isso
    sua própria história, até o momento em que serão saudáveis novamente
    para realizar história e se dedicar ao passado sob o domínio da
    vida, naqueles três sentidos, ou seja, monumental, antiquário ou
    crítico. Nesse momento, serão menos conhecedores do que os
    ``cultos'' da atualidade; pois muito terão desaprendido e mesmo
    perdido a vontade de vislumbrar aquilo que esses cultos querem saber
    antes de tudo; suas características são, do ponto de vista daqueles
    cultos, justamente ``incultura'': sua indiferença e
    introspectividade contra muita coisa célebre, até mesmo boa. Mas
    eles, no fim de sua cura, tornaram-se novamente \emph{humanos} e
    deixaram de ser agregados antropomorfos --- isto é alguma coisa!
    Ainda são esperanças! Não sorriem com satisfação, esperançosos?

    Perguntarão: como chegaremos àqueles fins? O deus délfico chama-os,
    desde o início de sua caminhada para aqueles fins, com sua sentença:
    ``Conhece-te a ti mesmo''. É uma sentença difícil: pois aquele deus
    ``não diz e nem oculta, mas dá sinais'',\footnote{Heráclito,
      ``Fragmento 93''. Trad. José Cavalcanti de Souza.
      \emph{Pré-Socráticos} (Coleção ``Os Pensadores''). São Paulo:
      Abril Cultural, 1991.} como disse Heráclito. Para onde ele os
    orienta?

    Houve séculos em que os gregos se encontravam no mesmo perigo em que
    nos encontramos, ou seja, de sucumbir na inundação do estrangeiro e
    do passado, na ``história''. Eles nunca viveram numa intangibilidade
    orgulhosa: sua ``cultura'', ao contrário, foi sempre, por muito
    tempo, um caos de formas e conceitos estrangeiros, semíticos,
    babilônicos, lídios, egípcios, e sua religião, uma verdadeira
    batalha de deuses de todo o Oriente: semelhante a como hoje a
    ``cultura alemã'' e a religião são em si um caos de toda uma terra
    estrangeira, de toda uma época anterior. E, apesar disso, a cultura
    helênica não era um aglomerado, graças àquela sentença apolínea. Os
    gregos aprenderam aos poucos a \emph{organizar o caos} ao se
    voltarem, segundo o ensinamento délfico, a refletir sobre si, isto
    é, sobre suas necessidades autênticas, e deixar perecer as
    necessidades ilusórias.\label{necessidadesilusorias} Assim eles tornaram a se apoderar de si; não
    permaneciam mais herdeiros e epígonos saturados de todo o Oriente;
    eles se tornaram, pela luta laboriosa consigo mesmos, através da
    prática da interpretação daquela sentença, os continuadores e
    multiplicadores do tesouro herdado e modelos e primogênitos de toda
    cultura futura do povo.

    Essa é uma alegoria para cada um de nós: deve-se organizar o caos
    que se tem em si, tornando a refletir sobre suas maiores
    necessidades. Sua honestidade, seu caráter corajoso e veraz, deve em
    algum momento se posicionar contra o fato de se tornar sempre algo
    repetido, reaprendido, imitado; começa-se então a compreender que a
    cultura pode ser algo diferente de uma \emph{decoração da vida},
    isto é, de algo, no fundo, sempre fingimento e dissimulação; pois
    todo adereço esconde o que adereça. Assim se desvela o conceito
    grego de cultura --- em oposição ao romano ---, o conceito de cultura
    como uma \emph{physis} {[}natureza{]} nova e aprimorada, sem
    interior e exterior, sem fingimento e convenção, a cultura como uma
    consonância entre vida, pensamento, aparência e querer. Assim ele
    aprende, por experiência própria, que isso era a força superior da
    natureza \emph{moral} que propiciou aos gregos a vitória sobre as
    outras culturas, e que aquele aumento de veracidade também deve ser
    uma exigência preparatória de uma \emph{verdadeira} cultura: se essa
    veracidade puder prejudicar seriamente, em uma oportunidade, a
    cultura em voga, ela poderá ajudar a levar à queda toda uma cultura
    decorativa.
