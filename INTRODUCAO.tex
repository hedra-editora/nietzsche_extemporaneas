\chapter{Introdução}{}{André Itaparica}\label{introduuxe7uxe3o}

\hyphenation{Nietz-sche}

Publicada em 1874, a segunda das quatro considerações extemporâneas,
\emph{Sobre a utilidade e a desvantagem da história para a vida}, foi a
que teve a menor repercussão quando de seu lançamento. Não foi recebida
com muito entusiasmo pelos círculos wagnerianos (causa pela qual
Nietzsche militava), foi objeto de críticas severas de seu amigo e
leitor das primeiras provas de impressão, Erwin Rohde, e teve como
resposta do grande historiador Jacob Burckhardt, seu colega na
Universidade de Basileia, uma carta em grande medida formal e distante
(embora Nietzsche não a tenha percebido exatamente dessa maneira). Sem
entrar no mérito dessas avaliações, o fato é que Nietzsche se abateu com
essa recepção imediata, e como resultado a obra passou a ser um livro
pouco comentado pelo próprio Nietzsche.\footnote{Sobre a história da
  curta elaboração da segunda extemporânea, da recepção inicial e de seu
  efeito sobre o ânimo de Nietzsche, há um importante ensaio de Jörg
  Salaquarda: ``Studien zur zweiten~Unzeitgemäßen Betrachtung''. In:
  \emph{Nietzsche-Studien}, 13, 1984, pp. 1--45.} Em sua autobiografia,
\emph{Ecce homo}, ao contrário das demais considerações extemporâneas,
que são exaltadas, a segunda ganha elogiosas, mas parcas linhas: ``A
\emph{segunda} extemporânea traz à luz o que há de perigoso, corrosivo e
envenenador da vida na nossa forma de prática científica: a vida
\emph{doente} dessa engrenagem e mecanismo desumanos, da
`\emph{im}pessoalidade' do trabalhador, da falsa economia da `divisão
do trabalho'. Perde-se o fim, a cultura: -- o meio, a moderna prática
científica, \emph{barbariza}\ldots{} Nesse tratado o `sentido histórico', do
qual nosso século se orgulha, foi pela primeira vez reconhecido como
doença, como signo típico da decadência''.\footnote{Nietzsche, F.
  \emph{Ecce homo.} Berlim/Munique: Walter de Gruyter, 1988, p. 316.}
Não deixa de ser irônico que posteriormente esse escrito sobre a
história passaria a ter um reconhecimento por parte da fortuna crítica
só comparável, entre as primeiras obras de Nietzsche, ao
\emph{Nascimento da tragédia}. Nos livros clássicos de Karl
Jaspers,\footnote{Jaspers, Karl. \emph{Nietzsche:}~Einführung~in das 
Verständnis seines Philosophierens. Berlim: Walter de Gruyter, 1936.} 
e Walter Kaufmann\footnote{Kaufmann, Walter.
  \emph{Nietzsche:}~Philosopher,~Psychologist,
  Antichrist. Princeton: Princeton University Press, 1950.} a
segunda extemporânea tem um lugar de destaque na discussão sobre a
filosofia da história de Nietzsche, e Martin Heidegger chegou a conduzir
um seminário exclusivo sobre ela, em Freiburg, no semestre de inverno de
1938--39. Hoje poucos discordariam de que é muito difícil discutir a
filosofia da história e a filosofia da cultura de Nietzsche sem recorrer
a esse livro complexo, ambíguo e provocativo.

Esse caráter de texto de intervenção das extemporâneas é um traço
patente também no escrito sobre a história, o que se revela desde as
primeiras linhas do prefácio, com a defesa goethiana do saber como
estímulo à ação. O próprio título da série de livros já atesta essa
preocupação: \emph{Unzeitgemässe Betrachtungen} pode ser parafraseado
como ``considerações, reflexões ou meditações que trazem a marca da
inconformidade e intempestividade para com a época presente''. E se as
primeiras linhas do prefácio trazem fortes palavras de Goethe, ele se
encerra, por sua vez, com as não menos fortes palavras de Nietzsche,
apresentando uma definição lapidar do que ele entende por
extemporaneidade: ``Para tanto, permito-me confessar, até pela minha
profissão de filólogo clássico: não saberia que sentido teria a
filologia clássica em nossos dias senão o de intervir extemporaneamente
-- isto é, contra a época, sobre a época e a favor de uma época
futura'' (p.\,\pageref{epocafutura}).

Se sua natureza aguerrida e sua prosa ao mesmo tempo poética e aguda são
fáceis de perceber, o mesmo não se pode dizer do tema e das teses do
livro, assunto sobre o qual seus leitores dificilmente chegam a um
acordo. Apesar de em seu título constar a palavra de origem latina
\emph{Historie} -- e não o termo de origem germânica \emph{Geschichte}
--, o que conotaria um estudo exclusivo sobre uma disciplina acadêmica,
a historiografia, sua discussão não se limita a essa fronteira.
Nietzsche debate sobretudo as consequências culturais da espécie de
relação com a história que é fomentada por uma época. Não se trata,
também, como muito se pensou e sua própria narrativa em \emph{Ecce homo}
pode nos levar a crer, de uma avaliação unicamente negativa da cultura
histórica, pois para ele há também uma forma afirmativa -- e mesmo
inelutável para o homem -- de se relacionar com a história: por meio do
tempo vivido e da memória. Nesse aspecto, vemos como a crítica cultural
e a reflexão filosófica sobre a temporalidade se encontram neste livro
de forma imprevista e surpreendente.

O título \emph{Sobre a utilidade e desvantagem da história para a vida},
se bem interpretado, já nos oferece muitas pistas sobre o tema e o
objetivo do livro. Nietzsche procura realizar um estudo sobre o
historicismo, nas inflexões que lhe são contemporâneas (particularmente
o hegelianismo e o positivismo), mas, e isso é o fundamental,
identificando nessas escolas antes de tudo uma exacerbação de uma
faculdade propriamente humana, o sentido histórico, ou seja, a
capacidade de perceber e dar significado ao passado através da memória.
Dada a impossibilidade de nos desvencilharmos da história, pois o homem
é essencialmente um ser histórico (o passado e a memória fazem parte de
sua experiência no mundo), cabe saber até que ponto ela auxilia ou
prejudica a vida, vista aqui não como um conceito biológico, mas como a
experiência da vida humana, que só pode ser pensada no interior de uma
cultura, como bem observaram Martin Heidegger\footnote{Heidegger, Martin. 
\emph{Zur Auslegung von Nietzsches \versal{II}.} Unzeitgemässer Betrachtung. 
Frankfurt am Main: Klostermann, 2005.} e Volker Gerhardt.\footnote{``Leben 
und Geschichte. Menschliches Handeln und historischer Sinn in Nietzsches zweiter 
`Unzeitgemäßer Betrachtung'''. In: \emph{Pathos und Distanz.} Studien zur
  Philosophie Friedrich Nietzsches. Stuttgart: Reclam-Verlag, 1988,
  133--162.} Vemos, assim, que há um contexto filosófico e
antropológico vasto e profundo nessa compreensão da história. Do mesmo
modo, a crítica de Nietzsche ao fazer histórico de sua época não se
reduz à disciplina histórica, nem às correntes filológicas de sua
época,\footnote{Sobre uma possível relação entre as três formas de
  história e as correntes filológicas da época de Nietzsche, ver:
  Jensen, Anthony K. ``Geschichte~or~Historie? Nietzsche's
  Second Untimely Meditation in the Context of Nineteenth-Century
  Philological Studies''. In: Dries, Manuel (ed.). \emph{Nietzsche on
  Time and~History.} Berlin: Walter de Gruyter, 2008.} mas às próprias
concepções de ciência e de conhecimento que permeiam essa prática e,
mais que isso, às consequências que essa prática pode ter em toda uma
cultura. Trata-se do perigo do excesso de sentido histórico, que leva a
uma exacerbação de estudos de caráter historiográfico em todas as
disciplinas das ciências do espírito, acabando por produzir uma mera
erudição sem relação com a vida e com o impulso à ação e à renovação da
cultura: ``É apenas na medida em que a história serve à vida que
queremos a ela servir; mas existe um grau no exercício e na valorização
da história em que a vida fenece e se degenera: um fenômeno que
experimentamos agora, tão necessário quanto doloroso possa ser, como um
estranho sintoma de nossa época'' (p.\,\pageref{sintomadenossaepoca}). Essa desmedida de estudos
históricos representa, segundo Nietzsche, aquela espécie de conhecimento
que Goethe consideraria digna de ódio, por não conduzir à ação. Por
outro lado, o reconhecimento de um serviço da história para a vida
aponta para o seu aspecto afirmativo, o que será desenvolvido nos três
primeiros capítulos do livro.

O primeiro capítulo inicia-se com uma referência a um poema de Giacomo
Leopardi (``Canto noturno de um pastor da Ásia''): ao observar os
animais, o homem descobre que é incapaz de ter a felicidade deles, pois
esta repousa na inaptidão que eles têm de ter consciência do passado,
enquanto o homem não consegue se desvencilhar de suas memórias. A
história, assim, aparece como um fundamento antropológico, mas também
como causa de miséria e sofrimento humanos. Incapaz de viver
aistoricamente como o animal, o homem tem de saber refrear o seu sentido
histórico, assim como possuir um grau de uma força plástica capaz de
absorver o passado e transformar sua matéria num impulso vital, a fim de
intervir no presente e edificar um futuro. O histórico e o aistórico
são, nesse sentido, duas possibilidades fundamentais da relação humana
com a história. Para a história servir à vida é necessário que haja um
equilíbrio entre o caráter histórico da memória e o caráter aistórico do
esquecimento: ``\emph{o histórico e o aistórico são igualmente
necessários para a saúde de um indivíduo, de um povo e de uma
cultura}''\emph{.}

Além do histórico e do aistórico, há ainda uma terceira forma de se
relacionar com o passado, o ponto de vista supra-histórico: uma visão
daquilo que é eterno. Ele é alcançado quando a pesquisa histórica, ao
identificar o acaso, a insensatez e a injustiça que alimentam os
processos históricos, ao mesmo tempo que supera o ponto de vista
histórico, conduz a uma perturbadora indiferença para com o passado.
Para os homens supra-históricos, ``o passado e o presente são uma e
mesma coisa, ou seja, em toda multiplicidade, são tipicamente iguais e,
como uma onipresença de tipos perpétuos, são uma imagem paralisada de
valor invariável e de significado eternamente idêntico'' (p.\,\pageref{eternamenteidentico}). Enquanto
o aistórico é como uma névoa envolvente e obscura que permite que o
homem seja parcial e por isso se arrisque na ação, e o histórico é o
portador da esperança num futuro, o ponto de vista supra-histórico,
diferentemente, representa o abandono da própria história. O
supra-histórico despreza a história, pois nela só vê o mesmo desenrolar
vazio do acaso, não tendo assim nada a ensinar ao presente nem nada a
ofertar ao futuro. A prática histórica, como um jogo entre esses pontos
de vista, não pode seguir o modelo de uma ciência pura, como queriam os
positivistas, mas constituir-se num exercício comungado com uma forma de
vida e com um traço de caráter individual, o que determinará as três
espécies possíveis de história -- monumental, antiquária e crítica --
que serão apresentadas nos segundo e terceiro capítulos da obra.

A discussão sobre as três espécies de história possui inegável apelo,
pela riqueza conceitual e pelo estímulo à reflexão sobre os princípios,
limites e objetivos do saber histórico. Cada uma dessas formas de fazer
história corresponde a uma espécie de homem. A monumental corresponde ao
homem de ação, que precisa ver na história os grandes homens e os
grandes feitos como exemplos a serem reproduzidos; a antiquária
corresponde ao homem reverente ao passado, que cultiva uma relação de
respeito e satisfação com a história de sua nação e de seus
antepassados; a crítica, enfim, corresponde ao homem que quer se
libertar dos grilhões da tradição, negando o passado. Assim como para
cada tipo de história existem traços de caráter que lhe convém, terá cada tipo
também suas vantagens e desvantagens para a vida.

A vantagem da história monumental consiste no fato de que ela leva o
homem à ação. Compreendendo a história através dos momentos de irrupção
dos grandes nomes e das grandes realizações, ela incentiva aquele que a
cultiva a realizar grandes atos, para se tornar, ele mesmo, uma figura
histórica importante. O historiador monumental rebela-se contra a
mediocridade do presente e pretende criar algo de grandioso, já que ele
entende a história como uma ``cordilheira'' formada pelas mais elevadas
realizações. Se essa é a vantagem da história monumental, sua
desvantagem consiste em mitificar o passado, transformando-o numa
ficção. Além disso, essa história desconsidera a diversidade de causas
que conduzem a história, ao cultivar a falsa crença de que os mesmos
atos podem ser repetidos independentemente das injunções das distintas
épocas. A fim de corroborar essa crítica, o futuro arauto do eterno
retorno, para a surpresa dos pósteros, não se esquivará de ridicularizar
a versão pitagórica dessa ideia: ``No fundo, aquilo que uma vez foi
possível só poderia ocorrer uma segunda vez se os pitagóricos estivessem
certos em acreditar que, dada uma constelação idêntica de corpos
celestes, as mesmas coisas deveriam repetir-se também na Terra, nos
mínimos detalhes, de tal modo que sempre que as estrelas estiverem numa
certa posição em relação às outras um estoico e um epicurista se aliarão
e assassinarão César e, num outro arranjo, Colombo novamente descobrirá
a América (\ldots{}). Provavelmente isso não acontecerá até que os astrônomos
voltem a ser astrólogos''.

No caso da história antiquária, o respeito ao passado representa uma
conduta nobre, em contraposição a uma época que exalta o novo e despreza
o antigo, como é a Modernidade. O historiador antiquário também
fortalece os laços de união entre o indivíduo e sua nação. Satisfeito
consigo mesmo e com seu passado, ele não busca modelos no estrangeiro,
valorizando e preservando suas tradições. A desvantagem da história
antiquária revela-se no seu exagero de reverência ao passado. Esta pode
tornar-se uma veneração indiscriminada por tudo que é pretérito, sem
nenhum critério ou medida de valor do que deve ser preservado, numa
sofreguidão de tudo guardar e colecionar. Todo o passado é considerado
digno de reverência, produzindo um nivelamento de todas as experiências
e uma deturpação da história. Além disso, essa forma de fazer história
leva ao imobilismo e à inação. Jubiloso com o seu passado, o homem
antiquário não vê necessidade de interferir em sua época,
transformando-se num fantasma do passado e num ``coveiro do presente''.

Por fim, a vantagem da história crítica é não ser subserviente ao
passado de maneira alguma, o que ocorre, em graus distintos, com a
história monumental e a história antiquária. O passado, para o historiador
crítico, ao contrário, deve ser condenado e destruído, a fim de que se
possa dele se libertar. A história mostra apenas a injustiça humana, e
por isso o homem crítico precisa ser injusto com o próprio passado. Mas
a negação de toda reverência pode ter sua desvantagem: os homens de
caráter crítico são perigosos. Seu ímpeto deletério já é uma herança
preocupante da ira e da violência atávicas que ele quer negar: ``Pois lá
onde somos resultado de gerações anteriores, somos também resultado de
seus desvios, paixões, erros e até mesmo crimes; não é possível se
livrar dessa cadeia. Se condenarmos aqueles desvios e nos tomarmos como
libertos deles, isso não elimina o fato de que deles descendemos'' (p.\,\pageref{delesdescendemos}).
Com isso a história crítica se revela, também, como uma forma de
falsificar o passado.

Com a exposição das três formas de história vemos o desenrolar do jogo
entre os pontos de vistas fundamentais. Exercidas por seu homem
correspondente, cada espécie de história traz consigo suas vantagens e
desvantagens para a vida. Cada uma delas responde ao impulso histórico
quando procura uma forma de relação com o passado. E responde ao impulso
aistórico quando falsifica de alguma maneira o passado, mesmo que
inadvertidamente. Como em toda realização que tem um êxito vital, também
o próprio fazer histórico precisa do aistórico para esse fim. Não é
possível, portanto, uma objetividade histórica: onde a vida viceja, uma
dose de criação e injustiça é necessária. O sonho positivista de uma
história como ciência objetiva e neutra não é apenas uma ilusão e um
erro, é também um desserviço à própria vida. O historiador, seja ele
monumental, antiquário ou crítico, não é um espectador imparcial do
passado; seu conhecimento já é uma perspectiva que recorta os eventos
pretéritos de acordo com seu caráter, pendor e desejo. E a não ser que
se tenha o sonho positivista, para Nietzsche não se deve lamentar essa
característica falsificadora da história, desde que ela seja bem dosada
e represente o exercício daquela força plástica com que o homem se
apodera do passado em nome da ação e da vida. Mais uma vez, trata-se de
saber ponderar as características de cada forma de história, para que
suas desvantagens não superem suas vantagens. O critério básico para
isso é que os fins vitais devam conduzir o conhecimento, e não o
contrário. Se foi a história que nos tornou homens, então temos de lidar
de alguma forma com ela. Os perigos dessa empresa, no entanto, são dois.
Em primeiro lugar, que uma espécie de história não seja praticada pelo
indivíduo que não possua as aptidões necessárias: a história monumental
sem a força para realizar grandes ações, a antiquária sem a reverência à
tradição, a crítica sem a necessidade de se libertar do passado. Em
segundo lugar, que o sentido histórico se torne dominante e onipresente,
levando toda uma cultura a uma séria crise, algo que Nietzsche, como
sabemos, considera acontecer no momento em que escreve sua extemporânea.

Em sua época, diz Nietzsche, uma relação profícua entre história e vida
se tornou impossível. Isso porque a história passou a ser considerada
uma ciência positiva, ocupando-se apenas de acumular os dados os mais
variados. Essa identificação entre ``cultura'' e ``cultura histórica''
seria estranha para um grego antigo, um povo que soube preservar o
aistórico em sua vida. O homem moderno, assim, torna-se um repositório
de informações dispersas, sem lhes imprimir uma forma ou direção. Isso resulta na inflação da interioridade,
levando a um descompasso entre o interior e o exterior, entre forma e
conteúdo, que é fatal para a vida. Para quem já tinha definido a
cultura, na primeira extemporânea, ``como unidade do estilo artístico em
todas as expressões vitais de um povo'' (p.\,\pageref{vitaisdeumpovo}), essa multiplicidade
desagregada provocada pelo excesso de história só poderia conduzir a uma
cultura artificial. E entre os povos modernos, para Nietzsche o alemão é
o que mais tem uma interioridade exacerbada, com prejuízo do sentido da
forma, produzindo assim uma cultura fraca e que acaba por se tornar um
pastiche de outras culturas mais fortes. Essa visão sobre os alemães sem
dúvida vem corroborar, nessa época, a militância por uma renovação da
cultura alemã sob os auspícios da obra de arte wagneriana.

A personalidade enfraquecida pelo excesso de interioridade, se for um
erudito, passará a expressar uma forma de indiferença em relação ao seu
objeto de estudo. Se o conhecimento passa a ser concebido como a prática
de acumular dados, compará-los e criticá-los, sem estabelecer uma
relação com a vida, então o pesquisador, tornado mais um trabalhador
braçal do que um instrumento da cultura, tratará seu material como um
entre tantos outros. O trabalho intelectual, assim, torna-se uma
atividade meramente burocrática e pouco inventiva. Nietzsche, ecoando a
crítica de Schopenhauer, lamenta a redução da filosofia a mera historiografia: 
``Em que condições inaturais, artificiais e em todo caso
indignas se encontra, em uma época que padece da cultura geral, a mais
veraz de todas as ciências, a honesta e nua deusa filosofia! Ela
permanece, naquele mundo de uniformidade superficial e forçada, como um
monólogo erudito do passeante solitário, butim fortuito do indivíduo,
oculto segredo de alcova ou conversinha frívola entre criançolas e
velhotes acadêmicos. Ninguém pode ousar seguir, em si, as leis da
filosofia, ninguém vive filosoficamente, com aquela hombridade singela
que um antigo exigia de alguém que se comportasse estoicamente, onde
quer que fosse ou o que fizesse, caso já tivesse jurado lealdade ao
estoicismo'' (p.\,\pageref{lealdadeaoestoicismo}).

Ao mesmo tempo, o homem moderno, com sua cultura histórica, por ter
passado os olhos pelas mais diversas eras e acumulado conhecimento sobre
elas, tem a ilusão de por isso ser mais justo do que os homens de épocas
anteriores. Essa suposta justiça, no entanto, tem seus critérios de
avaliação questionados por Nietzsche. O homem moderno, diz ele, acaba
por adequar e julgar o passado a partir dos seus próprios valores
atuais. Assim, a cultura histórica é, ironicamente, traída por seu
anacronismo. Com seu elogio à objetividade e sua suspeita do
subjetivismo, o historiador científico se ilude de que é capaz de
reproduzir o passado fidedignamente, quando na verdade apenas reproduz o
passado a partir das ideias vigentes em sua época. O problema, mais uma
vez, não é tanto o falseamento, mas a ilusão de objetividade. Para
Nietzsche, a história deve ser escrita com os olhos no presente, como
incentivo à ação, e não apenas como a reprodução ilusoriamente objetiva
de um passado. Desse modo, o que Nietzsche espera do historiador é que
ele tenha a força da criação e a capacidade da ação. Só assim o passado
servirá para a construção do futuro.

A forma moderna de praticar a história ignora a atmosfera aistórica que
é fundamental para a constituição de um conhecimento que esteja a
serviço da vida. Com isso, ela não só não alcança a pretendida
objetividade, como também acaba por matar o que há de vida em seu
objeto. Uma religião que fosse vista apenas pelo seu aspecto histórico,
diz Nietzsche, seria o suicídio da própria fé. Ela se tornaria puro
conhecimento da religião e não o exercício de uma crença, pois perderia
sua aura sobrenatural. A história, por isso, deveria aproximar-se da
arte, deveria exercer aquela força plástica, que é uma força artística,
e se transformar em obra de arte. Para Nietzsche, isso seria a
realização da história de acordo com os instintos humanos mais naturais.
O retrato que Nietzsche faz da cultura histórica, em resumo, é a imagem
de uma erudição vazia e balofa, que acumula informações sem seleção e
que se ilude de sua própria importância, quando na verdade representa a
negação de uma verdadeira cultura.

Nos capítulos finais, Nietzsche considera como a época em que prevalece
essa cultura histórica reflete uma consciência irônica sobre si mesma: a
de que ela é incapaz de triunfar sobre si mesma e de que seu valor
reside unicamente em ser epigonal. Como marca dessa autoconsciência, que
se concebe como o final de um processo universal, Nietzsche vê em Hegel
a grande influência, e em Eduard von Hartmann sua paródia. No caso de
Hegel, Nietzsche ironiza sua concepção da história como teleologia,
assim como sua pretensão de encontrar na história leis necessárias e o
progresso do espírito, numa \emph{boutade} célebre: ``para Hegel, o
ápice e o fim último do processo universal coincidem em sua própria
existência berlinense'' (p.\,\pageref{existenciaberlinense}). 
No caso de Hartmann, seu contemporâneo, o
tratamento é mais extenso e ainda mais corrosivo.

O comentador Anthony Jensen faz o seguinte resumo da filosofia de
Hartmann: ``A partir da união do pessimismo shopenhaueriano e o
absolutismo histórico hegeliano, Hartmann argumenta que a tarefa de todo
homem, mesmo o mais triste e distante da felicidade, é inadvertidamente
fazer sua parte para facilitar esse fracionamento histórico progressivo
da ideia consciente a partir da vontade inconsciente e, nessa época em
particular, para realizar as condições para permitir o `fim
providencial' que é o niilismo cultural''.\footnote{Jensen, Anthony K.
  ``The Rogue of All Rogues: Nietzsche's Presentation of Eduard von
  Hartmann's Philosophie des Unbewussten and Hartmann's Response
  to Nietzsche''. In: \emph{Journal of Nietzsche Studies,} 32, 2006, p.
  45.} Em outras palavras, Hartmann adota uma vontade inconsciente (mas
teleologicamente orientada) que se desenvolve num processo histórico
universal até a realização do absoluto, que representa o \emph{telos}
último e tem um caráter niilista. Diante dessa filosofia, Nietzsche a
caracterizará como uma paródia da filosofia da história hegeliana, que
não se reconheceria como pândega, numa espécie de humor involuntário.
Atribuindo a Hartmann pejorativamente o epíteto ``galhofeiro dos
galhofeiros'', Nietzsche cita algumas passagens da \emph{Filosofia do
inconsciente} como exemplo de uma época senil, que se satisfaz em
compreender como o fim último da existência a mediocridade cultural e
que não vê a história como o resultado da ação humana individual 
(Hartmann teve oportunidade de se vingar posteriormente, em 1891 e em
1898, quando Nietzsche já não estava consciente, negando-lhe maior
significado na filosofia e acusando-o de plagiário de Stirner).

No último capítulo, enfim, em contraposição a essa época senil,
Nietzsche faz uma exortação à juventude, como esperança para a tão
ansiada renovação da cultura alemã. É quando ele retorna, então, ao
aspecto afirmativo da história, lembrando que caberá à juventude
libertar-se da educação histórica que lhe é impingida e praticar a
história a serviço da vida. Os antídotos para a cura da doença do
excesso de história são retomados do início do livro: os pontos de vista
aistórico e supra-histórico. O primeiro para o exercício do
esquecimento; o segundo, agora visto em seu aspecto positivo, para
desviar o olhar em direção às realizações culturais que se projetam para
a eternidade: a arte e a religião. O modelo para a realização da nova
cultura, numa época moderna e multifacetada, é o dos gregos: ``Houve
séculos em que os gregos se encontravam no mesmo perigo em que nos
encontramos, ou seja, de sucumbir na inundação do estrangeiro e do
passado, na `história'. Eles nunca viveram numa intangibilidade
orgulhosa: sua `cultura', ao contrário, foi sempre, por muito tempo, um
caos de formas e conceitos estrangeiros (\ldots{}): semelhante a como hoje a
`cultura alemã' e a religião são em si um caos de toda terra
estrangeira, de toda época anterior. (\ldots{}) Os gregos aprenderam aos
poucos a \emph{organizar o caos} ao se voltarem, segundo o ensinamento
délfico, a refletir sobre si, isto é, sobre suas necessidades autênticas
e deixar perecer as necessidades ilusórias'' (p.\,\pageref{necessidadesilusorias}).

Como se vê, Nietzsche abarca, nesta breve \emph{Consideração
extemporânea}, um grande número de temas e preocupações, cujo núcleo é a
história, não apenas e propriamente como disciplina acadêmica, mas como
fundamento antropológico e forma de vida. A maior parte do livro, e a
menos explorada, trata das várias consequências do excesso de sentido
histórico para uma cultura. Resumir suas ideias numa curta introdução
não faz justiça à complexidade e à sutileza de sua argumentação. Mas
pelo menos poderá servir para direcionar o leitor nos intricados
caminhos deste livro conceitualmente rico, esteticamente belo e
filosoficamente fecundo.
